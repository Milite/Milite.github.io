\documentclass[a4paper,11pt]{report}

\usepackage[T1]{fontenc}
\usepackage[utf8]{inputenc}
\usepackage[italian]{babel}

\usepackage{mathdots}
\usepackage{mathtools}
\usepackage{graphicx}
\usepackage{amsfonts}
\usepackage{amsthm}
\usepackage{amsmath}
\usepackage{amssymb}
\usepackage{fancyhdr}
\usepackage{float}
\usepackage{geometry}
\geometry{a4paper, top=2.5cm, bottom=2cm, left=2cm, right=2cm}
\usepackage{hyperref}
\hypersetup{
	colorlinks=true,
	linkcolor=black,
	filecolor=blue,
	citecolor = black,      
	urlcolor=cyan,
}

\swapnumbers
\theoremstyle{remark}
\newtheorem*{oss}{Oss}
\newtheorem*{nb}{N.B}
\newtheorem*{coro}{Corollario}
\theoremstyle{definition}
\newtheorem*{teo}{Teorema}
\newtheorem*{Def}{Def}

\newcommand{\C}{\mathbb{C}}
\newcommand{\R}{\mathbb{R}}
\newcommand{\angs}{\text{\AA}}

\DeclarePairedDelimiter{\abs}{\lvert}{\rvert}

\begin{document}
	\date{}
	\author{Marco Militello}
	\title{Meccanica Quantistica}
	\maketitle
	\tableofcontents
	\newpage
	
\chapter{La crisi della fisica classica}

Concetti incompatibili con la fisica classica, che non si riescono a spiegare. Per fisica classica si intende
\begin{enumerate}
    \item meccanica newtoniana
    \item termodinamica fisica statistica classica
    \item elettromagnetismo e le leggi di Maxwell
    \item relatività ristretta
\end{enumerate}

\section{Corpo nero}

Universalità: stesso spettro di emissione. Basi termodinamiche $\rightarrow$ Kirchoff. \newline
1900 Planck: giunge a risultato giusto, ma partendo da principi sbagliati
\begin{equation*}
    g(\nu, T) = \frac{8\pi h}{c^3} \frac{\nu^3}{e^{\frac{h\nu}{K_B T}}-1}
\end{equation*}
Ottiene questa relazione grazie a fit dei dati sperimentali; aggiusta i paramentri: si ottiene per la prima volta $h$, la costante di Planck
\begin{equation}
    h = 6.6 \times 10^{-34} J \cdot s
\end{equation}
ha la stessa dimesione di un momento angolare

\section{Effetto fotoelettrico}

1905 Einstein \newline
Proposta di Einstein che spiega l'effetto: scambio di energia come multiplo di $h\nu \rightarrow$ QUANTI DI RADIAZIONE. \newline
Nel grafico dell'energia massima in funzione della frequenza, Millikan misurò la pendenza della retta che risultò essere $\frac{h}{e}$
Si continuava però a pensare che l'energia si propagasse nel continuo, mentre durante le interazioni ci fossere scambi a pacchetti. Questa idea viene abbandonata con l'effetto Compton

\section{Effetto Compton}
Raggi-x su un metallo; $E_m >> E_{legame}$: in questo modo gli elettroni sono visti come liberi.
\begin{equation*}
    E^2 - p^2c^2 = m^2c^4
\end{equation*}
se v=c allora m=0. Quindi
\begin{equation*}
    p^2c^2 = E^2 = {(h\nu)}^2 \Rightarrow p=\frac{h\nu}{c}
\end{equation*}

\subsection*{Esperimento}
Tratto fotone come particella con 
$\begin{cases}
    E = h\nu \\
    p = \frac{h\nu}{c}
\end{cases}$
\begin{equation*}
    \lambda' - \lambda = \frac{h}{m_e c}(1 - \sin(\theta))
\end{equation*}
onda elettromagnetica non completamente descritta dalle leggi di Maxwell \newline
$\frac{h}{m_e c} \simeq 2.4 \times 10^{-12} m$: lunghezza d'onda di Compton per l'elettrone $\Rightarrow$ ha la dimensione di una lunghezza. \newline
1926 Lewis: assegna nome ai fotoni

\section{Spettri atomici}
Spettro solare ha delle righe nere: radiazione viene assorbita dallo strato esterno del sole e poi viene diffusa, quindi ci arriva meno intensa $\Rightarrow$ righe nere. $He$ scoperto grazie allo spettro solare. \newline
1897: scoperta elettrone \newline
Modelli atomici $\rightarrow$ Rutherford: però non riesce a spiegare spettri atomici \newline
Emissione e assorbimento non sono un continuo \newline
Bohr: Energia quantizzata
\begin{equation*}
    \nu_{mn} = (E_m - E_n)\frac{1}{h}
\end{equation*}
ma non dà alcun tipo di spiegazione
\begin{equation*}
    m_e v r = n \hslash \qquad \text{ momento angolare è multiplo intero di } \hslash = \frac{h}{2\pi}
\end{equation*}

\chapter{Interferenza e diffrazione onde elettromagnetiche}
Luce descrivibile con fenomeni ondulatori. \newline
Esperimento doppia fenditura: metto uno schermo a una distanza $L >> \lambda$ dalle fenditure. In ogni punto ho che $ I \neq I_1 + I_2$, ma ho che 
\begin{equation*}
    I = {|\vec{E}_1 + \vec{E}_2|}^2 \neq {|\vec{E}_1|}^2 + {|\vec{E}_2|}^2
\end{equation*}
La differenza di fase vale $d\sin(\theta)$; se la differenza di fase vale $n\frac{\lambda}{2}$ allora ho interferenza distruttiva. Il primo punto di buio si ha per 
\begin{equation*}
    \theta = \frac{n}{2}\frac{\lambda}{d} \text{ con } n=1
\end{equation*}
Se prendo una lampadina gialla (580 nm) da 100 W, allora vengono emessi $\sim 10^{20}$ fotoni al secondo. Cosa succede se abbasso intensità fino ad avere emissione di un fotone al secondo? L'esperimento della doppia fenditura porta allo stesso risultato precedente; però se ripeto di nuovo l'esperimento la posizione di ogni singolo fotone è diverso, anche se la figura che si viene a creare porta allo stesso risultato. Allora devo introdurre una trattazione probabilistica: posso solo dare distribuzione probabilità \newline
Se chiudo una delle due fenditure non osservo alcuna figura di interferenza: il fotone "sente" la presenza di entrambe le fenditure. Ma con chi interferisce il fotone? Da che fenditura passa? \newline
Esperimenti per capire da che fenditura sia passato il fotone distruggono la figura di interferenza $\Rightarrow$ misura microscopica disturba il fenomeno. \newline
Risultati esperimento:
\begin{enumerate}
    \item Fotone colpisce schermo in un punto ben preciso $\Rightarrow$ deposita tutta la sua energia $h\nu$
    \item Con pochi eventi lo schermo sembra riempirsi in maniera casuale; non sappiamo esattemente dove andrà un fotone, ma possiamo dire dove si sono addensati maggiormente $\Rightarrow$ PROBABILIT\'A
    \item Se chiudo una fenditura sparisce figura di interferenza: fotone con $\lambda << d$ "sente" le 2 fenditure; ogni tentativo di capire da quale fenditura passi il fotone distrugge la figura di interferenza $\rightarrow$ disturbo con la misura. La misura su un sistema microscopico lo può disturbare in maniera significativa.
\end{enumerate}
In fisica classica, date le condizioni inziali posso completamente determinare il moto di una particella; invece non posso determinare moto di un fotone: fotone ha una certa probabilità di colpire lo schermo proporzionale all'intensità I(x) [probabilità che fotone finisca in un punto preciso dello schermo]ù

\subsection*{Dualismo onda-particella}
In alcuni esperimenti è più facile interpretare come particella altre volte come onda; la vera natura è l'elettrodinamica quantistica
\begin{enumerate}
    \item La radiazione elettromagnetica si comporta come un flusso di particelle
    \item Previsioni sul comportamento sono solo probabilistiche
    \item In un certo punto $\vec{r}$ dello schermo al tempo t $\rightarrow$ fotone è portato da campo elettrico $\vec{E}(\vec{r}, t)$ che è soluzione delle equazioni di Maxwell; campo elettrico va interpretato come un'ampiezza di probabilità di trovare un fotone in un istante t in un punto $\vec{r}$
    \begin{equation*}
        {|E(\vec{r},t)|}^2 \rightarrow \text{ densità di probabilità}
    \end{equation*}
    \item Le equazioni di MAxwell sono lineari in $\vec{E} \Rightarrow$ vale principio di sovrapposizione: se $E_1, E_2$ sono soluzioni delle equazioni di Maxwell allora anche $\vec{E} = \lambda_1\vec{E}_1 + \lambda_2 \vec{E}_2$ con $\lambda_1,\lambda_2 \in \R$ è soluzione delle equazioni di Maxwell. Principio di sovrapposizione è anche quello che ci permette di spiegare interferenza
\end{enumerate}
"Ogni fotone interfesce solo con se stesso"

\chapter{La polarizzazione della luce}
La luce ha una direzione privilegiata. Es.: laser, smartphone \newline

\section*{Esperimento}
Mettere immagine \newline
Interpretazione classica: c'è onda che si propaga lungo z
\begin{equation*}
    \vec{E}(\vec{r},t) = E_0 \vec{\epsilon}_p e^{i(kz-wt)}+ c.c. \qquad k = \frac{2\pi}{\lambda} \; w = 2\pi\nu \; \lambda\nu = c
\end{equation*} 
Il campo elettrico è libero, non ci sono cariche $\Rightarrow$ $\div\vec{E}=0 \rightarrow \vec{k}\cdot \vec{\epsilon}_p =0$. Quindi trasverso rispetto alla direzione di propagazione \newline
$\vec{k} // \vec{z} \Rightarrow$ il vettore di polarizzazione appartiene al pino perpendicolare alla direzione di propagazione
\begin{equation*}
    I_0 = {|E_0|}^2
\end{equation*} 
Dopo polarizzazione: $\vec{E}'(\vec{r},t) = E_0'\vec{e}_x e^{i(kz-wt)}$
\begin{equation*}
    E_0' = E_0 cos\theta \Rightarrow I = I_0 \cos^2\theta
\end{equation*}
Dopo polarizzatore passa solo componente parallela $\rightarrow$ interpretazione che viene data. \newline
Adesso faccio esperimento in cui diminuisco intensità fascio fino a che emetta solo un fotone alla volta. Conseguenze:
\begin{enumerate}
    \item Rilevatore o vede il fotone o non lo vede
    \item Quando vede fotone lo vede tutto $\rightarrow $ tutta energia $h\nu$
    \item se $\begin{cases}
        \theta = 0 \Rightarrow \text{ rilevatore vede tutto fotone} \\
        \theta = \frac{\pi}{2} \Rightarrow \text{ rilevatore non lo vede mai}
    \end{cases}$ \newline
    La probabilità con cui vedo fotone è $\cos^2\theta \Rightarrow$ Numero fotoni arrivato $ = N_0\cos^2\theta$. Ogni singolo fotone può essere rappresentato come miscela di 2 stati
    \begin{itemize}
        \item uno stato con probabilità di passare 1: $\psi_x$
        \item uno stato con probabilità di passare 0: $\psi_y$
    \end{itemize}
    Questi valori sono mutualmente esclusivi
    \item Descrivo come somma di due stati $\psi_p = \psi_x \cos\theta + \psi_y \sin\theta \rightarrow$ inizio principio di decomposizione spettrale
    \begin{itemize}
        \item $\psi_x$ passa con probabilità ${|\cos\theta|}^2=\cos^2\theta$
        \item $\psi_y$ non passa con probabilità ${|\sin\theta|}^2=\sin^2\theta$
    \end{itemize}
    Interpretazione probabilistica è sensata perchè $\sin^2\theta + \cos^2\theta = 1$
    \item Se io ora metto un altro polarizzatore lungo x, tutti i fotoni passeranno perchè ora tutti i fotoni sono $\psi_x$: dopo polarizzatore lo stato fotone è diventato $\psi_x$. La misura ha fatto precipitare lo stato del sistema da $\psi_p$ a $\psi_x$
\end{enumerate}

\chapter{De Broglie}
Maxwell introduce andamento ondulatorio della radiazione elettromagnetico, mentre Einstein ipotizza andamento corpuscolare. De Broglie nel 1923 introduce ipotesi andamento ondulatorio elettroni. \newline
Onda $\rightarrow \exp(i \vec{k}\cdot \vec{x}- iwt)$; per essere invariante secondo Loretz $(\vec{k},w),(\vec{p},E)$ diventano tetravettori
\begin{equation*}
    |\vec{p}| = \frac{E}{c} = \frac{h\nu}{c} = \frac{h}{\lambda} = \hslash |\vec{k}| \Rightarrow \frac{h}{|\vec{p}|}
\end{equation*}
Ad ogni particella di massa è associato $\vec{k} = \frac{\vec{p}}{\hslash}$ e quindi un'onda $\lambda = \frac{h}{|\vec{p}|}$ \newline
Numericamente
\begin{enumerate}
    \item Elettroni: $m_e = 0.9 \times 10^{-30} Kg = 0.511 MeV = 0.511 \times 10^6 eV [eV = 1.6 \times 10^{-19} V\cdot J]$. Un elettrone non relativistico
    \begin{equation*}
        E = \frac{p^2}{2m} \rightarrow \lambda = \frac{h}{p} = \frac{h}{\sqrt[]{2m_eE}} = \frac{123}{\sqrt[]{Volt}}\times 10^{-10} m (\angs)
    \end{equation*}
    $\lambda \simeq 1 \angs \Rightarrow$ comparabile con raggi-x e distanza atomi cristallo
    \item Neutroni termici (bassa temperatura termica): 
    \begin{equation*}
        \lambda = \frac{h}{p} = \frac{h}{\sqrt[]{2m_nE}} = \frac{h}{\sqrt[]{2m_n \frac{3}{2}K_BT}} =\frac{30}{\sqrt[]{T (Kelvin)}}\times 10^{-10} m (\angs)
    \end{equation*}
    \item Granello polvere: $1\mu m$ di dimensione 
    \begin{equation*}
        v = 1 \frac{mm}{s} \; m = 10^{-10} Kg \Rightarrow \lambda = 6.6 \times 10^{-11} \angs 
    \end{equation*}
\end{enumerate} 
Non posso vederlo per oggetti macroscopici, ma solo a livello microscopico

\section*{Esperimento di Davisson-Germer 1927}
Fascio di elettroni su un cristallo di Nichel $\rightarrow$ ogni punto come sorgente $\Rightarrow$ nasce figura di interferenza/diffrazione 
\section*{Esperimento doppia fenditura con elettroni 1950}
\section*{Esperimento doppia fenditura con elettrone singolo 1974: Merli-Missiroli-Pozzi}
\begin{enumerate}
    \item Elettrone arriva tutto intero; la carica non si sparpaglia sullo schermo
    \item \'E un grande numero di elettroni che dà origine a figura di interferenza
    \item Arrivo sullo schermo sembra un arrivo casuale
    \item Interpreto come distribuzione di probabilità
    \item Se chiudo una fenditura $\Rightarrow$ sparisce figura interferenza
    \item Ogni tentativo di capire da quale fenditura sia passato l'elettrone distrugge la figura di intereferenza
    \item Cade il concetto di traiettoria 
\end{enumerate}
Per conservazione del momento $\rightarrow$ durante interazione con lo schermo c'è trasferimento di momento

\chapter{Il principio di indeterminazione}
Traiettoria $\rightarrow$ determinare posizione e velocità della particella istante per istante $\Rightarrow$ determino procedura di misura \newline
Per determinare da quale fenditura sia passata la mia particella pongo uno schermo (1 fenditura), ma particella forma figura di diffrazione al passaggio nella fenditura \newline
Facendo un'analisi degli ordini di grandezza, per una particella che passa da una fenditura di larghezza $\Delta y$, con una lunghezza d'onda $\lambda$, che forma un angolo $\theta$ si ha che
\begin{equation*}
    \Delta p_y \simeq p\theta = p \frac{\lambda}{\Delta y} = \frac{h}{\Delta y}
\end{equation*}
ottengo così il principio di indeterminazione di Heisenberg
\begin{equation}
    \Delta y \cdot \Delta p_y \simeq h
\end{equation}
più voglio determinare la posizione (minuisco $\Delta y$) più aumenta l'intervallo $\Delta p_y$; cade così il concetto di traiettoria. \newline
Questa è un'approssimazione: significato completamente diverso da quello che ricaveremo esattamente \newline

\noindent Principio di indeterminazione è l'effetto della perturbazione causata dallo strumento di misura $\rightarrow$ perturbazione ha ruolo essenziale in meccanica quantistica perchè ha lo stesso ordine di grandezza della mia misura
\begin{itemize}
    \item Elettrone: oltre a carica e massa ha anche un momento magnetico, che è una proprietà dell'elettrone $\Rightarrow$ spin $\left(\pm \frac{h}{2}\right)$ [elettrone come piccola calamita]
\end{itemize}

\chapter{L'equazione di Schrodinger}
\'E ricavata da ipotesi plausibili, ma non è dimostrabile; descrive fenomeni quantistici, ma non relativistici. \newline
Deve contenere: 
\begin{enumerate}
    \item Principio di indeterminazione di Heisenberg $\rightarrow$ incapacità operativa di calcolare traiettoria particella
    \item Corretto limite classico $\rightarrow$ quando posso trascurare dettagli quantistici devo riavere fisica classica
    \item Equazione lineare per poter sovrapporre soluzioni $\Rightarrow$ principio sovrapposizione. Algoritmo per costruire diffrazione (del tipo ${|\vec{E}_1 + \vec{E}_2|}^2 \neq {|\vec{E}_1|}^2 + {|\vec{E}_2|}^2$) $\rightarrow$ somma coerente di 2 oggetti $\Rightarrow {|?|}^2$: densità di probabilità
    \item Somma coerente richiede lunghezza d'onda $\rightarrow$ ipotesi di De Broglie
    \item Per somma coerente non è necessario campo vettoriale; massima semplicità: campi scalari complessi $\rightarrow$ in ogni punto ho bisogno di 2 soli valori, non di 3. Ipotesi valida perchè: ${|c_1 + c_2|}^2 \neq {|c_1|}^2 + {|c_2|}^2$
\end{enumerate}
Costruzione\dots
\begin{itemize}
    \item Particella libera in 1 dimensione
    \item f(x): campo scalare complesso
    \item ${|f(x)|}^2 \, dx = dP  \rightarrow$ probabilità infinitesima di trovare particella tra x e x + dx $\Rightarrow {|f(x)|}^2$: densità di probabilità
    \item $\int dP = 1 \rightarrow$ postulato probabilità
    \begin{equation*}
        \int_{-\infty}^\infty dx \, {|f(x)|}^2 =1 
    \end{equation*}
    \item f(x) ampiezza di probabilità (campo quantistico)
    \item Rappresento particella come pacchetto d'onde molto localizzato; ambito naturale è trasformata di Fourier
    \begin{equation*}
        f(x) = \int_{-\infty}^\infty dk \, g(k)e^{ikx}
    \end{equation*}
    Esempio uso gaussiana $g(k) = e^{\alpha{(k-k_0)}^2}$ che ha come ordine di grandezza della larghezza $\frac{1}{\sqrt[]{\alpha}}$
    \begin{equation*}
        f(x) = \sqrt[]{\frac{\pi}{\alpha}}e^{ik_0x}e^{-\frac{x^2}{4\alpha}} \qquad {|f(x)|}^2 = \frac{\pi}{\alpha} e^{-\frac{x^2}{2\alpha}}
    \end{equation*}
    Posso calcolare 
    \begin{equation*}
        \begin{cases}
            <x> = \int dx \, xf(x) \\
            \Delta x^2 = \int dx \, {(x-<x>)}^2f(x)
        \end{cases}
    \end{equation*}
    ottenendo che $\Delta x= \sqrt[]{\alpha}$ e $\Delta k = \frac{1}{\sqrt[]{2}\sqrt[]{\alpha}}$ e quindi che
    \begin{equation*}
        \Delta x \Delta k = \frac{1}{\sqrt[]{2}}
    \end{equation*}
    In generale si può dimostare che per qualunque funzione vale che 
    \begin{equation*}
        \Delta x \Delta k \geq \frac{1}{2}
    \end{equation*}
    \item Se voglio che sia valido De Broglie chiamo: $\hslash k = p \rightarrow k = \frac{p}{\hslash}$
    \begin{equation*}
        \Delta x \Delta p \geq \hslash
    \end{equation*}
    che ci rimanda al principio di indeterminazione: la trasformata di Fourier ci dà informazioni sui momenti
    \item Tutto qeusto è valido per una particella ad un istante $t = \bar{t} = 0$; devo inserire il tempo: per farlo uso la condizione che il baricentro del pacchetto d'onda soffisfi equazione di Newton $\rightarrow$ posizione classica della particella è il baricentro <x> del pacchetto d'onda
    \item Per un'onda piana qualsiasi vale che $e^{i(kx - w(k)t)}$, ma per le onde elettromagnetiche nel vuoto $w(k) = 2\pi \nu = 2\pi \frac{c}{\lambda} = kc$, quindi
    \begin{equation*}
        e^{ik(x-ct)}
    \end{equation*}
    ogni singola onda piana si muove con velocità c. Per soddisfare equazioni di Maxwell $f(x,t) \mapsto f(x-ct)$
    \begin{equation*}
        f(x,t) = \int_{-\infty}^\infty dk \, g(k) e^{i(kx-w(k)t)}
    \end{equation*}
    espandendo attorno a $k_0 \rightarrow w(k)= w(k_0) + (k-k_0)\left(\frac{dw}{dk}\right)_{k_0} + \frac{1}{2} {(k-k_0)}^2 \left(\frac{d^2w}{dk^2}\right)_{k_0} + \dots $, ponendo $k' = k -k_0$
    \begin{equation*}
        f(x,t) = e^{i(k_0x - w(k_0)t)} \int dk' \, g(k')e^{i(x-v_gt)}e^{i\beta t k'^2} \qquad v_g = \left(\frac{dw}{dk}\right)_{k_0} \; \beta = \left(\frac{d^2w}{dk^2}\right)_{k_0}
    \end{equation*}
    $v_g$: velocità di gruppo con cui si propaga il centro del pacchetto d'onda. \newline
    I pacchetti sono centrati $\rightarrow p_0 = \hslash k_0$
    \item Impongo equazioni di Newton
    \begin{equation*}
        v_g = \frac{p_0}{m} = \frac{\hslash k_0}{m} \Rightarrow \left(\frac{dw}{dk}\right)_{k_0} = \frac{\hslash k_0}{m}
    \end{equation*}
    integrando si ottiene la relazione di dispersione per una particella libera
    \begin{equation*}
        w = \frac{\hslash}{m}\frac{k^2}{2} = \frac{1}{2m\hslash}p^2
    \end{equation*}
    mettendo insieme
    \begin{equation*}
        f(x,t) = \int dk \, g(k) e^{i\left(kx - \frac{\hslash}{2m}k^2t\right)}\underbrace{=}_{p=\hslash k} \frac{1}{\hslash}\int dp \, g(p)e^{\frac{i}{\hslash}\left(px - \frac{p^2}{2m}t\right)} \underbrace{=}_{E=\frac{p^2}{m}} \frac{1}{\hslash}\int dp \, g(p)e^{\frac{i}{\hslash}(px - Et)}
    \end{equation*}
    \begin{equation*}
        \psi (x,t) = \int dp \, \phi(p) e^{\frac{i}{\hslash}(px - Et)}
    \end{equation*}
    \item Impongo condizione che 
    \begin{equation*}
        \int_{-\infty}^\infty dV \, {|\psi(x,t)|}^2 = 1 
    \end{equation*}
    \item Cerco equazione lineare più semplice che soddisfa $\psi$:
    \begin{itemize}
        \item $\frac{\delta \psi}{\delta t} = \int dp \, \Phi (p) \left(-\frac{i}{\hslash}\frac{p^2}{2m}\right)e^{\frac{i}{\hslash}(px-Et)}$
        \item  $\frac{\delta \psi}{\delta z} = \int dp \, \Phi (p) \frac{i}{\hslash}p e^{\frac{i}{\hslash}(px-Et)}$
        \item $\frac{\delta^2 \psi}{\delta x^2} = \int dp \, \Phi (p) {\left(-\frac{i}{\hslash}\right)}^2p^2 e^{\frac{i}{\hslash}(px-Et)}$
    \end{itemize}
    \begin{equation*}
        i\hslash \frac{\delta \psi}{\delta t} = \frac{1}{2m} \int dp \, \Phi (p) p^2 e{\frac{i}{\hslash}(px-Et)} = -\frac{\hslash^2}{2m} \frac{\delta^2 \psi}{\delta x^2}
    \end{equation*}
    EQUAZIONI DI SCHRODINGER di 1 particella libera in una dimensione 
    \begin{equation}
        i\hslash \frac{\delta \psi}{\delta t} = -\frac{\hslash^2}{2m} \frac{\delta^2 \psi}{\delta x^2}
    \end{equation}
\end{itemize}

\end{document}