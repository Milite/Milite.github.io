\documentclass[a4paper,11pt]{report}

\usepackage[T1]{fontenc}
\usepackage[utf8]{inputenc}
\usepackage[italian]{babel}

\usepackage{mathdots}
\usepackage{mathtools}
\usepackage{graphicx}
\usepackage{amsfonts}
\usepackage{amsthm}
\usepackage{amsmath}
\usepackage{amssymb}
\usepackage{fancyhdr}
\usepackage{float}
\usepackage{geometry}
\geometry{a4paper, top=2.5cm, bottom=2cm, left=2cm, right=2cm}
\usepackage{hyperref}
\hypersetup{
	colorlinks=true,
	linkcolor=black,
	filecolor=blue,
	citecolor = black,      
	urlcolor=cyan,
}

\swapnumbers
\theoremstyle{remark}
\newtheorem*{oss}{Oss}
\newtheorem*{nb}{N.B}
\newtheorem*{coro}{Corollario}
\theoremstyle{definition}
\newtheorem*{teo}{Teorema}
\newtheorem*{Def}{Def}

\newcommand{\C}{\mathbb{C}}
\newcommand{\R}{\mathbb{R}}
\newcommand{\angs}{\text{\AA}}
\newcommand{\x}{\hat{x}}
\newcommand{\p}{\hat{p}}

\DeclarePairedDelimiter{\abs}{\lvert}{\rvert}

\begin{document}
	\date{}
	\author{Marco Militello}
	\title{Meccanica Quantistica}
	\maketitle
	\tableofcontents
	\newpage
	
\chapter{La crisi della fisica classica}

Concetti incompatibili con la fisica classica, che non si riescono a spiegare. Per fisica classica si intende
\begin{enumerate}
    \item meccanica newtoniana
    \item termodinamica fisica statistica classica
    \item elettromagnetismo e le leggi di Maxwell
    \item relatività ristretta
\end{enumerate}

\section{Corpo nero}

Universalità: stesso spettro di emissione. Basi termodinamiche $\rightarrow$ Kirchoff. \newline
1900 Planck: giunge a risultato giusto, ma partendo da principi sbagliati
\begin{equation*}
    g(\nu, T) = \frac{8\pi h}{c^3} \frac{\nu^3}{e^{\frac{h\nu}{K_B T}}-1}
\end{equation*}
Ottiene questa relazione grazie a fit dei dati sperimentali; aggiusta i paramentri: si ottiene per la prima volta $h$, la costante di Planck
\begin{equation}
    h = 6.6 \times 10^{-34} J \cdot s
\end{equation}
ha la stessa dimesione di un momento angolare

\section{Effetto fotoelettrico}

1905 Einstein \newline
Proposta di Einstein che spiega l'effetto: scambio di energia come multiplo di $h\nu \rightarrow$ QUANTI DI RADIAZIONE. \newline
Nel grafico dell'energia massima in funzione della frequenza, Millikan misurò la pendenza della retta che risultò essere $\frac{h}{e}$
Si continuava però a pensare che l'energia si propagasse nel continuo, mentre durante le interazioni ci fossere scambi a pacchetti. Questa idea viene abbandonata con l'effetto Compton

\section{Effetto Compton}
Raggi-x su un metallo; $E_m >> E_{legame}$: in questo modo gli elettroni sono visti come liberi.
\begin{equation*}
    E^2 - p^2c^2 = m^2c^4
\end{equation*}
se v=c allora m=0. Quindi
\begin{equation*}
    p^2c^2 = E^2 = {(h\nu)}^2 \Rightarrow p=\frac{h\nu}{c}
\end{equation*}

\subsection*{Esperimento}
Tratto fotone come particella con 
$\begin{cases}
    E = h\nu \\
    p = \frac{h\nu}{c}
\end{cases}$
\begin{equation*}
    \lambda' - \lambda = \frac{h}{m_e c}(1 - \sin(\theta))
\end{equation*}
onda elettromagnetica non completamente descritta dalle leggi di Maxwell \newline
$\frac{h}{m_e c} \simeq 2.4 \times 10^{-12} m$: lunghezza d'onda di Compton per l'elettrone $\Rightarrow$ ha la dimensione di una lunghezza. \newline
1926 Lewis: assegna nome ai fotoni

\section{Spettri atomici}
Spettro solare ha delle righe nere: radiazione viene assorbita dallo strato esterno del sole e poi viene diffusa, quindi ci arriva meno intensa $\Rightarrow$ righe nere. $He$ scoperto grazie allo spettro solare. \newline
1897: scoperta elettrone \newline
Modelli atomici $\rightarrow$ Rutherford: però non riesce a spiegare spettri atomici \newline
Emissione e assorbimento non sono un continuo \newline
Bohr: Energia quantizzata
\begin{equation*}
    \nu_{mn} = (E_m - E_n)\frac{1}{h}
\end{equation*}
ma non dà alcun tipo di spiegazione
\begin{equation*}
    m_e v r = n \hslash \qquad \text{ momento angolare è multiplo intero di } \hslash = \frac{h}{2\pi}
\end{equation*}

\chapter{Interferenza e diffrazione onde elettromagnetiche}
Luce descrivibile con fenomeni ondulatori. \newline
Esperimento doppia fenditura: metto uno schermo a una distanza $L >> \lambda$ dalle fenditure. In ogni punto ho che $ I \neq I_1 + I_2$, ma ho che 
\begin{equation*}
    I = {|\vec{E}_1 + \vec{E}_2|}^2 \neq {|\vec{E}_1|}^2 + {|\vec{E}_2|}^2
\end{equation*}
La differenza di fase vale $d\sin(\theta)$; se la differenza di fase vale $n\frac{\lambda}{2}$ allora ho interferenza distruttiva. Il primo punto di buio si ha per 
\begin{equation*}
    \theta = \frac{n}{2}\frac{\lambda}{d} \text{ con } n=1
\end{equation*}
Se prendo una lampadina gialla (580 nm) da 100 W, allora vengono emessi $\sim 10^{20}$ fotoni al secondo. Cosa succede se abbasso intensità fino ad avere emissione di un fotone al secondo? L'esperimento della doppia fenditura porta allo stesso risultato precedente; però se ripeto di nuovo l'esperimento la posizione di ogni singolo fotone è diverso, anche se la figura che si viene a creare porta allo stesso risultato. Allora devo introdurre una trattazione probabilistica: posso solo dare distribuzione probabilità \newline
Se chiudo una delle due fenditure non osservo alcuna figura di interferenza: il fotone "sente" la presenza di entrambe le fenditure. Ma con chi interferisce il fotone? Da che fenditura passa? \newline
Esperimenti per capire da che fenditura sia passato il fotone distruggono la figura di interferenza $\Rightarrow$ misura microscopica disturba il fenomeno. \newline
Risultati esperimento:
\begin{enumerate}
    \item Fotone colpisce schermo in un punto ben preciso $\Rightarrow$ deposita tutta la sua energia $h\nu$
    \item Con pochi eventi lo schermo sembra riempirsi in maniera casuale; non sappiamo esattemente dove andrà un fotone, ma possiamo dire dove si sono addensati maggiormente $\Rightarrow$ PROBABILIT\'A
    \item Se chiudo una fenditura sparisce figura di interferenza: fotone con $\lambda << d$ "sente" le 2 fenditure; ogni tentativo di capire da quale fenditura passi il fotone distrugge la figura di interferenza $\rightarrow$ disturbo con la misura. La misura su un sistema microscopico lo può disturbare in maniera significativa.
\end{enumerate}
In fisica classica, date le condizioni inziali posso completamente determinare il moto di una particella; invece non posso determinare moto di un fotone: fotone ha una certa probabilità di colpire lo schermo proporzionale all'intensità I(x) [probabilità che fotone finisca in un punto preciso dello schermo]ù

\subsection*{Dualismo onda-particella}
In alcuni esperimenti è più facile interpretare come particella altre volte come onda; la vera natura è l'elettrodinamica quantistica
\begin{enumerate}
    \item La radiazione elettromagnetica si comporta come un flusso di particelle
    \item Previsioni sul comportamento sono solo probabilistiche
    \item In un certo punto $\vec{r}$ dello schermo al tempo t $\rightarrow$ fotone è portato da campo elettrico $\vec{E}(\vec{r}, t)$ che è soluzione delle equazioni di Maxwell; campo elettrico va interpretato come un'ampiezza di probabilità di trovare un fotone in un istante t in un punto $\vec{r}$
    \begin{equation*}
        {|E(\vec{r},t)|}^2 \rightarrow \text{ densità di probabilità}
    \end{equation*}
    \item Le equazioni di MAxwell sono lineari in $\vec{E} \Rightarrow$ vale principio di sovrapposizione: se $E_1, E_2$ sono soluzioni delle equazioni di Maxwell allora anche $\vec{E} = \lambda_1\vec{E}_1 + \lambda_2 \vec{E}_2$ con $\lambda_1,\lambda_2 \in \R$ è soluzione delle equazioni di Maxwell. Principio di sovrapposizione è anche quello che ci permette di spiegare interferenza
\end{enumerate}
"Ogni fotone interfesce solo con se stesso"

\chapter{La polarizzazione della luce}
La luce ha una direzione privilegiata. Es.: laser, smartphone \newline

\section*{Esperimento}
Mettere immagine \newline
Interpretazione classica: c'è onda che si propaga lungo z
\begin{equation*}
    \vec{E}(\vec{r},t) = E_0 \vec{\epsilon}_p e^{i(kz-wt)}+ c.c. \qquad k = \frac{2\pi}{\lambda} \; w = 2\pi\nu \; \lambda\nu = c
\end{equation*} 
Il campo elettrico è libero, non ci sono cariche $\Rightarrow$ $\div\vec{E}=0 \rightarrow \vec{k}\cdot \vec{\epsilon}_p =0$. Quindi trasverso rispetto alla direzione di propagazione \newline
$\vec{k} // \vec{z} \Rightarrow$ il vettore di polarizzazione appartiene al pino perpendicolare alla direzione di propagazione
\begin{equation*}
    I_0 = {|E_0|}^2
\end{equation*} 
Dopo polarizzazione: $\vec{E}'(\vec{r},t) = E_0'\vec{e}_x e^{i(kz-wt)}$
\begin{equation*}
    E_0' = E_0 cos\theta \Rightarrow I = I_0 \cos^2\theta
\end{equation*}
Dopo polarizzatore passa solo componente parallela $\rightarrow$ interpretazione che viene data. \newline
Adesso faccio esperimento in cui diminuisco intensità fascio fino a che emetta solo un fotone alla volta. Conseguenze:
\begin{enumerate}
    \item Rilevatore o vede il fotone o non lo vede
    \item Quando vede fotone lo vede tutto $\rightarrow $ tutta energia $h\nu$
    \item se $\begin{cases}
        \theta = 0 \Rightarrow \text{ rilevatore vede tutto fotone} \\
        \theta = \frac{\pi}{2} \Rightarrow \text{ rilevatore non lo vede mai}
    \end{cases}$ \newline
    La probabilità con cui vedo fotone è $\cos^2\theta \Rightarrow$ Numero fotoni arrivato $ = N_0\cos^2\theta$. Ogni singolo fotone può essere rappresentato come miscela di 2 stati
    \begin{itemize}
        \item uno stato con probabilità di passare 1: $\psi_x$
        \item uno stato con probabilità di passare 0: $\psi_y$
    \end{itemize}
    Questi valori sono mutualmente esclusivi
    \item Descrivo come somma di due stati $\psi_p = \psi_x \cos\theta + \psi_y \sin\theta \rightarrow$ inizio principio di decomposizione spettrale
    \begin{itemize}
        \item $\psi_x$ passa con probabilità ${|\cos\theta|}^2=\cos^2\theta$
        \item $\psi_y$ non passa con probabilità ${|\sin\theta|}^2=\sin^2\theta$
    \end{itemize}
    Interpretazione probabilistica è sensata perchè $\sin^2\theta + \cos^2\theta = 1$
    \item Se io ora metto un altro polarizzatore lungo x, tutti i fotoni passeranno perchè ora tutti i fotoni sono $\psi_x$: dopo polarizzatore lo stato fotone è diventato $\psi_x$. La misura ha fatto precipitare lo stato del sistema da $\psi_p$ a $\psi_x$
\end{enumerate}

\chapter{De Broglie}
Maxwell introduce andamento ondulatorio della radiazione elettromagnetico, mentre Einstein ipotizza andamento corpuscolare. De Broglie nel 1923 introduce ipotesi andamento ondulatorio elettroni. \newline
Onda $\rightarrow \exp(i \vec{k}\cdot \vec{x}- iwt)$; per essere invariante secondo Loretz $(\vec{k},w),(\vec{p},E)$ diventano tetravettori
\begin{equation*}
    |\vec{p}| = \frac{E}{c} = \frac{h\nu}{c} = \frac{h}{\lambda} = \hslash |\vec{k}| \Rightarrow \frac{h}{|\vec{p}|}
\end{equation*}
Ad ogni particella di massa è associato $\vec{k} = \frac{\vec{p}}{\hslash}$ e quindi un'onda $\lambda = \frac{h}{|\vec{p}|}$ \newline
Numericamente
\begin{enumerate}
    \item Elettroni: $m_e = 0.9 \times 10^{-30} Kg = 0.511 MeV = 0.511 \times 10^6 eV [eV = 1.6 \times 10^{-19} V\cdot J]$. Un elettrone non relativistico
    \begin{equation*}
        E = \frac{p^2}{2m} \rightarrow \lambda = \frac{h}{p} = \frac{h}{\sqrt[]{2m_eE}} = \frac{123}{\sqrt[]{Volt}}\times 10^{-10} m (\angs)
    \end{equation*}
    $\lambda \simeq 1 \angs \Rightarrow$ comparabile con raggi-x e distanza atomi cristallo
    \item Neutroni termici (bassa temperatura termica): 
    \begin{equation*}
        \lambda = \frac{h}{p} = \frac{h}{\sqrt[]{2m_nE}} = \frac{h}{\sqrt[]{2m_n \frac{3}{2}K_BT}} =\frac{30}{\sqrt[]{T (Kelvin)}}\times 10^{-10} m (\angs)
    \end{equation*}
    \item Granello polvere: $1\mu m$ di dimensione 
    \begin{equation*}
        v = 1 \frac{mm}{s} \; m = 10^{-10} Kg \Rightarrow \lambda = 6.6 \times 10^{-11} \angs 
    \end{equation*}
\end{enumerate} 
Non posso vederlo per oggetti macroscopici, ma solo a livello microscopico

\section*{Esperimento di Davisson-Germer 1927}
Fascio di elettroni su un cristallo di Nichel $\rightarrow$ ogni punto come sorgente $\Rightarrow$ nasce figura di interferenza/diffrazione 
\section*{Esperimento doppia fenditura con elettroni 1950}
\section*{Esperimento doppia fenditura con elettrone singolo 1974: Merli-Missiroli-Pozzi}
\begin{enumerate}
    \item Elettrone arriva tutto intero; la carica non si sparpaglia sullo schermo
    \item \'E un grande numero di elettroni che dà origine a figura di interferenza
    \item Arrivo sullo schermo sembra un arrivo casuale
    \item Interpreto come distribuzione di probabilità
    \item Se chiudo una fenditura $\Rightarrow$ sparisce figura interferenza
    \item Ogni tentativo di capire da quale fenditura sia passato l'elettrone distrugge la figura di intereferenza
    \item Cade il concetto di traiettoria 
\end{enumerate}
Per conservazione del momento $\rightarrow$ durante interazione con lo schermo c'è trasferimento di momento

\chapter{Il principio di indeterminazione}
Traiettoria $\rightarrow$ determinare posizione e velocità della particella istante per istante $\Rightarrow$ determino procedura di misura \newline
Per determinare da quale fenditura sia passata la mia particella pongo uno schermo (1 fenditura), ma particella forma figura di diffrazione al passaggio nella fenditura \newline
Facendo un'analisi degli ordini di grandezza, per una particella che passa da una fenditura di larghezza $\Delta y$, con una lunghezza d'onda $\lambda$, che forma un angolo $\theta$ si ha che
\begin{equation*}
    \Delta p_y \simeq p\theta = p \frac{\lambda}{\Delta y} = \frac{h}{\Delta y}
\end{equation*}
ottengo così il principio di indeterminazione di Heisenberg
\begin{equation}
    \Delta y \cdot \Delta p_y \simeq h
\end{equation}
più voglio determinare la posizione (minuisco $\Delta y$) più aumenta l'intervallo $\Delta p_y$; cade così il concetto di traiettoria. \newline
Questa è un'approssimazione: significato completamente diverso da quello che ricaveremo esattamente \newline

\noindent Principio di indeterminazione è l'effetto della perturbazione causata dallo strumento di misura $\rightarrow$ perturbazione ha ruolo essenziale in meccanica quantistica perchè ha lo stesso ordine di grandezza della mia misura
\begin{itemize}
    \item Elettrone: oltre a carica e massa ha anche un momento magnetico, che è una proprietà dell'elettrone $\Rightarrow$ spin $\left(\pm \frac{h}{2}\right)$ [elettrone come piccola calamita]
\end{itemize}

\chapter{L'equazione di Schrodinger}
\'E ricavata da ipotesi plausibili, ma non è dimostrabile; descrive fenomeni quantistici, ma non relativistici. \newline
Deve contenere: 
\begin{enumerate}
    \item Principio di indeterminazione di Heisenberg $\rightarrow$ incapacità operativa di calcolare traiettoria particella
    \item Corretto limite classico $\rightarrow$ quando posso trascurare dettagli quantistici devo riavere fisica classica
    \item Equazione lineare per poter sovrapporre soluzioni $\Rightarrow$ principio sovrapposizione. Algoritmo per costruire diffrazione (del tipo ${|\vec{E}_1 + \vec{E}_2|}^2 \neq {|\vec{E}_1|}^2 + {|\vec{E}_2|}^2$) $\rightarrow$ somma coerente di 2 oggetti $\Rightarrow {|?|}^2$: densità di probabilità
    \item Somma coerente richiede lunghezza d'onda $\rightarrow$ ipotesi di De Broglie
    \item Per somma coerente non è necessario campo vettoriale; massima semplicità: campi scalari complessi $\rightarrow$ in ogni punto ho bisogno di 2 soli valori, non di 3. Ipotesi valida perchè: ${|c_1 + c_2|}^2 \neq {|c_1|}^2 + {|c_2|}^2$
\end{enumerate}

\section{Equazione di Schrodinger per una particella libera}
Costruzione\dots
\begin{itemize}
    \item Particella libera in 1 dimensione
    \item f(x): campo scalare complesso
    \item ${|f(x)|}^2 \, dx = dP  \rightarrow$ probabilità infinitesima di trovare particella tra x e x + dx $\Rightarrow {|f(x)|}^2$: densità di probabilità
    \item $\int dP = 1 \rightarrow$ postulato probabilità
    \begin{equation*}
        \int_{-\infty}^\infty dx \, {|f(x)|}^2 =1 
    \end{equation*}
    \item f(x) ampiezza di probabilità (campo quantistico)
    \item Rappresento particella come pacchetto d'onde molto localizzato; ambito naturale è trasformata di Fourier
    \begin{equation*}
        f(x) = \int_{-\infty}^\infty dk \, g(k)e^{ikx}
    \end{equation*}
    Esempio uso gaussiana $g(k) = e^{\alpha{(k-k_0)}^2}$ che ha come ordine di grandezza della larghezza $\frac{1}{\sqrt[]{\alpha}}$
    \begin{equation*}
        f(x) = \sqrt[]{\frac{\pi}{\alpha}}e^{ik_0x}e^{-\frac{x^2}{4\alpha}} \qquad {|f(x)|}^2 = \frac{\pi}{\alpha} e^{-\frac{x^2}{2\alpha}}
    \end{equation*}
    Posso calcolare 
    \begin{equation*}
        \begin{cases}
            <x> = \int dx \, xf(x) \\
            \Delta x^2 = \int dx \, {(x-<x>)}^2f(x)
        \end{cases}
    \end{equation*}
    ottenendo che $\Delta x= \sqrt[]{\alpha}$ e $\Delta k = \frac{1}{\sqrt[]{2}\sqrt[]{\alpha}}$ e quindi che
    \begin{equation*}
        \Delta x \Delta k = \frac{1}{\sqrt[]{2}}
    \end{equation*}
    In generale si può dimostare che per qualunque funzione vale che 
    \begin{equation*}
        \Delta x \Delta k \geq \frac{1}{2}
    \end{equation*}
    \item Se voglio che sia valido De Broglie chiamo: $\hslash k = p \rightarrow k = \frac{p}{\hslash}$
    \begin{equation*}
        \Delta x \Delta p \geq \hslash
    \end{equation*}
    che ci rimanda al principio di indeterminazione: la trasformata di Fourier ci dà informazioni sui momenti
    \item Tutto qeusto è valido per una particella ad un istante $t = \bar{t} = 0$; devo inserire il tempo: per farlo uso la condizione che il baricentro del pacchetto d'onda soffisfi equazione di Newton $\rightarrow$ posizione classica della particella è il baricentro <x> del pacchetto d'onda
    \item Per un'onda piana qualsiasi vale che $e^{i(kx - w(k)t)}$, ma per le onde elettromagnetiche nel vuoto $w(k) = 2\pi \nu = 2\pi \frac{c}{\lambda} = kc$, quindi
    \begin{equation*}
        e^{ik(x-ct)}
    \end{equation*}
    ogni singola onda piana si muove con velocità c. Per soddisfare equazioni di Maxwell $f(x,t) \mapsto f(x-ct)$
    \begin{equation*}
        f(x,t) = \int_{-\infty}^\infty dk \, g(k) e^{i(kx-w(k)t)}
    \end{equation*}
    espandendo attorno a $k_0 \rightarrow w(k)= w(k_0) + (k-k_0)\left(\frac{dw}{dk}\right)_{k_0} + \frac{1}{2} {(k-k_0)}^2 \left(\frac{d^2w}{dk^2}\right)_{k_0} + \dots $, ponendo $k' = k -k_0$
    \begin{equation*}
        f(x,t) = e^{i(k_0x - w(k_0)t)} \int dk' \, g(k')e^{i(x-v_gt)}e^{i\beta t k'^2} \qquad v_g = \left(\frac{dw}{dk}\right)_{k_0} \; \beta = \left(\frac{d^2w}{dk^2}\right)_{k_0}
    \end{equation*}
    $v_g$: velocità di gruppo con cui si propaga il centro del pacchetto d'onda. \newline
    I pacchetti sono centrati $\rightarrow p_0 = \hslash k_0$
    \item Impongo equazioni di Newton
    \begin{equation*}
        v_g = \frac{p_0}{m} = \frac{\hslash k_0}{m} \Rightarrow \left(\frac{dw}{dk}\right)_{k_0} = \frac{\hslash k_0}{m}
    \end{equation*}
    integrando si ottiene la relazione di dispersione per una particella libera
    \begin{equation*}
        w = \frac{\hslash}{m}\frac{k^2}{2} = \frac{1}{2m\hslash}p^2
    \end{equation*}
    mettendo insieme
    \begin{equation*}
        f(x,t) = \int dk \, g(k) e^{i\left(kx - \frac{\hslash}{2m}k^2t\right)}\underbrace{=}_{p=\hslash k} \frac{1}{\hslash}\int dp \, g(p)e^{\frac{i}{\hslash}\left(px - \frac{p^2}{2m}t\right)} \underbrace{=}_{E=\frac{p^2}{m}} \frac{1}{\hslash}\int dp \, g(p)e^{\frac{i}{\hslash}(px - Et)}
    \end{equation*}
    \begin{equation*}
        \psi (x,t) = \int dp \, \phi(p) e^{\frac{i}{\hslash}(px - Et)}
    \end{equation*}
    \item Impongo condizione che 
    \begin{equation*}
        \int_{-\infty}^\infty dV \, {|\psi(x,t)|}^2 = 1 
    \end{equation*}
    \item Cerco equazione lineare più semplice che soddisfa $\psi$:
    \begin{itemize}
        \item $\frac{\delta \psi}{\delta t} = \int dp \, \Phi (p) \left(-\frac{i}{\hslash}\frac{p^2}{2m}\right)e^{\frac{i}{\hslash}(px-Et)}$
        \item  $\frac{\delta \psi}{\delta z} = \int dp \, \Phi (p) \frac{i}{\hslash}p e^{\frac{i}{\hslash}(px-Et)}$
        \item $\frac{\delta^2 \psi}{\delta x^2} = \int dp \, \Phi (p) {\left(-\frac{i}{\hslash}\right)}^2p^2 e^{\frac{i}{\hslash}(px-Et)}$
    \end{itemize}
    \begin{equation*}
        i\hslash \frac{\delta \psi}{\delta t} = \frac{1}{2m} \int dp \, \Phi (p) p^2 e{\frac{i}{\hslash}(px-Et)} = -\frac{\hslash^2}{2m} \frac{\delta^2 \psi}{\delta x^2}
    \end{equation*}
    EQUAZIONI DI SCHRODINGER di 1 particella libera in una dimensione 
    \begin{equation}
        i\hslash \frac{\delta \psi}{\delta t} = -\frac{\hslash^2}{2m} \frac{\delta^2 \psi}{\delta x^2}
    \end{equation}
\end{itemize}
Conclusioni:
\begin{itemize}
    \item equazione lineare del secondo ordine nelle coordinate spaziali e del primo ordine in quelle temporali
    \item non è un'equazione relativistica
    \item $\psi$ è una funzione complessa
\end{itemize}
Commenti gaussiana centrata in $\frac{\hslash k_0}{m}t$
\begin{enumerate}
    \item Centro si sposta con velocità $v_g = \frac{\hslash k_0}{m}$
    \item La larghezza del pacchetto 
    \begin{equation*}
        \Delta x = \frac{\alpha}{2}\sqrt[]{1 + \frac{4\hslash^2 t^2}{m^2\alpha^4}}
    \end{equation*}
    aumenta con il tempo; con il passare del tempo è sempre più difficile determinare la posizione 
    \item L'allargamento  di $\Delta x$ per la particella libera non è una caratteristica del solo pacchetto gaussiano: per ogni distribuzione iniziale, se faccio passare un tempo sufficientemente lungo, la larghezza del pacchetto cresce
    \item La distribuzione dei momenti non cambia. $e^{-\frac{\alpha^2}{4}(k-k_0)}$ rimane lo stesso, compreso il valore centrale dei momenti $\hslash k_0$. La distribuzione dei momenti non cambia perchè pacchetto è libero, non ci sono forze $\rightarrow$ MOMENTO CONSERVATO 
\end{enumerate}

\section{Equazione di Schrodinger per un potenziale arbitrario V(x)}

Per particella libera: $E = \frac{p^2}{2m} \rightarrow
\begin{cases}
    i\hslash \frac{\delta \psi}{\delta t} = \int dp \, \varPhi(p)E e^{\frac{i}{\hslash}(px-Et)} \\
    i\frac{\hslash^2}{2m} \frac{\delta^2 \psi}{\delta t^2} = \int dp \, \varPhi(p)E e^{\frac{i}{\hslash}(px-Et)} \\
\end{cases}$
Se ho un potenziale V(x) l'energia vale
\begin{equation*}
    E = \frac{p^2}{2m} + V(x)
\end{equation*}
se moltiplico entrambi i membri per $\int dp \, \varPhi(p)E e^{\frac{i}{\hslash}(px-Et)}$ ottengo
\begin{equation*}
    i\hslash \frac{\delta \psi}{\delta t} = \left[-\frac{\hslash^2}{2m}\frac{\delta^2}{\delta x^2} + V(x)\right]\psi
\end{equation*}
Se generalizzo in 3 dimensioni ottengo l'equazione di Schrodinger in 3 dimensioni
\begin{equation*}
    i\hslash \frac{\delta}{\delta t} \psi(t,\vec{x}) = -\frac{\hslash^2}{2m}\nabla^2 \psi(t,\vec{x}) + V(t,\vec{x})\psi(t,\vec{x}) \qquad \vec{x} = (x,y,z) \; \nabla^2 = \frac{\delta^2}{\delta x^2} + \frac{\delta^2}{\delta y^2} + \frac{\delta^2}{\delta z^2}
\end{equation*}
Per avere interpretazione probabilistica voglio che a ogni tempo valga 
\begin{equation*}
    \int d^3x \, {|\psi(t,\vec{x})|}^2 = 1
\end{equation*}
dP è la probabilità infinitesima di trovare particella in un punto $\vec{x}$ nel volume di $d^3 xs$

\chapter{Conservazione locale probabilità ed equazione di continuità}

\'E vero sempre che vale la sequente relazione?
\begin{equation*}
    \int d^3x \, {|\psi(t,\vec{x})|}^2 = 1
\end{equation*}
Considero l'equazione di Schrodinger 
\begin{equation*}
    i\hslash \frac{\delta}{\delta t} \psi = -\frac{\hslash^2}{2m}\nabla^2 \psi + V\psi
\end{equation*}
e ne faccio il complesso coniugato
\begin{equation*}
    -i\hslash \frac{\delta}{\delta t} \psi^* = -\frac{\hslash^2}{2m}\nabla^2 \psi^* + V\psi^*
\end{equation*}
Prendo l'equazione 
\begin{equation*}
    \frac{\delta}{\delta t}{|\psi|}^2 = \frac{\delta}{\delta t} \psi^*\psi = \frac{\delta \psi^*}{\delta t}\psi + \psi^* \frac{\delta \psi}{\delta t}
\end{equation*}
sostituendo con ciò che abbiamo ottenuto dalle due equazioni precedenti si ottiene 
\begin{equation*}
    \frac{\delta}{\delta t}{|\psi|}^2 = \frac{1}{i\hslash} \left[-\frac{\hslash^2}{2m} \psi^* \nabla^2 \psi + \psi^*V\psi +\frac{\hslash^2}{2m} \psi\nabla^2\psi^* - \psi V\psi^* \right] 
\end{equation*}
che in definitiva si ottiene 
\begin{equation}
    \label{eq:2}
    \frac{\delta}{\delta t}{|\psi|}^2 = -\frac{\hslash}{2im} (\psi^*\nabla^2\psi - \psi\nabla^2\psi^*)
\end{equation}
Ora posso definire $\vec{J}$, densità di corrente di probabilità
\begin{equation*}
    \vec{J} = \frac{\hslash}{2im} (\psi^* \vec{\nabla} \psi - \psi \vec{\nabla}\psi^*)
\end{equation*}
calcolo la divergenza di $\vec{J}$
\begin{equation*}
    \mbox{div}\vec{J} = \frac{\hslash}{2im} (\psi^*\nabla^2\psi - \psi\nabla^2\psi^*)
\end{equation*}
definendo la densità di probabilità $\rho$
\begin{equation*}
    \rho = {|\psi(t,\vec{x})|}^2
\end{equation*}
l'\autoref{eq:2} diventa
\begin{equation}
    \label{eq:3}
    \frac{\delta \rho}{\delta t} + \mbox{div}\vec{J} = 0
\end{equation}
Commenti:
\begin{enumerate}
    \item $\vec{J}$ è reale: $\vec{J}^* = \vec{J}$
    \item A meno di interpretazione l' \autoref{eq:3} è uguale all'equazione di continuità della carica elettrica: conservazione della carica locale e non globale
\end{enumerate}
Prendo un volume $\Omega$ arbitrario ($\delta \Omega$ è la superficie del volume)
\begin{equation*}
    \frac{\delta}{\delta t } \int_\Omega d^3x \, \rho(t,\vec{x}) = - \int_\Omega d^3x \, \mbox{div}\vec{J}(t,\vec{x})
\end{equation*}
usando il teorema di Gauss
\begin{equation*}
    \frac{\delta}{\delta t} \int_\Omega d^3x \, \rho = \int_{\delta\Omega} \vec{J} \cdot d\vec{s} 
\end{equation*}
se $\Omega$ è tutto lo spazio 
\begin{equation*}
    \int_\Omega d^3x \, {|\psi|}^2 = \mbox{finito}
\end{equation*}
affinchè $\psi$ possa essere integrata deve "morire" all'infinito $\Rightarrow$ non c'è $\vec{J}$
\begin{equation*}
    \frac{\delta}{\delta t} \int d^3x \, {|\psi(t,\vec{x})|}^2 = 0
\end{equation*}
quindi
\begin{equation*}
    \int_{-\infty}^\infty d^3x \, {|\psi(t,\vec{x})|}^2 = \mbox{cost} = 1 \qquad \forall \, t
\end{equation*}

\chapter{Considerazione sugli operatori: parte I}

\chapter{Potenziali indipendenti dal tempo e stati stazionari}

\begin{equation*}
    V(t,\vec{x}) \rightarrow V(\vec{x})
\end{equation*}
Potenziale che non dipende dal tempo non implica che $\psi(t,x)$ non dipenda dal tempo. Cerco soluzioni in una famiglia a variabili separabili
\begin{equation*}
    \psi(t,\vec{x}) = \phi(\vec{x})T(t)
\end{equation*}
\begin{equation*}
    i\hslash \phi(\vec{x}) \frac{\delta T}{\delta t} = -\frac{\hslash^2}{2m} T(t) \nabla^2\phi(\vec{x}) + V(\vec{x})T(t)\phi(\vec{x})
\end{equation*}
Divido per $\phi(\vec{x})T(t)$
\begin{equation*}
    i\hslash \frac{1}{T} \frac{dT}{dt} = \frac{1}{\phi} \left[-\frac{\hslash^2}{2m}\nabla^2\phi + V\phi \right] = E(cost)
\end{equation*}
\begin{equation*}
    i\hslash \frac{dT}{T} = E dt \rightarrow \mbox{integro } \int_{T_i}^{T_f} \; \int_{t_i}^{t_f}
\end{equation*}
\begin{equation*}
    T = c e^{-\frac{iEt}{\hslash}} \qquad T(0) = c
\end{equation*}
E interpretabile come energia totale del sistema che è costante, a patto che il potenziale non dipenda dal tempo. Esistono stati particolari, che chiamiamo stati stazionari, che dipendono dal tempo con un puro fattore di fase
\begin{equation*}
    \psi(t,\vec{x}) = \phi(\vec{x}) e^{-\frac{iEt}{\hslash}}
\end{equation*}
allora $|\psi| = |\phi|$: non dipende dal tempo 

\subsection*{Equazione di Schrodinger per gli stati stazionari}

\begin{equation}
    -\frac{\hslash^2}{2m} \nabla^2\phi(\vec{x}) + V(\vec{x})\phi(\vec{x}) = E\phi(\vec{x}) \qquad E = \frac{\hat{p}^2}{2m} + V(\vec{x})
\end{equation}
Operatore hamiltoniano
\begin{equation*}
    \hat{H} = \frac{\hat{p}^2}{2m} + V(\hat{x}) \underbrace{=}_{\mbox{rappr. coord}} -\frac{\hslash^2}{2m}\nabla^2 + V(\vec{x})
\end{equation*}
Allora l'equazione di Schrodinger per gli stati stazionari diventa
\begin{equation*}
    \hat{H}\phi = E\phi 
\end{equation*}

\chapter{Proprietà soluzioni in 1 dimensione}

\begin{equation*}
    \hat{H}\phi = E\phi \iff -\frac{\hslash^2}{2m}\frac{d^2\phi}{dx^2} + V(x)\phi(x) = E \phi(x) \qquad \phi(x) \in L^2(\R)
\end{equation*}
con $\hat{H}$ operatore autoaggiunto o hermitiano
\begin{equation*}
    (g,f) = \int_{-\infty}^\infty dx \, g^*f
\end{equation*}
\begin{enumerate}
    \item Spettro discreto
    \begin{itemize}
        \item $E_n$ sono reali
        \item Autofunzioni corrispondenti ad autovalori di $E_n$ diversi sono ortogonali
        \begin{equation*}
            \phi_1,\phi_2 \; E_1 \neq E_2 \Rightarrow \int dx \, \phi_1\phi_2^* = 0
        \end{equation*}
        \item $\hat{H}\phi = E\phi$ ha una sola soluzione per ogni valore di E: gli stati dello spettro discreto non sono degeneri
        \item $\phi_n$ dello spettro discreto sono reali
        \item $\phi_n$ corrispondente ad autovalore (n+1)-esimo di energia si azzera per n punti finiti di x
        \item Se V(x) è una funzione pari (V(x)=V(-x)) allora gli stati stazionari $\phi_n(x)$ sono o pari o dispari   
    \end{itemize}
    \item Spettro continuo: le autofunzioni $\not\in L^2(\R)$, ma possono essere utilizzate per costruire soluzioni fisiche 
    \begin{itemize}
        \item Data $\phi_E \rightarrow \hat{H}\phi_E = E\phi_E \Rightarrow \hat{H}$ è limitato
        \item $\int_{E-\Delta E}^{E+ \Delta E} E c(E)\phi_E(x) \; \in L^2(\R) \Rightarrow$ c(E) hanno spettro finito 
    \end{itemize}
\end{enumerate}

\chapter{Considerazioni sugli operatori: parte II}

Considero i due operatori $\hat{x}$ e $\hat{p}$ e una cenerica osservabile classica $A(\vec{x},\vec{p})$. A livello quantistico abbiamo $A(\hat{x},\hat{p})$, ma abbiamo un problema: se a livello classico non c'è preferenza su ordine, in meccanica quantistica cosa succede se A contiene px e non xp?
\begin{equation*}
    \frac{xp + px}{2} \mapsto \frac{\hat{x}\hat{p} + \hat{p}\hat{x}}{2}
\end{equation*}
N.B.
\begin{enumerate}
    \item A noi interessano operatori Hermitiani: $\hat{x}\hat{p}$ non è hermitiano
    \begin{equation*}
        {(\hat{x}\hat{p})}^+ = \p^+\x^+ = \p\x \neq \x\p 
    \end{equation*}
    ma 
    \begin{equation*}
        {\left(\frac{\hat{x}\hat{p} + \hat{p}\hat{x}}{2}\right)}^+ = \frac{\hat{p}^+\hat{x}^+ + \hat{x}^+\hat{p}^+}{2} = \frac{\p\x + \x\p}{2} = \frac{\x\p + \p\x}{2}
    \end{equation*}
    \item $\hat{H} = \frac{\p^2}{2m} + V(\x)$
\end{enumerate}

\chapter{I postulati della meccanica quantistica}
Ipotesi meccanica quantistica: ogni sistema descritto da funzione d'onda $\psi(t,\vec{x})$. \newline
Voglio prendere misura di un'osservabile A; considero tutti i sistemi descritti dalla stessa funzione d'onda $\psi(\vec{x})$ \newline
Cosa succede se misuro A? Generalmente succede che non avrò lo stesso risultato, INDIPENDENTEMENTE dagli errori di misura. Misuro $A_1,A_2,A_3,\dots,A_i$, che possono essere sia valori discreti (per semplicità il caso che analizziamo), sia continui \newline
Mi aspetto che 
\begin{equation*}
    <\hat{A}> = \int dx \, \psi^*(x)\hat{A}(\hat{x},\hat{p})\psi(x)
\end{equation*}
corrisponda al valore classico, quindi
\begin{equation*}
    \sum_i^N \frac{a_i}{N} \sim <\hat{A}>
\end{equation*}
Per lo scarto quadratico medio posso scrivere l'equivalente di quello che otteniamo classicamente
\begin{equation*}
    {(\Delta \hat{A})}^2 = {(\hat{A}-<\hat{A}>)}^2  
\end{equation*}
\begin{equation}
    \label{eq:1}
    SQM = \int dx \, \psi^* (\hat{A}-<\hat{A}>)(\hat{A}-<\hat{A}>)\psi
\end{equation}
Esistono casi in meccanica quantistica in cui, data $\psi$, ottengo sempre lo stesso risultato di misura? \newline
Se esiste $\psi$ che mi fa ottenere sempre lo stesso risultato, allora $SQM = 0$; ciò significa che esiste uno stato particolare $\bar{\psi}$ che rende nullo l'integrale \ref{eq:1}
\begin{equation*}
    (\hat{A}-<\hat{A}>)\bar{\psi}
\end{equation*}
Se esiste una $\bar{\psi}$ per cui la misura è sempre la stessa allora deve valere la seguente relazione
\begin{equation*}
    \label{eq:autovalori}
    \hat{A}\psi_a = a\psi_a \qquad \text{EQUAZIONE AGLI AUTOVALORI}
\end{equation*}
NON abbiamo detto che \{a\} siano gli unici valori delle misure \newline
Adesso prendo uno stato generico $\psi$ e faccio una certa misura di un'osservabile; pochissimo tempo dopo faccio di nuovo stessa misura della stessa grandezza: se non lascio evolvere il sistema, mi aspetto di ottenere lo stesso risultato, perchè non ho perturbato il sistema $\rightarrow$ riproducibilità in piccolo \newline
Questa è una ragionevole assunzione, non la posso dimostrare \newline
Se faccio una misura e ottengo il valore $a_3$, e subito dopo ottengo lo stesso valore, allora lo stato subito dopo la prima misura si deve essere trasformato nello stato $\psi_3$: solo su questo stato ho la certezza di avere di nuovo $a_3$. Allora la prima misura deve aver trasformato pesantemente il sistema $\rightarrow$ lo proietta su uno degli autostati $(\psi \mapsto \psi_a)$ \newline
SE \'E COS\'I ALLORA GLI UNICI RISULTATI POSSIBILI DI UNA MISURA SONO GLI AUTOVALORI DELL'OSSERVABILE MISURATA \newline
Commenti:
\begin{enumerate}
    \item Se $\psi$ generica fosse autofunzione di A ($\psi_{a_j}$): allora prima misura mi darebbe sempre $a_j$
    \item Se $\psi$ non è autofunzione di A: allora $\psi$ deve precipitare su $\psi_{a_j}$ perchè seconda misura deve essere $a_j$
    \item Le osservabili sono rappresentate da operatori i cui autovalori sono tutti e soli i valori di una misura
    \item All'atto della misura funzione precipita su autovettore corrispondente ad autovalore misurato
\end{enumerate}
In maniera imprevedibile precipitiamo su autostato. A priori non possiamo sapere su quale autostato precipito. Possiamo solo e soltanto ottenere i valori $a_j$ \newline
Possiamo prevedere con quale probabilità arriviamo su certo autostato? \newline
Suppongo di fare esperimento su diversi sistemi con valore $\psi$, per farmi idea statistica di ciò che ottengo
\begin{itemize}
    \item se $\psi$ coincide con $\psi_{a_3}$ allora $Prob(a_3)=1$
    \item se $\psi \simeq \psi_{a_3}$ allora $Prob(a_3)\sim 1$
\end{itemize}
\begin{enumerate}
    \item Probabilità devono essere proporzionali a coefficienti di sviluppo in serie della $\psi$ su $\psi_{a_i}$
    \item Deve valere principio di sovrapposizione delle ampiezze
    \begin{equation*}
        \psi = \sum_{i=1}^N c_i\psi_i
    \end{equation*}
\end{enumerate}
Per avere interpretazione probabilistica 
\begin{equation*}
    \int dx \, \psi^*\psi = 1
\end{equation*}
Se gli autovettori fossero ortogonali:
\begin{equation*}
    \int dx \, \psi^*\psi = \sum_i \sum_j \int dx \, c_i^* \psi_{a_i}^*c_j\psi_{a_j} = \sum_i c_i^*c_j \delta_{a_ia_j}= \sum {|c_i|}^2 = 1
\end{equation*}
allora posso interpretare ${|c_i|}^2$ come probabilità di ottenere $a_i$. Se ortogonali allora buona congettura che ${|c_i|}^2$ è una probabilità \newline
Data $\psi = \sum c_i \psi_{a_i}$
\begin{equation*}
    \int dx \, \psi_{a_j}^*\psi = \int dx \, \psi_{a_j}^*\psi_{a_i} = c_j
\end{equation*}
Conseguenze formali: \newline
Se "$a_j$" sono tutti e soli i valori di misura allora devono essere reali $\rightarrow$ autovettori ortogonali. Esiste classe di operatori che ha sempre autovalori reali $\rightarrow$ spazio di Hilbert con operatori hermitiani o autoaggiunti

\section*{I postulati}
\begin{enumerate}
    \item Qualunque stato quantistico è descritto vettore di stato (ket) $|\psi>$ che vive in uno spazio di Hilbert
    \item Ad ogni quantità misurabile A è associato operatore $\hat{A}$ che agisce su spazio di Hilbert; $\hat{A}$ è hermitiano ed è chiamato osservabile
    \item I soli possibili risultati di una misura di una quantità sono gli autovalori associati ad $\hat{A}$ 
    \item Supponendo di avere uno spettro discreto e non degenere, allora quando osservo A su stato normalizzato $|\psi>$ 
    \begin{equation*}
        P(a_n) = {|<\psi_{a_n}|\psi>|}^2
    \end{equation*}
    dove $\psi_{a_n}$ è autovettore normalizzato associato ad autovalore $a_n$ dell'operatore $\hat{A}$ 
    \item Supponendo di avere uno spettro discreto e non degenere, se faccio misura di A descritto da stato $|\psi>$ e ottengo $a_n$ allora la misura successiva si trova su $|\psi_{a_n}>$ corrispondente ad $a_n$
    \item L'evoluzione temporale $|\psi(t)>$ obbedisce all'equazione di Schrodinger
    \begin{equation*}
        i\hslash \frac{\delta}{\delta t}|\psi(t)> = \hat{H}(t)|\psi(t)>
    \end{equation*}
    $\hat{H}(t)$ osservabile associata all'energia totale del sistema $\rightarrow$ hamiltoniana
\end{enumerate}

\chapter{Richiami spazi di Hilbert}

Uno spazio di Hilbert è uno spazio vettoriale, normato (norma segue da definizione prodotto scalare) e completo. Perchè usiamo uno spazio di Hilbert?
\begin{enumerate}
    \item Sovrapposizione di funzioni d'onda voglio che siano funzioni d'onda: spazio lineare
    \begin{equation*}
        \psi_1,\psi_2 \in H \Rightarrow \lambda_1\psi_1 + \lambda_2\psi_2 \in H \qquad \lambda_1,\lambda_2 \in \C 
    \end{equation*}
    \item Prodotto scalare è un numero complesso
    \begin{equation*}
        (\phi,\psi) = <\phi | \psi> = \int d^3x \, \phi^*(\vec{x})\psi(\vec{x})
    \end{equation*}
    \item $L^2(\R)$ funzioni a quadrato sommabile
    \begin{equation*}
        \int_{-\infty}^\infty d^3x \, {|\psi(\vec{x})|}^2 < +\infty
    \end{equation*}
\end{enumerate}

\subsection*{Proprietà prodotto scalare}

\begin{itemize}
    \item $\phi | \psi > = {<\psi | \phi >}^*$
    \item \'E lineare nel secondo argomento $\Rightarrow <\phi | \lambda_1\psi_1 + \lambda_2\psi_2> = \lambda_1 <\phi | \psi_1 > + \lambda_2 < \phi | \psi_2 >$
    \item \'E antilineare nel secondo argomento $\Rightarrow <\lambda_1 \phi_1 + \lambda_2 \phi_2 | \psi> = \lambda_1^*<\phi_1 | \psi> + \lambda_2^* <\phi_2 |\psi> $
\end{itemize}

\subsection*{Norma}
\'E un numero reale $\ge 0$ 
\begin{equation*}
    <\psi | \psi > = \int d^3x \, {|\psi(\vec{x})|}^2 = {||\psi||}^2 
\end{equation*}
$<\psi | \psi > = 0 \iff \psi(\vec{x})=0$

\subsection*{Basi ortonormate}
\begin{enumerate}
    \item $|n> $ con n = 1,2,\dots tale che $<n|m> = \delta_{nm}$
    \item Trasformate di Fourier, polinomi di Hermite $\Rightarrow$ basi improprie (continue)
\end{enumerate}

\subsection*{Spazio completo}

\begin{equation*}
    |\psi> = \sum_{n=1}^\infty \psi_n |n> \qquad \psi_n = <n | \psi>
\end{equation*}
\begin{equation*}
    |\phi > = \sum_{n=1}^\infty \phi_n |n>
\end{equation*}
Allora 
\begin{equation*}
    <\psi| \phi> = \sum_{n,m} \psi_n^* \phi_n <n|m> = \sum_n \psi^*_n \phi_n
\end{equation*}

\subsection*{Notazione di Dirac}
Serve qualcosa di più astratto per descrivere lo stato della particella; ad esempio $\psi(\vec{x}), \tilde{\phi(\vec{p})}$ portano stesse informazioni, solo che a volte le coordinate vengono indicate con le x, altre volte con le p $\Rightarrow |\psi>$ \newline
Indico lo stato fisico di un sistema con un KET $|\psi>$; allo spazio dei ket associo un BRA $<\phi|$ spazio duale: spazio dei funzionali lineare nello spazio di Hilbert, dato $<\phi|$ prendo un elemento dello spazio H e ci associo un numero complesso. \newline
Se ho ket 
\begin{equation*}
    |\lambda_1\psi_1 + \lambda_2\psi_2> = \lambda_1|\psi_1> + \lambda_2|\psi_2>
\end{equation*}
il suo duale vale 
\begin{equation*}
    <\lambda_1\psi_1 + \lambda_2\psi_2| = \lambda_1^*<\psi_1| + \lambda_2^*<\psi_2|
\end{equation*} 
Quindi lo spazio dei ket è lineare, mentre lo spazio dei bra è antilineare
\begin{equation*}
    <\psi|\phi> = \sum_n \psi_n^* \phi_n = (\psi_1^*\psi_2^*, \dots, \psi_n^*)
    \left(\begin{array}[]{c}
        \phi_1 \\
        \phi_2 \\
        \vdots \\
        \phi_n
    \end{array}\right)
\end{equation*}
Possiamo rappresentare $|\phi>$ con i suoi coefficienti su una data base e li organizziamo in un vettore colonna, mentre rappresentiamo $<\psi|$ in un vettore riga. Per passare da $|\psi>$ a $<\psi|$ devo fare il trasposto e il complesso coniugato

\chapter{Osservabili compatibili e incompatibili}

\subsection*{Notazione}

Dati due operatori $\hat{A}$ e $\hat{B}$ che commutano, il ket viene identificato dali $a_i,b_i$
\begin{align*}
    \hat{A}|a_ib_j> = a_i |a_ib_j> \\
    \hat{B}|a_ib_j> = b_j |a_ib_j>
\end{align*}
I $b'_j$ trovati dalla diagonalizzazione di $B_3$ sono diversi tra loro? Se alcuni autovalori sono uguali, allora conoscere il valore di $a_i$ e $b_j$ non è sufficiente a conoscere l'autovettore (ci sono k autovalori legati a $b_j$). Si cerca allora il terzo operatore $\hat{C}$ che commuta simultaneamente con $\hat{A}$ e $\hat{B}$
\begin{equation*}
    [\hat{A},\hat{B}] = [\hat{B}, \hat{C}] = [\hat{A}, \hat{C}] = 0
\end{equation*}
Etichettiamo allora l'autovettore con $|a_ib_jc_k>$ affinchè l'autostato sia univocamente definito. Una volta trovati tutti gli operatori necessari per fare ciò, si dice che si che di ha RISOLTO LA DEGENERAZIONE. Tutti gli operatori usati sono contemporaneamente diagonali sulla base usata. \newline
Il compito fondamentale della meccanica quantistica è quindi di ricercare il set completo di osservabili che commutano per poter porre un label a ogni autostato in modo che esso sia univocamente determinato $\rightarrow$ complete set of commuting onservables. Il set si dice completo se aggiungendo un'altra osservabile che commuta con tutte quelle già esistenti questa osservabile è funzione di quelle già presenti nel set. \newline
Per dare una descrizione completa di un sistema, ovvero a dare la massima informazione possibile, dobbia determinare il numero massimo di osservabili indipendenti e che commutano fra di loro. Questa è detta OSSERVAZIONE MASSIMA: dato uno stato $|a_ib_jc_kd_l \dots>$ misurare $\hat{A},\hat{B},\hat{C},\hat{D}$ darà con certezza i risultati $a_i,b_j,c_k,d_l$. Per fare la ricerca del set si osserva l'hamiltoniana e le simmetria del problema

\chapter{Evoluzione temporale del valor medio di un'osservabile}

Consideriamo un operatore $\hat{A}(\hat{x},\hat{p},t)$ e il generico stato $|\psi(t)>$. Il valore medio sarà
\begin{equation*}
    <\psi(t)|\hat{A}|\psi(t)> = \int d^3x \, \psi^*(x,t)\hat{A}\left(\hat{x},-i\hslash \frac{\delta}{\delta x}, t\right)\psi(x,t)
\end{equation*}
e sarà un numero t-dipendent. La dipendenza dal tempo veniva sia dall'operatore che dalla funzione d'onda. Per notazione scriviamo il valore medio come
\begin{equation*}
    <\hat{A}>(t)= <\psi(t)|\hat{A}|\psi(t)>
\end{equation*}
allora la derivata sarà
\begin{equation*}
    \frac{d}{dt} <\hat{A}>(t) = \left(\frac{d}{dt}<\psi(t)|\right) \hat{A}|\psi(t)> + <\psi(t)|\frac{\delta}{\delta t}\hat{A}|\psi(t)> + <\psi(t)|\hat{A}|\frac{d}{dt}\psi(t)>
\end{equation*}
L'ultimo pezzo è noto in quanto deve soddisfare che 
\begin{equation*}
    i\hslash \frac{d}{dx}|\psi> = \hat{H}|\psi>
\end{equation*}
da essa si ricava che 
\begin{equation*}
    -i\hslash \frac{d}{dt}<\psi| <\psi|\hat{H}
\end{equation*}
\begin{equation*}
    \frac{d}{dt} <\hat{A}>(t) = \frac{1}{i\hslash}<\psi(t)| [\hat{A}(t)\hat{H(t)}-\hat{H}(t)\hat{A}(t)]|\psi(t)> + <\psi(t)|\frac{\delta}{\delta t}\hat{A}| \psi(t)>
\end{equation*}
Ottenendo
\begin{equation}
    \label{*}
    \frac{d}{dt} <\hat{A}>(t) = \frac{1}{i\hslash} \underbrace{<[\hat{A},\hat{H}]>}_{\mbox{valor medio del commutatore}} + \underbrace{<\frac{\delta A}{\delta t}>}_{\mbox{valor medio della derivata}}
\end{equation}

\chapter{Limite classico e teorema di Ehrenfest}

Consideriamo una particella senza spin in un potenziale stazionario $V(\vec{x})$. L'hamiltoniana sarà
\begin{equation*}
    \hat{H} = \frac{\hat{p}^2}{2m} + V(\vec{x})
\end{equation*}
Applichiamo \eqref{*} a $\hat{x},\hat{p}$
\begin{equation*}
    \frac{d}{dt}<\hat{x}> = \frac{1}{i\hslash}<[\hat{x},\hat{H}]> = \frac{1}{i\hslash} \left[\hat{x}, \frac{\hat{p}^2}{2m}\right] = \frac{1}{i\hslash 2m}<\hat{p}[\hat{x},\hat{p}]+[\hat{x},\hat{p}]\hat{p}> = \frac{1}{m} <\hat{p}> 
\end{equation*}
\begin{equation*}
    \frac{d}{dt}<\hat{p}> = \frac{1}{i\hslash}<[\hat{p},\hat{H}]> = \frac{1}{i\hslash} <[\hat{p},V(\vec{x})]> = - <\vec{\nabla}V(\vec{\hat{x}})>
\end{equation*}
Teorema di Ehrenfest
\begin{align*}
    \frac{d}{dt}<\hat{x}> = \frac{1}{m} <\hat{p}> \\
    \frac{d}{dt}<\hat{p}> = - <\vec{\nabla}V(\vec{\hat{x}})>
\end{align*}
Possiamo oa combinare le equazioni
\begin{equation*}
    m \frac{d^2}{dt^2}<\hat{x}> = - <\vec{\nabla}V(\vec{\hat{x}})>
\end{equation*}
Nel limite classico


\end{document}