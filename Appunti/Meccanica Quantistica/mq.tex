\documentclass[a4paper,11pt]{report}

\usepackage[T1]{fontenc}
\usepackage[utf8]{inputenc}
\usepackage[italian]{babel}

\usepackage{mathdots}
\usepackage{mathtools}
\usepackage{graphicx}
\usepackage{amsfonts}
\usepackage{amsthm}
\usepackage{amsmath}
\usepackage{amssymb}
\usepackage{fancyhdr}
\usepackage{float}
\usepackage{geometry}
\geometry{a4paper, top=2.5cm, bottom=2cm, left=2cm, right=2cm}
\usepackage{hyperref}
\hypersetup{
	colorlinks=true,
	linkcolor=black,
	filecolor=blue,
	citecolor = black,      
	urlcolor=cyan,
}

\swapnumbers
\theoremstyle{remark}
\newtheorem*{oss}{Oss}
\newtheorem*{nb}{N.B}
\newtheorem*{coro}{Corollario}
\theoremstyle{definition}
\newtheorem*{teo}{Teorema}
\newtheorem*{Def}{Def}

\newcommand{\C}{\mathbb{C}}
\newcommand{\R}{\mathbb{R}}

\DeclarePairedDelimiter{\abs}{\lvert}{\rvert}

\begin{document}
	\date{}
	\author{Marco Militello}
	\title{Meccanica Quantistica}
	\maketitle
	\tableofcontents
	\newpage
	
\chapter{La crisi della fisica classica}

Concetti incompatibili con la fisica classica, che non si riescono a spiegare. Per fisica classica si intende
\begin{enumerate}
    \item meccanica newtoniana
    \item termodinamica fisica statistica classica
    \item elettromagnetismo e le leggi di Maxwell
    \item relatività ristretta
\end{enumerate}

\section{Corpo nero}

Universalità: stesso spettro di emissione. Basi termodinamiche $\rightarrow$ Kirchoff. \newline
1900 Planck: giunge a risultato giusto, ma partendo da principi sbagliati
\begin{equation*}
    g(\nu, T) = \frac{8\pi h}{c^3} \frac{\nu^3}{e^{\frac{h\nu}{K_B T}}-1}
\end{equation*}
Ottiene questa relazione grazie a fit dei dati sperimentali; aggiusta i paramentri: si ottiene per la prima volta $h$, la costante di Planck
\begin{equation}
    h = 6.6 \times 10^{-34} J \cdot s
\end{equation}
ha la stessa dimesione di un momento angolare

\section{Effetto fotoelettrico}

1905 Einstein \newline
Proposta di Einstein che spiega l'effetto: scambio di energia come multiplo di $h\nu \rightarrow$ QUANTI DI RADIAZIONE. \newline
Nel grafico dell'energia massima in funzione della frequenza, Millikan misurò la pendenza della retta che risultò essere $\frac{h}{e}$
Si continuava però a pensare che l'energia si propagasse nel continuo, mentre durante le interazioni ci fossere scambi a pacchetti. Questa idea viene abbandonata con l'effetto Compton

\section{Effetto Compton}
Raggi-x su un metallo; $E_m >> E_{legame}$: in questo modo gli elettroni sono visti come liberi.
\begin{equation*}
    E^2 - p^2c^2 = m^2c^4
\end{equation*}
se v=c allora m=0. Quindi
\begin{equation*}
    p^2c^2 = E^2 = {(h\nu)}^2 \Rightarrow p=\frac{h\nu}{c}
\end{equation*}

\subsection*{Esperimento}
Tratto fotone come particella con 
$\begin{cases}
    E = h\nu \\
    p = \frac{h\nu}{c}
\end{cases}$
\begin{equation*}
    \lambda' - \lambda = \frac{h}{m_e c}(1 - \sin(\theta))
\end{equation*}
onda elettromagnetica non completamente descritta dalle leggi di Maxwell \newline
$\frac{h}{m_e c} \simeq 2.4 \times 10^{-12} m$: lunghezza d'onda di Compton per l'elettrone $\Rightarrow$ ha la dimensione di una lunghezza. \newline
1926 Lewis: assegna nome ai fotoni

\section{Spettri atomici}
Spettro solare ha delle righe nere: radiazione viene assorbita dallo strato esterno del sole e poi viene diffusa, quindi ci arriva meno intensa $\Rightarrow$ righe nere. $He$ scoperto grazie allo spettro solare. \newline
1897: scoperta elettrone \newline
Modelli atomici $\rightarrow$ Rutherford: però non riesce a spiegare spettri atomici \newline
Emissione e assorbimento non sono un continuo \newline
Bohr: Energia quantizzata
\begin{equation*}
    \nu_{mn} = (E_m - E_n)\frac{1}{h}
\end{equation*}
ma non dà alcun tipo di spiegazione
\begin{equation*}
    m_e v r = n \hslash \qquad \text{ momento angolare è multiplo intero di } \hslash = \frac{h}{2\pi}
\end{equation*}

\chapter{Interferenza e diffrazione onde elettromagnetiche}
Luce descrivibile con fenomeni ondulatori. \newline
Esperimento doppia fenditura: metto uno schermo a una distanza $L >> \lambda$ dalle fenditure. In ogni punto ho che $ I \neq I_1 + I_2$, ma ho che 
\begin{equation*}
    I = {|\vec{E}_1 + \vec{E}_2|}^2 \neq {|\vec{E}_1|}^2 + {|\vec{E}_2|}^2
\end{equation*}
La differenza di fase vale $d\sin(\theta)$; se la differenza di fase vale $n\frac{\lambda}{2}$ allora ho interferenza distruttiva. Il primo punto di buio si ha per 
\begin{equation*}
    \theta = \frac{n}{2}\frac{\lambda}{d} \text{ con } n=1
\end{equation*}
Se prendo una lampadina gialla (580 nm) da 100 W, allora vengono emessi $\sim 10^{20}$ fotoni al secondo. Cosa succede se abbasso intensità fino ad avere emissione di un fotone al secondo? L'esperimento della doppia fenditura porta allo stesso risultato precedente; però se ripeto di nuovo l'esperimento la posizione di ogni singolo fotone è diverso, anche se la figura che si viene a creare porta allo stesso risultato. Allora devo introdurre una trattazione probabilistica: posso solo dare distribuzione probabilità \newline
Se chiudo una delle due fenditure non osservo alcuna figura di interferenza: il fotone "sente" la presenza di entrambe le fenditure. Ma con chi interferisce il fotone? Da che fenditura passa? \newline
Esperimenti per capire da che fenditura sia passato il fotone distruggono la figura di interferenza $\Rightarrow$ misura microscopica disturba il fenomeno. \newline
Risultati esperimento:
\begin{enumerate}
    \item Fotone colpisce schermo in un punto ben preciso $\Rightarrow$ deposita tutta la sua energia $h\nu$
    \item Con pochi eventi lo schermo sembra riempirsi in maniera casuale; non sappiamo esattemente dove andrà un fotone, ma possiamo dire dove si sono addensati maggiormente $\Rightarrow$ PROBABILIT\'A
    \item Se chiudo una fenditura sparisce figura di interferenza: fotone con $\lambda << d$ "sente" le 2 fenditure; ogni tentativo di capire da quale fenditura passi il fotone distrugge la figura di interferenza $\rightarrow$ disturbo con la misura. La misura su un sistema microscopico lo può disturbare in maniera significativa.
\end{enumerate}
In fisica classica, date le condizioni inziali posso completamente determinare il moto di una particella; invece non posso determinare moto di un fotone: fotone ha una certa probabilità di colpire lo schermo proporzionale all'intensità I(x) [probabilità che fotone finisca in un punto preciso dello schermo]ù

\subsection*{Dualismo onda-particella}
In alcuni esperimenti è più facile interpretare come particella altre volte come onda; la vera natura è l'elettrodinamica quantistica
\begin{enumerate}
    \item La radiazione elettromagnetica si comporta come un flusso di particelle
    \item Previsioni sul comportamento sono solo probabilistiche
    \item In un certo punto $\vec{r}$ dello schermo al tempo t $\rightarrow$ fotone è portato da campo elettrico $\vec{E}(\vec{r}, t)$ che è soluzione delle equazioni di Maxwell; campo elettrico va interpretato come un'ampiezza di probabilità di trovare un fotone in un istante t in un punto $\vec{r}$
    \begin{equation*}
        {|E(\vec{r},t)|}^2 \rightarrow \text{ densità di probabilità}
    \end{equation*}
    \item Le equazioni di MAxwell sono lineari in $\vec{E} \Rightarrow$ vale principio di sovrapposizione: se $E_1, E_2$ sono soluzioni delle equazioni di Maxwell allora anche $\vec{E} = \lambda_1\vec{E}_1 + \lambda_2 \vec{E}_2$ con $\lambda_1,\lambda_2 \in \R$ è soluzione delle equazioni di Maxwell. Principio di sovrapposizione è anche quello che ci permette di spiegare interferenza
\end{enumerate}
"Ogni fotone interfesce solo con se stesso"

\chapter{La polarizzazione della luce}
La luce ha una direzione privilegiata. Es.: laser, smartphone \newline

\section*{Esperimento}
Mettere immagine \newline
Interpretazione classica: c'è onda che si propaga lungo z
\begin{equation*}
    \vec{E}(\vec{r},t) = E_0 \vec{\epsilon}_p e^{i(kz-wt)}+ c.c. \qquad k = \frac{2\pi}{\lambda} \; w = 2\pi\nu \; \lambda\nu = c
\end{equation*} 
Il campo elettrico è libero, non ci sono cariche $\Rightarrow$ $\div\vec{E}=0 \rightarrow \vec{k}\cdot \vec{\epsilon}_p =0$. Quindi trasverso rispetto alla direzione di propagazione \newline
$\vec{k} // \vec{z} \Rightarrow$ il vettore di polarizzazione appartiene al pino perpendicolare alla direzione di propagazione
\begin{equation*}
    I_0 = {|E_0|}^2
\end{equation*} 
Dopo polarizzazione: $\vec{E}'(\vec{r},t) = E_0'\vec{e}_x e^{i(kz-wt)}$
\begin{equation*}
    E_0' = E_0 cos\theta \Rightarrow I = I_0 \cos^2\theta
\end{equation*}
Dopo polarizzatore passa solo componente parallela $\rightarrow$ interpretazione che viene data. \newline
Adesso faccio esperimento in cui diminuisco intensità fascio fino a che emetta solo un fotone alla volta. Conseguenze:
\begin{enumerate}
    \item Rilevatore o vede il fotone o non lo vede
    \item Quando vede fotone lo vede tutto $\rightarrow $ tutta energia $h\nu$
    \item se $\begin{cases}
        \theta = 0 \Rightarrow \text{ rilevatore vede tutto fotone} \\
        \theta = \frac{\pi}{2} \Rightarrow \text{ rilevatore non lo vede mai}
    \end{cases}$ \newline
    La probabilità con cui vedo fotone è $\cos^2\theta \Rightarrow$ Numero fotoni arrivato $ = N_0\cos^2\theta$. Ogni singolo fotone può essere rappresentato come miscela di 2 stati
    \begin{itemize}
        \item uno stato con probabilità di passare 1: $\psi_x$
        \item uno stato con probabilità di passare 0: $\psi_y$
    \end{itemize}
    Questi valori sono mutualmente esclusivi
    \item Descrivo come somma di due stati $\psi_p = \psi_x \cos\theta + \psi_y \sin\theta \rightarrow$ inizio principio di decomposizione spettrale
    \begin{itemize}
        \item $\psi_x$ passa con probabilità ${|\cos\theta|}^2=\cos^2\theta$
        \item $\psi_y$ non passa con probabilità ${|\sin\theta|}^2=\sin^2\theta$
    \end{itemize}
    Interpretazione probabilistica è sensata perchè $\sin^2\theta + \cos^2\theta = 1$
    \item Se io ora metto un altro polarizzatore lungo x, tutti i fotoni passeranno perchè ora tutti i fotoni sono $\psi_x$: dopo polarizzatore lo stato fotone è diventato $\psi_x$. La misura ha fatto precipitare lo stato del sistema da $\psi_p$ a $\psi_x$
\end{enumerate}

\chapter{De Broglie}
Maxwell introduce andamento ondulatorio della radiazione elettromagnetico, mentre Einstein ipotizza andamento corpuscolare. De Broglie nel 1923 introduce ipotesi andamento ondulatorio elettroni. \newline
Onda $\rightarrow \exp(i \vec{k}\cdot \vec{x}- iwt)$; per essere invariante secondo Loretz $(\vec{k},w),(\vec{p},E)$ diventano tetravettori
\begin{equation*}
    |\vec{p}| = \frac{E}{c} = \frac{h\nu}{c} = \frac{h}{\lambda} = \hslash |\vec{k}| \Rightarrow \frac{h}{|\vec{p}|}
\end{equation*}
Ad ogni particella di massa è associato $\vec{k} = \frac{\vec{p}}{\hslash}$ e quindi un'onda $\lambda = \frac{h}{|\vec{p}|}$ \newline
Numericamente
\begin{enumerate}
    \item Elettroni
\end{enumerate} 

\end{document}