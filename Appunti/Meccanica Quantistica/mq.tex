\documentclass[a4paper,11pt]{report}

\usepackage[T1]{fontenc}
\usepackage[utf8]{inputenc}
\usepackage[italian]{babel}

\usepackage{mathdots}
\usepackage{mathtools}
\usepackage{graphicx}
\usepackage{amsfonts}
\usepackage{amsthm}
\usepackage{amsmath}
\usepackage{amssymb}
\usepackage{fancyhdr}
\usepackage{float}
\usepackage{geometry}
\geometry{a4paper, top=2.5cm, bottom=2cm, left=2cm, right=2cm}
\usepackage{hyperref}
\hypersetup{
	colorlinks=true,
	linkcolor=black,
	filecolor=blue,
	citecolor = black,      
	urlcolor=cyan,
}

\swapnumbers
\theoremstyle{remark}
\newtheorem*{oss}{Oss}
\newtheorem*{nb}{N.B}
\newtheorem*{coro}{Corollario}
\theoremstyle{definition}
\newtheorem*{teo}{Teorema}
\newtheorem*{Def}{Def}

\newcommand{\C}{\mathbb{C}}
\newcommand{\R}{\mathbb{R}}

\DeclarePairedDelimiter{\abs}{\lvert}{\rvert}

\begin{document}
	\date{}
	\author{Marco Militello}
	\title{Meccanica Quantistica}
	\maketitle
	\tableofcontents
	\newpage
	
\chapter{La crisi della fisica classica}

Concetti incompatibili con la fisica classica, che non si riescono a spiegare. Per fisica classica si intende
\begin{enumerate}
    \item meccanica newtoniana
    \item termodinamica fisica statistica classica
    \item elettromagnetismo e le leggi di Maxwell
    \item relatività ristretta
\end{enumerate}

\section{Corpo nero}

Universalità: stesso spettro di emissione. Basi termodinamiche $\rightarrow$ Kirchoff. \newline
1900 Planck: giunge a risultato giusto, ma partendo da principi sbagliati
\begin{equation*}
    g(\nu, T) = \frac{8\pi h}{c^3} \frac{\nu^3}{e^{\frac{h\nu}{K_B T}}-1}
\end{equation*}
Ottiene questa relazione grazie a fit dei dati sperimentali; aggiusta i paramentri: si ottiene per la prima volta $h$, la costante di Planck
\begin{equation}
    h = 6.6 \times 10^{-34} J \cdot s
\end{equation}
ha la stessa dimesione di un momento angolare

\section{Effetto fotoelettrico}

1905 Einstein \newline
Proposta di Einstein che spiega l'effetto: scambio di energia come multiplo di $h\nu \rightarrow$ QUANTI DI RADIAZIONE. \newline
Nel grafico dell'energia massima in funzione della frequenza, Millikan misurò la pendenza della retta che risultò essere $\frac{h}{e}$
Si continuava però a pensare che l'energia si propagasse nel continuo, mentre durante le interazioni ci fossere scambi a pacchetti. Questa idea viene abbandonata con l'effetto Compton

\section{Effetto Compton}
Raggi-x su un metallo; $E_m >> E_{legame}$: in questo modo gli elettroni sono visti come liberi.
\begin{equation*}
    E^2 - p^2c^2 = m^2c^4
\end{equation*}
se v=c allora m=0. Quindi
\begin{equation*}
    p^2c^2 = E^2 = {(h\nu)}^2 \Rightarrow p=\frac{h\nu}{c}
\end{equation*}

\subsection*{Esperimento}
Tratto fotone come particella con 
$\begin{cases}
    E = h\nu \\
    p = \frac{h\nu}{c}
\end{cases}$
\begin{equation*}
    \lambda' - \lambda = \frac{h}{m_e c}(1 - \sin(\theta))
\end{equation*}
onda elettromagnetica non completamente descritta dalle leggi di Maxwell \newline
$\frac{h}{m_e c} \simeq 2.4 \times 10^{-12} m$: lunghezza d'onda di Compton per l'elettrone $\Rightarrow$ ha la dimensione di una lunghezza. \newline
1926 Lewis: assegna nome ai fotoni

\section{Spettri atomici}
Spettro solare ha delle righe nere: radiazione viene assorbita dallo strato esterno del sole e poi viene diffusa, quindi ci arriva meno intensa $\Rightarrow$ righe nere. $He$ scoperto grazie allo spettro solare. \newline
1897: scoperta elettrone \newline
Modelli atomici $\rightarrow$ Rutherford: però non riesce a spiegare spettri atomici \newline
Emissione e assorbimento non sono un continuo \newline
Bohr: Energia quantizzata
\begin{equation*}
    \nu_{mn} = (E_m - E_n)\frac{1}{h}
\end{equation*}
ma non dà alcun tipo di spiegazione
\begin{equation*}
    m_e v r = n \hslash \qquad \text{ momento angolare è multiplo intero di } \hslash = \frac{h}{2\pi}
\end{equation*}

\chapter{Interferenza e diffrazione onde elettromagnetiche}
Luce descrivibile con fenomeni ondulatori. \newline
Esperimento doppia fenditura: metto uno schermo a una distanza $L >> \lambda$ dalle fenditure. In ogni punto ho che $ I \neq I_1 + I_2$, ma ho che 
\begin{equation*}
    I = {|\vec{E}_1 + \vec{E}_2|}^2 \neq {|\vec{E}_1|}^2 + {|\vec{E}_2|}^2
\end{equation*}
La differenza di fase vale $d\sin(\theta)$; se la differenza di fase vale $n\frac{\lambda}{2}$ allora ho interferenza distruttiva. Il primo punto di buio si ha per 
\begin{equation*}
    \theta = \frac{n}{2}\frac{\lambda}{d} \text{ con } n=1
\end{equation*}
Se prendo una lampadina gialla (580 nm) da 100 W, allora vengono emessi $\sim 10^{20}$ fotoni al secondo. Cosa succede se abbasso intensità fino ad avere emissione di un fotone al secondo? L'esperimento della doppia fenditura porta allo stesso risultato precedente; però se ripeto di nuovo l'esperimento la posizione di ogni singolo fotone è diverso, anche se la figura che si viene a creare porta allo stesso risultato. Allora devo introdurre una trattazione probabilistica: posso solo dare distribuzione probabilità \newline
Se chiudo una delle due fenditure non osservo alcuna figura di interferenza: il fotone "sente" la presenza di entrambe le fenditure. Ma con chi interferisce il fotone? Da che fenditura passa? \newline
Esperimenti per capire da che fenditura sia passato il fotone distruggono la figura di interferenza $\Rightarrow$ misura microscopica disturba il fenomeno

\end{document}