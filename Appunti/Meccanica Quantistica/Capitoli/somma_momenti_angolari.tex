\chapter{Somma di momenti angolari}

\begin{equation*}
    \hat{J} = \sum_n \hat{L}_n + \sum_n \hat{S}_n
\end{equation*}
La somma di due momenti angolari è ancora un momento angolare

\noindent Date due particelle allora il momento angolare totale vale
\begin{equation*}
    \hat{J} = \hat{J}_1 + \hat{J}_2
\end{equation*}
La dimensione dello spazio H contenente il momento angolare J vale 
\begin{equation*}
    (2j_1+1)(2j_2+1)
\end{equation*}

\noindent Adesso posso usare due diverse basi
\begin{enumerate}
    \item posso usare come CSCO \((\hat{J}^2, \hat{J}_z, \hat{J}^2_1, \hat{J}^2_2)\) cioè la base 
    \begin{equation*}
        |j_1,j_2,j,m>
    \end{equation*}
    \item oppure posso usare come CSCO \((\hat{J}^2_1,\hat{J}^2_2, \hat{J}_{1z},\hat{J}_{2z})\) cioè
    \begin{equation*}
        |j_1,j_2,m_1,m_2>
    \end{equation*}
\end{enumerate}
Per passare da una base all'altra si usa una trasformazione unitaria che utilizza i COEFFICIENTI DI CLEBSH-GORDON

\noindent Proprietà
\begin{itemize}
    \item I coefficienti sono nulli eccetto quando si ha conservazione della proiezione del momento angolare lungo z
    \begin{equation*}
        m = m_1+m_2
    \end{equation*}
    \item I coefficienti sono nulli a meno che 
    \begin{equation*}
        |j_1-j_2| \leq j \leq j_1+j_2
    \end{equation*}
\end{itemize}

\section{Somma del momento di spin 1/2 di 2 particelle}

Stati del tripletto (simmetrici)

\begin{gather*}
    |1,+1> = |++> \\
    |1,0> = \frac{1}{\sqrt[]{2}}[|-+>+|+->] \\
    |1,-1> = |--> 
\end{gather*}
Stato del singoletto (antisimmetrico)

\begin{equation*}
    |0,0> = \frac{1}{\sqrt[]{2}}[|+-> - |-+> ]
\end{equation*}