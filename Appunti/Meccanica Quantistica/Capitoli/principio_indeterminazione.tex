\chapter{Il principio di indeterminazione}
Traiettoria $\rightarrow$ determinare posizione e velocità della particella istante per istante $\Rightarrow$ determino procedura di misura \newline
Per determinare da quale fenditura sia passata la mia particella pongo uno schermo (1 fenditura), ma particella forma figura di diffrazione al passaggio nella fenditura \newline
Facendo un'analisi degli ordini di grandezza, per una particella che passa da una fenditura di larghezza $\Delta y$, con una lunghezza d'onda $\lambda$, che forma un angolo $\theta$ si ha che
\begin{equation*}
    \Delta p_y \simeq p\theta = p \frac{\lambda}{\Delta y} = \frac{h}{\Delta y}
\end{equation*}
ottengo così il principio di indeterminazione di Heisenberg
\begin{equation}
    \Delta y \cdot \Delta p_y \simeq h
\end{equation}
più voglio determinare la posizione (minuisco $\Delta y$) più aumenta l'intervallo $\Delta p_y$; cade così il concetto di traiettoria. \newline
Questa è un'approssimazione: significato completamente diverso da quello che ricaveremo esattamente \newline

\noindent Principio di indeterminazione è l'effetto della perturbazione causata dallo strumento di misura $\rightarrow$ perturbazione ha ruolo essenziale in meccanica quantistica perchè ha lo stesso ordine di grandezza della mia misura
\begin{itemize}
    \item Elettrone: oltre a carica e massa ha anche un momento magnetico, che è una proprietà dell'elettrone $\Rightarrow$ spin $\left(\pm \frac{h}{2}\right)$ [elettrone come piccola calamita]
\end{itemize}