\chapter{Metodi perturbativi}

Vengono utilizzati per determinare autovalori e autofunzioni di hamiltoniane che non dipendono dal tempo; metodi approssimati validi solo per soluzioni analitiche 

\section{Teoria perturbazioni non dipendenti dal tempo}

\begin{equation*}
    \hat{H} |\varphi_n> = E_n |\varphi_n> \qquad  \hat{H} = \hat{H}_0 + \hat{H}_p = \hat{H}_0 + \lambda \hat{H}_1
\end{equation*}
ipotizzo \(\lambda\) un parametro piccolo detto PARAMETRO DI ACCOPPIAMENTO

\noindent Se riesco a risolvere esattamente \(\hat{H}_0\) allora posso cercare soluzioni come serie di potenze di \(\lambda\) che siano rapidamente convergenti in modo da guardare i primi termini della serie per capirne l'andamento

\noindent N.B. \; Non è detto che \(\hat{H}\) siano funzioni analitiche in \(\lambda\)

\noindent Per usare questo metodo devo essere sicuro che il mio sistema sia stabile, in modo che una piccola perturbazione nel sistema non cambi molto

\subsection{Metodo generale calcolo perturbativo}

Sapendo risolvere esattamente per \(\hat{H}_0\) posso riscrivere \(|\psi_n\) sulle autofunzioni \(\varphi_n\)

\noindent Quindi il calcolo PERTURBATO è portato avanti con rappresentazione degli stati imperturbati

\noindent Esistono due diversi casi, il caso in cui \(E^{(0)}_n\) sia non genere e il caso un cui invece sia degenere

\subsection{Caso non degenere}

Posso scrivere 
\begin{equation*}
    |\psi_n> = N(\lambda)\left[|\varphi_n> + \sum_{k \neq 0 }c_{nk}(\lambda)|\varphi_n>\right]
\end{equation*}
Se \(\lambda = 0\) allora
\begin{itemize}
    \item N(0) = 1
    \item \(c_{nk}(0)=0\)
\end{itemize}
Per trovare \(c_{nk}(\lambda)\) cerco soluzioni analitiche espandendo in termini di \(\lambda\); faccio lo stesso con \(E_n\)

\noindent La prima correzione dell'energia all'ordine 1 in \(\lambda\) è il valore di aspettazione della perturbazione tra gli stati imperturbati
\begin{equation*}
    \lambda E^{(1)}_n = <\varphi_n| \hat{H}_p | \varphi_n>
\end{equation*}
Se \(\varphi_n\) ha parità definita e la perturbazione è dispari allora l'integrale è nullo perchè diventa l'integrale di una funzione dispari 

\noindent Calcolando \(c_{nk}(\lambda)\) si trova che la perturbazione mescola gli stati con diversi pesi, proporzionale al valore di aspettazione della perturbazione tra i due stati impertubati diviso per la differenza di energia tra i due stati imperturbati
\begin{equation*}
    \lambda c_{nk}^{(1)} = \frac{\lambda < \varphi_k |\hat{H_1} | \varphi_n> }{E^{(0)}_n - E^{(0)}_k }
\end{equation*}

\noindent Affinchè la serie in \(\lambda\) sia convergente voglio che i coefficienti siano sempre più piccoli; voglio:
\begin{enumerate}
    \item che l'energia all'ordine 1 sia piccola rispetto all'energia all'ordine 0
    \item che gli elementi off the diagonal degli stati  della perturbazione tra gli stati imperturbati siano piccoli rispetto alla differenza di energia degli stati imperturturbati
\end{enumerate}

\noindent All'ordine 2 si ottiene 
\begin{equation*}
    \lambda^2 E^{(2)}_n = \sum_{k \neq n} \frac{{|<\varphi_k | \hat{H}_p | \varphi_n >|}^2}{E^{(0)}_n - E^{(0)}_k}
\end{equation*}

\subsection*{Commenti}

\begin{itemize}
    \item Per l'energia all'ordine 1 ho il valore di aspettazione della perturbazione tra gli stati imperturbati; all'ordine 2 trovo gli elementi fuori dalla diagonale
    \item \(E^{(0)}_n - E^{(0)}_k\): per k diverso da n il denominatore non si annulla mai; gli stati lontani da \(E^{(0)}_n\) contribuiscono sempre meno alla somma: allora ho la speranza di troncare la sommatoria ai primi termini
    \item Se \(\varphi_n\) è lo stato fondamentale allora le correzioni al secondo ordine sono negative 
    \item Suppongo che lo stato k sia "più importante": allora se k è sopra il livello "n" le correzioni al secondo ordine sono negative, quindi il livello n viene spinto verso il basso; se k invece è sotto il livello n allora le correzioni al secondo ordine sono positive e quindi il livello n viene spinto verso l'alto. Si ha quella che viene chiamata TENDENZA DEI LIVELLI A RESPINGERSI
\end{itemize}

\section{Teoria perturbazioni sugli stati degeneri}

Esistono più stati con la stessa energia \(E^{(0)}_n\)

\noindent All'ordine 1 in \(\lambda\) si trova che per risolvere il problema dobbiamo diagonalizzare la perturbazione nel sottospazio degenere
\begin{equation*}
    \sum \alpha_i < \varphi^j_n | \hat{H}_1 | \varphi^i_n> = E^{(1)}_n \alpha_j
\end{equation*}
Troviamo che la correzione all'energia all'ordine 1 ha \(g_n\) valori, dove \(g_n\) rappresenta il numero della degenerazione del livello n-esimo. Per ogni valore di \(E^{(1)}_n\) corrispe un autovettore (\(g_n\) autovettori che diagonalizzano l'hamiltoniana perturbata nel sottospazio degenere)

\noindent N.B. \; Non stiamo diagonalizzando la perturbazione in tutto lo spazio, altrimenti saremmo in grado di trovare gli autostati esatti

