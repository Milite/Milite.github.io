\chapter{Gli atomi idrogenoidi: prima approssimazione}

\begin{itemize}
    \item Nucleo come puntiforme
    \item Trascuro spin
    \item Trascuro effetti relativistici
    \item Interazione solo coulombiana
\end{itemize}
\begin{equation*}
    V(r) = - \frac{Ze^2}{r} \qquad e^2 = \frac{q^2}{4\pi \varepsilon_0}
\end{equation*}
L'hamiltoniana vale
\begin{equation*}
    \hat{H} = \frac{p^2}{2\mu} - \frac{Ze^2}{r}
\end{equation*}
Soluzioni radiali equazione di Schrodinger:
\begin{align*}
    R_{10}(r) &= 2 \bra{\frac{Z}{a_0}}^\frac{3}{2} e^{-\frac{Z r}{a_0}} \\
    R_{20}(r) &= \frac{\sqrt[]{2}}{2} \bra{\frac{Z}{a_0}}^\frac{3}{2} \bra{2-\frac{Zr}{2a_0}} e^{-\frac{Z r}{2a_0}} \\
     R_{21}(r) &= \frac{\sqrt[]{3}}{3} \bra{\frac{Z}{a_0}}^\frac{3}{2} \frac{Zr}{a_0} e^{-\frac{Z r}{a_0}} \\
    R_{30}(r) &= 2 \bra{\frac{Z}{3a_0}}^\frac{3}{2} \bra{1 - 2 \frac{Zr}{3a_0} + 2 \frac{{Zr}^2}{27a_0^2}} e^{-\frac{Z r}{3a_0}} \\
     R_{31}(r) &= \frac{4\sqrt[]{2}}{9} \bra{\frac{Z}{3a_0}}^\frac{3}{2} \bra{1-\frac{Zr}{6a_0}} e^{-\frac{Z r}{3a_0}} \\
     R_{32}(r) &= \frac{2\sqrt[]{2}}{27\sqrt[]{5}} \bra{\frac{Z}{3a_0}}^\frac{3}{2} \bra{\frac{Zr}{a_0}}^2 e^{-\frac{Z r}{3a_0}}
\end{align*}
Autofunzioni:
\begin{align*}
    \varphi_{100} &= \frac{1}{\sqrt[]{\pi a_0^3}} e^{-\frac{r}{a_0}} \\
    \varphi_{200} &= \frac{1}{\sqrt[]{8\pi a_0^3}} \bra{1-\frac{r}{2a_0}} e^{-\frac{r}{2a_0}} \\
    \varphi_{211} &= -\frac{1}{\sqrt[]{8\pi a_0^3}} \frac{r}{a_0} e^{-\frac{r}{2a_0}} \sin\theta e^{i\varphi} \\
    \varphi_{210} &= \frac{1}{4\; \sqrt[]{\pi a_0^3}} \frac{r}{a_0} e^{-\frac{r}{2a_0}} \cos\theta \\
    \varphi_{21-1} &= \frac{1}{8\; \sqrt[]{\pi a_0^3}} \frac{r}{a_0} e^{-\frac{r}{2a_0}} \sin\theta e^{-i\varphi} 
\end{align*}
Condizioni numeri quantici:
\begin{gather*}
    l = 0,1,2,\dots, n-1 \\
    m = -l,-l+1,\dots,0,\dots,l-1,l
\end{gather*}
Quantità utili:
\begin{equation*}
    r_n = \frac{n^2\hslash^2}{\mu Ze^2} \qquad v_n = \frac{Ze^2}{n\hslash} \qquad E_n = \frac{1}{2}\frac{Z^2\mu e^4}{n^2\hslash^2}
\end{equation*}
\begin{equation*}
    m_p \sim 2000 m_e
\end{equation*}
Costante di accoppiamento elettromagnetico
\begin{equation*}
    \alpha = \frac{e^2}{c\hslash} = \frac{q^2}{4\pi \varepsilon_0 c\hslash}
\end{equation*}