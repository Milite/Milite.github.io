\chapter{Formule}

\section{Equazione Schrodinger per una particella libera 1D}
\begin{equation*}
    i\hslash \frac{\delta \psi}{\delta t} = -\frac{\hslash^2}{2m}\frac{\delta^2 \psi}{\delta x^2}
\end{equation*}

\section{Equazione Schrodinger per un potenziale arbitrario V(x)}
\subsection{1D}
\begin{equation*}
    i\hslash \frac{\delta \psi}{\delta t } = \left[-\frac{\hslash^2}{2m}\frac{\delta^2}{\delta x^2} + V(x)\right]\psi
\end{equation*}
\subsection{3D}
\begin{equation*}
    i\hslash \frac{\delta}{\delta t} \psi(t,\vec{x}) = -\frac{\hslash^2}{2m} \nabla^2 \psi(t,\vec{x}) + V(t,\vec{x})\psi(t,\vec{x})
\end{equation*}

\section{Operatori}
\begin{itemize}
    \item $\hat{H} = \frac{\hat{p}}{2m} + V(\hat{x})$ \qquad hamiltoniana
    \item $\hat{a} = \frac{1}{\sqrt[]{2}}\left(\sqrt[]{\frac{m\omega}{\hslash}}\hat{x}+ i \frac{1}{\sqrt[]{m\hslash\omega}}\hat{p}\right)$
    \item $\hat{a}^+ = \frac{1}{\sqrt[]{2}}\left(\sqrt[]{\frac{m\omega}{\hslash}}\hat{x} - i \frac{1}{\sqrt[]{m\hslash\omega}}\hat{p}\right)\qquad$  creatore 
    \item $\hat{x} = \sqrt[]{\frac{\hslash}{2m\omega}}(\hat{a}^+ + \hat{a}) \qquad$ distruttore
    \item $\hat{p} = i \; \sqrt[]{\frac{m\hslash\omega}{2}}(\hat{a}^+ - \hat{a})$ 
    \item $\hat{N} = (\hat{a}^+\hat{a})$ \qquad operatore numero
\end{itemize}
\subsection{Rappresentazione coordinate}
\begin{itemize}
    \item $\hat{p}_x = -i\hslash \frac{\delta}{\delta x}$
    \item $\hat{x} = x$
    \item $\hat{H} =-\frac{\hslash^2}{2m}\nabla^2 + V(\vec{x})$
\end{itemize}
\subsection{Rappresentazione momenti}
\begin{itemize}
    \item $\hat{p}_x = p$
    \item $\hat{x} = i\hslash\frac{\delta}{\delta p}$
\end{itemize}
\subsection{Applicazione operatori su stati}
\begin{itemize}
    \item $a^+|n> = \sqrt[]{n+1}|n+1>$
    \item $a|n> = \sqrt[]{n}|n-1>$
    \item $\hat{x}|n> = \sqrt[]{\frac{\hslash}{2m\omega}}(\sqrt[]{n+1}|n+1> + \sqrt[]{n}|n-1>)$
    \item $\hat{p}|n> = \sqrt[]{\frac{m\hslash\omega}{2}}(\sqrt[]{n+1}|n+1> - \sqrt[]{n}|n-1>)$
\end{itemize}

\section{Commutatore}
\begin{itemize}
    \item $[\hat{A},\hat{B}] = \hat{A}\hat{B}-\hat{B}\hat{A}$
    \item $[\hat{x},\hat{p}_x] = i\hslash$
    \item $[\hat{a},\hat{a}^+] = 1$
\end{itemize}

\section{Stati stazionari}
\begin{equation*}
    \hat{H}\varphi = E\varphi  
\end{equation*}

\section{Continuità della funzione d'onda e della derivata}

\begin{itemize}
    \item $V \to \infty \Rightarrow \varphi = 0$
    \item Salto in $x = x_0 \Rightarrow \dot{\varphi} \in C^1$ (la funzione non ha salti)
    \item V(x) ha delta in $x = x_0 \Rightarrow V = V_0 \delta(x-x_0)$
    \begin{equation*}
        \dot{\varphi}(x_0 + \varepsilon) - \dot{\varphi}(x_0 - \varepsilon) = \frac{2m}{\hslash}V_0 \varphi(x_0)
    \end{equation*}
\end{itemize}

\section{Soluzione equazione stati stazionari}

\begin{enumerate}
    \item Soluzione del tipo oscillatorio
    \begin{equation*}
        \frac{d^2 \psi}{dx^2} + \alpha^2\psi = 0
    \end{equation*}
    Le soluzioni sono del tipo
    \begin{equation*}
        c_1\sin(\alpha x) + c_2\cos(\alpha x)
    \end{equation*}
    \item Soluzioni del tipo esponenziali
    \begin{equation*}
        \frac{d^2 \psi}{dx^2} - \alpha^2\psi = 0
    \end{equation*}
    Le soluzioni sono del tipo
    \begin{equation*}
        c_1 e^{\alpha x} + c_2 e^{-\alpha x}
    \end{equation*}
\end{enumerate}

\section{Particella in una scatola con potenziale infinito}
\subsection{Energia}
\begin{equation*}
    E_n = \frac{\pi^2\hslash^2}{2ma^2}n^2
\end{equation*}
\subsection{Autofunzioni}
\begin{equation*}
    \varphi_n = \sqrt[]{\frac{2}{a}}\sin\left(\frac{n\pi x}{a}\right)
\end{equation*}

\section{Evoluzione temporale}
\begin{equation*}
    \psi(t,\vec{x}) = \sum c_n\varphi_n e^{-\frac{iE_nt}{\hslash}} + \int dk \, c(k)\varphi_ke^{-\frac{iE_kt}{\hslash}}
\end{equation*}

\section{Oscillatore armonico}
\subsection{Hamiltoniana}
\begin{equation*}
    H = \frac{p^2}{2m} + \frac{1}{2}m\omega^2x^2
\end{equation*}

\subsection{Energia}
\begin{equation*}
    E_n = \hslash\omega \left(n + \frac{1}{2}\right)
\end{equation*}

\chapter{Per esame}

\section{Traformata di Fourier}
\begin{equation*}
    \hat{f}(k) = \frac{1}{\sqrt[]{2\pi}} \int dx \, e^{-ikx}f(x) \qquad k = \frac{p}{\hslash}
\end{equation*}

\section{Rappresentazione coordinate}
\begin{gather*}
    \hat{p}_x = -i\hslash \frac{\delta}{\delta x} \\
    \hat{x} = x 
\end{gather*}

\section{Rappresentazione coordinate}
\begin{gather*}
    \hat{p} = p \\
    \hat{x} = i \hslash \frac{\delta}{\delta p}
\end{gather*}

\section{Operatore hamiltoniana}
\begin{equation*}
    \hat{H} = \frac{\hat{p}^2}{2m} + V(\hat{x})
\end{equation*}

\section{Valor medio}
\begin{gather*}
    <A> = \int dx \, \psi^*(x)A\psi(x) \\
    <A> = \int dx \, \tilde{\psi}^*(x) A \tilde{\psi}(x)
\end{gather*}

\section{Buche di potenziale}

\begin{itemize}
    \item Particella in una scatola con potenziale infinito
        \subsection*{Energia}
        \begin{equation*}
            E_n = \frac{n^2 \pi^2 \hslash^2}{2a^2m}
        \end{equation*}
        \subsection*{Autofunzioni}
        \begin{equation*}
            \varphi_n = \sqrt[]{\frac{2}{a}}\sin \bra{\frac{n\pi x}{a}}
        \end{equation*}
\end{itemize}

\section{Equazione Schrodinger}
\begin{equation*}
    \ddot{\psi} + \frac{2m}{\hslash^2}[E-V(x)]\psi(x) = 0
\end{equation*}

\section{Oscillatore armonico}
\begin{equation*}
    \hat{x}|n> = \sqrt[]{\frac{\hslash}{2m\omega}} (\sqrt[]{n+1}|n+1> \sqrt[]{n}|n-1>)
\end{equation*}

\section{Formule trigonometriche}

\begin{equation*}
    \sin (2x) = 2 \sin x \cos x
\end{equation*}

