\documentclass[a4paper,11pt]{report}

\usepackage[T1]{fontenc}
\usepackage[utf8]{inputenc}
\usepackage[italian]{babel}

\usepackage{wrapfig}
\usepackage{mathtools}
\usepackage{graphicx}
\usepackage{amsfonts}
\usepackage{amsthm}
\usepackage{amsmath}
\usepackage{amssymb}
\usepackage{fancyhdr}
\usepackage{float}
\usepackage{geometry}
\geometry{a4paper, top=2.5cm, bottom=2cm, left=2cm, right=2cm}
\usepackage{hyperref}
\hypersetup{
	colorlinks=true,
	linkcolor=black,
	filecolor=blue,
	citecolor = black,      
	urlcolor=cyan,
}

\swapnumbers
\theoremstyle{remark}
\newtheorem*{oss}{Oss}
\newtheorem*{nb}{N.B}
\newtheorem*{coro}{Corollario}
\theoremstyle{definition}
\newtheorem*{teo}{Teorema}
\newtheorem*{Def}{Def}
\newtheorem*{prop}{Prop}

\newcommand{\C}{\mathbb{C}}
\newcommand{\R}{\mathbb{R}}

\DeclarePairedDelimiter{\abs}{\lvert}{\rvert}

\begin{document}
	\date{}
	\author{Marco Militello}
	\title{Meccanica classica}
	\maketitle
	\tableofcontents
	\newpage

$(q_1,\dots,q_d) \to (\tilde{q}_1,\dots,\tilde{q}_d) $

\[\begin{cases}
	q_j = q_j (\tilde{q}_1,\dots, \tilde{q}_d) \to \dot{q}_j = \dot{q}_j(\tilde{\underline{q}},\dot{\underline{\tilde{q}}}) \qquad J_{kj} = \frac{\delta q_k}{\delta \tilde{q}_j} \quad \det J \ne 0\\
	j = 1,\dots,d 
\end{cases}\]

$\begin{cases}
	\tilde{q}_j (q_1,\dots,q_d) \\
	j=1,\dots,d
\end{cases}$
\newline

\noindent $\mathcal{L} (q_1,\dots,q_d,\dot{q}_1,\dots,\dot{q}_d) \qquad \tilde{\mathcal{L}}(\underline{\tilde{q}},\underline{\dot{\tilde
q}}) = \mathcal{L} (\underline{q}(\underline{\tilde{q}}),\dot{q}(\underline{\tilde{q}},\underline{\dot{\tilde{q}}}))$

\begin{prop}
	q(t) è soluzione E-L per $\mathcal{L}(\underline{q},\underline{\dot{q}}) \iff \tilde{q}(t)$ è soluzione per $\mathcal{L}(\underline{\tilde{q}}, \underline{\dot{\tilde{q}}})$
\end{prop}

\begin{proof}
	$(\Rightarrow) \; q(t)$ è soluzione E-L per $\mathcal{L}$ 
	\[\frac{d}{dt}\frac{\delta \tilde{\mathcal{L}}}{\delta \dot{\tilde{q}}_j} = \frac{d}{dt} \left(\sum_{i=1}^d \frac{\delta \mathcal{L}}{\delta \dot{q}_i} \frac{\delta \dot{q}_i}{\delta \dot{\tilde{q}}_j} \right) = \frac{d}{dt} \left(\sum_{i=1}^d \frac{\delta \mathcal{L}}{\delta \dot{q}_i} \overbrace{\frac{\delta q_i}{\delta \tilde{q_j}}}^{lemma 1}\right) = \sum_{i=1}^d \left[\left(\frac{d}{dt}\frac{\delta\mathcal{L}}{\delta \dot{q}_i}\right)\frac{\delta q_i}{\delta \dot{\tilde{q}}_j} + \frac{\delta \mathcal{L}}{\delta\dot{q}_i}\underbrace{\frac{d}{dt}\frac{\delta q_i}{\delta \tilde{q}_j}}_{lemma 2}\right] =\]
	\[=\sum_{i=1}^d \left[\left(\frac{d}{dt}\frac{\delta\mathcal{L}}{\delta\dot{q}_i}\right)\frac{\delta q_i}{\delta \dot{q}_j} + \frac{\delta\mathcal{L}}{\delta \dot{q}_i}\frac{\delta \dot{q}_i}{\delta\tilde{q}_j}\right]\]
	Uso ipotesi
	\[= \sum_{i=1}^d \left(\frac{\delta \mathcal{L}}{\delta q_i}\frac{\delta q_i}{\delta \dot{\tilde{q}}_j} + \frac{\delta\mathcal{L}}{\delta\dot{q}_i} \frac{\delta\dot{q}_i}{\delta\tilde{q}_j}\right) = \frac{\delta\tilde{\mathcal{L}}}{\delta\tilde{q}_j}\]
\end{proof}

\begin{prop}
	$\forall$ scelta di F(\underline{q},t) e $\alpha \ne 0 \in \mathbb{R}$ allora 
	\[\mathcal{L}(\underline{q},\underline{\dot{q}},t) \quad \mathcal{L}' (\underline{q},\underline{\dot{q}},t) = \alpha \mathcal
	(\underline{q},\underline{\dot{q}},t) + \frac{dF}{dt}\]
	conducono alla stessa soluzion per E-L
\end{prop}

\begin{proof}\hfill\\
	$L_0 = \frac{dF}{dt} = \dot{F} \qquad \mathcal{L}' = \alpha\mathcal{L}+ L_0$
	\[\frac{d}{dt}\frac{\delta\mathcal{L}'}{\delta \dot{q}_j} = \alpha \frac{d}{dt}\frac{\delta\mathcal{L}}{\delta \dot{q}_j} + \frac{d}{dt}\frac{\delta L_0}{\delta \dot{q}_j}\]	
	$\frac{d}{dt}\frac{\delta L_0}{\delta \dot{q}_j} = \frac{d}{dt}\frac{\delta \dot{F}}{\delta \dot{q}_j} \underbrace{=}_{Lemma 1} \frac{d}{dt}\frac{\delta F}{\delta q_j} \underbrace{=}_{Lemma 2} \frac{\delta \dot{F}}{\delta \dot{q}_j} = \alpha \frac{d}{dt}\frac{\delta \mathcal{L}}{\delta \dot{q}_j} + \frac{\delta L_0}{\delta q_j}$
\end{proof}

\end{document}