\documentclass[a4paper,11pt]{report}

\usepackage[T1]{fontenc}
\usepackage[utf8]{inputenc}
\usepackage[italian]{babel}

\usepackage{wrapfig}
\usepackage{mathtools}
\usepackage{graphicx}
\usepackage{amsfonts}
\usepackage{amsthm}
\usepackage{amsmath}
\usepackage{amssymb}
\usepackage{fancyhdr}
\usepackage{float}
\usepackage{geometry}
\geometry{a4paper, top=2.5cm, bottom=2cm, left=2cm, right=2cm}
\usepackage{hyperref}
\hypersetup{
	colorlinks=true,
	linkcolor=black,
	filecolor=blue,
	citecolor = black,      
	urlcolor=cyan,
}

\swapnumbers
\theoremstyle{remark}
\newtheorem*{oss}{Oss}
\newtheorem*{nb}{N.B}
\newtheorem*{coro}{Corollario}
\newtheorem*{lemma}{Lemma}
\theoremstyle{definition}
\newtheorem*{teo}{Teorema}
\newtheorem*{Def}{Def}

\newcommand{\C}{\mathbb{C}}
\newcommand{\R}{\mathbb{R}}

\DeclarePairedDelimiter{\abs}{\lvert}{\rvert}
\DeclarePairedDelimiter{\norma}{\lVert}{\rVert}

\begin{document}
	\date{}
	\author{Marco Militello}
	\title{Analisi II}
	\maketitle
	\tableofcontents
	\newpage
	
\chapter{Spazi funzionali ed equazioni differenziali}	
\section{Spazi metrici}
Una METRICA su un insieme $X$ è una funzione
\[d(distanza): \, X\times X \to [0,\infty) \subseteq \R\]
che gode delle seguenti proprietà:
\begin{itemize}
	\item $d(x,y) \geq 0 \qquad d(x,y) =0 \iff x=y$
	\item $d(x,y)=d(y,x)$ \quad simmetria
	\item $d(x,y) \leq d(x,z)+d(z,y)$ \quad disuguaglianza triangolare
\end{itemize}
\noindent $X$ è uno spazio metrico
\begin{description}
	\item[metrica euclidea] $\displaystyle d(x,y) =\norma{x-y} = \sqrt{\sum_{j=1}^{n} {(x_j-y_j)}^2}$
	\item[metrica Manhattan] $\displaystyle d(x,y) = \sum_{j=1}^{n} \abs{x_j-y_j}$
	\item[metrica discreta] $d(x,y) = \begin{cases} 1 & x \ne y \\ 0 & x=y \end{cases} $
\end{description}
\begin{Def}[\textbf{Metriche equivalenti}]\hfil\\
	Due metriche si dicono equivalenti quando danno luogo alla stessa famiglia di aperti (cioè se gli intorni di una sono gli aperti dell'altra)
\end{Def}
\begin{Def}[\textbf{Norma}]\hfil\\
	dato V spazio vettoriale su $\R$. Una norma su V è 
	\[N: \, V \to R\]
	tale che 
	\begin{enumerate}
		\item $N(v) \geq 0 \qquad N(v)=0 \iff v=0$
		\item $N(\lambda v)= \abs{\lambda}N(v) \quad \lambda \in \R$
		\item $N(v+w) \leq N(v)+N(w)$
	\end{enumerate}
\end{Def}
\begin{description}
	\item[norma euclidea] $ \displaystyle \norma{x}= \sqrt{\sum_{j=1}^{n} x^2_j}$
	\item[norma uniforme] ${\norma{x}}_\infty = \max_{j=1,\dots,n} \abs{x_j}$
	\item[norma $L^1$] $\displaystyle {\norma{x}}_1 = \sum_{j=1}^{n} \abs{x_j}$
\end{description}
\noindent Se N è una norma, allora $d(x,y)=N(x-y)$ è una metrica
\begin{Def}[\textbf{Prodotto scalare}]\hfil\\
	Un prodotto scalare su uno spazio vettoriale $V \subseteq \R$ è
	\begin{gather*}
	S: \, V \times V  \to \R \\
	S(v,w) \mapsto v \cdot w 
	\end{gather*}
	\begin{enumerate}
		\item $S(v,v) \geq 0 \qquad S(v,v)=0 \iff v=0$
		\item $S(v,w)=S(w,v)$
		\item $S(au +bv,w) = aS(u,w) + bS(v,w)$
	\end{enumerate}
\end{Def}
\noindent Se S è un prodotto scalare allora 
\[N(v) = {(S(v,v))}^{1 \over 2}\]
\section{Continuità e successioni}
\begin{Def}
	\begin{gather*}
		X \rightarrow Y \\
		\lim_{x \to x_0} f(x) = y_0 \iff \forall \varepsilon>0 \, \exists \delta(\varepsilon)>0 \,: \forall x \in \underbrace{0<d_x(x,x_0)<\delta}_{B_\delta (x_0)} \Rightarrow f(x)\in \underbrace{d_y(f(x),y_0)<\varepsilon}_{ B_\varepsilon (y_0)}
	\end{gather*}
\end{Def}
\begin{Def}\hfil\\
	$f$ continua in $x_0$ se
	\[\lim_ {x \to x_0} f(x) =f(x_0)\]
\end{Def}
\begin{teo}
	$f: \, x \rightarrow Y$ è continua ($\forall x \in X$) $\iff f^{-1}$ è un aperto $\forall A \subset Y$ aperto \newline
	Allora la composizione di funzioni continue è continua
\end{teo}
\begin{Def}[\textbf{Funzione limitata}]\hfill\\
	Una funzione si dice limitata se l'immagine $f(X) \subset Y$ è limitata in Y 
\end{Def}
\begin{Def}
	\[d_\infty (f,g) = \sup_{x \in X} d_\infty (f(x),g(x)) \qquad d_\infty \text{ metrica uniforme}\]
\end{Def}
\begin{Def}
	\[C(X,Y) \subset B(X,Y)\]
	è l'insieme delle funzioni continue e limitate
\end{Def}
\begin{Def}[Funzione Lipshitziana]\hfill\\
	$f:\, X\rightarrow Y$ è Lipschitziana in $X$ se 
	\[\exists k \in \R \; | \; \forall x,y \in X \Rightarrow d_y (f(x),f(y)) \leq k d_x(x,y)\]
	Se una funzione è Lipshitziana allora è anche continua
\end{Def}
\section{Successioni}
\begin{Def}\hfil\\
$x_n \to \bar{x}$ converge a $\bar{x}$ (in $X$ spazio metrico) se 
\[\forall \varepsilon>0 \; \exists N >>0 \;|\; n>N \Rightarrow d(x_n,\bar{x}) <\varepsilon\]
\end{Def}
\begin{Def}[\textbf{Successione di Cauchy}]
	\[\forall \varepsilon>0 \; \exists N>>0 \;|\; n,m>N \Rightarrow d(x_n,x_m)<\varepsilon\]
\end{Def}
\begin{teo}
	$f_n \to \bar{f}$ in X metrica uniforme $\iff f_n \mapsto \bar{f}$ (converge uniformemente)
\end{teo}
\section{Completezza}
\begin{Def}[\textbf{Spazio metrico completo}]\hfil\\
	$X$ è uno spazio metrico completo $\iff$ ogni successione di Cauchy in $X$ è convergente in $X$
\end{Def}
\begin{lemma}
	Se $X$ è completo e $C \subset X$ è un sottospazio chiuso $\Rightarrow$ $C$ è completo
\end{lemma}
\begin{Def}[\textbf{Spazio di Banach}]\hfil\\
	Uno spazio di Banach è uno spazio vettoriale completo dotato di norma
\end{Def}
\begin{Def}[\textbf{Spazio di Hilbert}]\hfill\\
	Uno spazio di Hilbert è uno spazio vettoriale completo dotato di prodotto scalare
\end{Def}
\begin{lemma}
	Hilbert $\Rightarrow$ Banach
\end{lemma}
\begin{teo}
	Sia $S$ un insieme, $Y$ uno spazio metrico, $d_\infty$ metrica uniforme e $B(S,Y) =$ funzioni limitate da $ S\to Y$ \newline
	Se $Y$ è completo $\Rightarrow B(S,Y)$ è completo  
\end{teo}
\begin{teo}
	X spazio metrico, Y spazio metrico completo e $C(X,Y) \subset B(X,Y)$ funzioni continue e limitate con metrica uniforme $d_\infty$ \newline
	$C(X,Y)$ è completo
\end{teo}

\end{document}