\chapter{Evoluzione stellare}

$M = 15 M_{sole}$: 
\begin{itemize}
    \item brucia H in He;  $10^7 yr$ sulla sequenza principale
    \item $He \rightarrow C,O,Ne \Rightarrow$ passa $10^6 yr$ a bruciare He: passa nel ramo delle giganti
    \item $C \rightarrow Ne,Mg, Na \Rightarrow 10^3 \mbox{yr}$ 
    \item $Ne \rightarrow O,Mg \Rightarrow 0.7 \mbox{yr}$
    \item $O \rightarrow Si,Ar,Ca \Rightarrow 2.6 \mbox{yr}$
    \item $Si\rightarrow Fe, Ni \Rightarrow 18 \mbox{days}$
\end{itemize}
\begin{enumerate}
    \item FOTODISINTEGRAZIONE 
    \begin{equation*}
        \gamma + ^{56}Fe \rightarrow 13 \,  ^4He + 4n 
    \end{equation*}
    \begin{equation*}
        \gamma + ^4He \rightarrow 2p + 2n 
    \end{equation*}
    \item NEUTRONIZZAZIONE (inverse $\beta$ decay)
    \begin{equation*}
        e^- + p \rightarrow n + \nu_e
    \end{equation*}
    $\rho \sim 10^{11} gcm^{-3} \Rightarrow n_e=n_p = \frac{1}{10}n_n$ 
\end{enumerate}
\begin{itemize}
    \item $8 < M < 20M_{sole} \Rightarrow$ stella di neutroni
    \item $M > 20 M_{sole} \Rightarrow$ buco nero
\end{itemize}
In entrambi i casi si ha prima un'esplosione di supernova; se la massa della stella però troppo massiva si può avere direttamente un buco nero senza avere l'esplosione di supernova. Durante queste esplosioni avviene rilascio degli stati più esterni, esplosione è quindi fondamentale anche per la nostra vita.
