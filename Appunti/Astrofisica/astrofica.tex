\documentclass[a4paper,11pt]{report}

\usepackage[T1]{fontenc}
\usepackage[utf8]{inputenc}
\usepackage[italian]{babel}

\usepackage{mathdots}
\usepackage{mathtools}
\usepackage{graphicx}
\usepackage{amsfonts}
\usepackage{amsthm}
\usepackage{amsmath}
\usepackage{amssymb}
\usepackage{fancyhdr}
\usepackage{float}
\usepackage{geometry}
\geometry{a4paper, top=2.5cm, bottom=2cm, left=2cm, right=2cm}
\usepackage{hyperref}
\hypersetup{
	colorlinks=true,
	linkcolor=black,
	filecolor=blue,
	citecolor = black,      
	urlcolor=cyan,
}

\swapnumbers
\theoremstyle{remark}
\newtheorem*{oss}{Oss}
\newtheorem*{nb}{N.B}
\newtheorem*{coro}{Corollario}
\theoremstyle{definition}
\newtheorem*{teo}{Teorema}
\newtheorem*{Def}{Def}

\newcommand{\C}{\mathbb{C}}
\newcommand{\R}{\mathbb{R}}

\DeclarePairedDelimiter{\abs}{\lvert}{\rvert}

\begin{document}
	\date{}
	\author{Marco Militello}
	\title{Astrofisica}
	\maketitle
	\tableofcontents
	\newpage
	
\chapter{Introduzione}

\section*{Spettro elettromagnetico}
Spettro corpo nero: 
$\begin{cases}
    \nu_{max} = 10^{11} Hz \\
    \lambda_{max} T = 3 \times 10^6 nm \cdot K
\end{cases}$ \newline
Osservazione di tutto lo spettro elettromagnetico: telescopi terrestri e telescopi spaziali a seconda della frequenza

\section*{Tecniche di osservazione}
Risoluzione angolare $\rightarrow \theta \simeq 1.22 \frac{\lambda}{D}$ con $\lambda$ lunghezza d'onda e $D$ diametro del telescopio; per l'occhio si ha una risoluzione angolare di 60 arcsec: se la dimensione dell'oggetto da osservare è minore di 60 arcsec allora l'occhio lo vede come un punto; se la distanza tra due oggetti è minone di 60 arcsec allora li vediamo sovrapposti. \newline
Gli effetti atmosferici peggiorano la visione: per esempio l'umidità, che deve essere bassa. \newline
Tecniche interferometriche migliorano la risoluzione (esempio l'utilizzo di diversi telescopi). \newline
Oltre a misurare la posizione degli oggetti si può misurare il flusso, lo spettro e la polarizzazione

\section*{Coordinate}
\begin{itemize}
    \item Celesti: per astronomia amatoriale, non professionale
    Da equatore celeste divido la volta celeste in 2 emisferi
    \begin{itemize}
        \item Nord 
        $\begin{cases}
            \text{Declinazione: da 0° a 90° (asse di rotazione terrestre)} \\
            \text{Ascensione retta: da 0 a 24 ore} 
        \end{cases}$
        \item Sud
        $\begin{cases}
            \text{Declinazione: da 0° a -90° (asse di rotazione terrestre)} \\
            \text{Ascensione retta: da 0 a 24 ore}
        \end{cases}$
    \end{itemize}
    \item Galattiche: da piano della galassia in cui ci troviamo. 
    \begin{itemize}
        \item Latitudine: da -90° a 90° $\rightarrow$ 0° piano della galassia
        \item Longitudine: da 0° a 360° $\rightarrow$ senso antiorario
    \end{itemize}
\end{itemize}
Centro galattico è il centro della galassia

\section*{Proiezioni}
\begin{itemize}
    \item Proiezione azimutale equidistante: centro latitudine corrisponde a 90°, mentre l'esterno corrisponde a 0°; si hanno minime distorsioni, ma visualizzo solo metà volta celeste
    \item Proiezione di Hammer-Aitoff: rappresentazione completa della volta celeste, ma si hanno forti distorsioni ai poli. Il piano della galassia è il piano centrale orizzontale. Solitamente in astrofisica si usa questa rappresentazione
\end{itemize}

\section*{Visulizzazioni}
\begin{itemize}
    \item Ottico: a causa delle polveri il centro della galassia è molto oscurato. L'assorbimento galattico è descritto dalla funzione
    \begin{equation*}
        I= I_0 e^{-\alpha r}
    \end{equation*}
    dove r è la distanza tra noi e l'oggetto osservato, mentre $\alpha$ è proporzionale a $\lambda^{-1}$
    \item Infrarosso: l'ssorbimento galattico è minore, quindi si vede molto meglio il centro della galassia (si rimuove parte delle polveri). Si possono trovare alcune bande infrarosso anche sulla terra. Anche noi emettiamo infrarosso, quindi rischio di avere rumore termico nelle mie misure: posso tenere il telescopio a basse temperature per diminuire il problema. \newline
    Dall'infrarosso in poi si usano falsi colori: associo colore in base all'intensità della radiazione.
    \item Microonde: porta a scoperta della radiazione di fondo cosmica
    \item Radio: porta a due scoperte
    \begin{enumerate}
        \item Stella di neutroni, grazie ad un segnale periodico; pulsar non si vedono nell'ottico
        \item Quasar
    \end{enumerate}
    \item Raggi-x: si vedono solo in orbita, non a terra. Portano prima prova dell'esistenza dei buchi neri: si capta emissione di materia che cadeva nel buco nero. Il buco nero invece non emette radiazione.
    \item Raggi-$\gamma$: porta scoperta dei gamma-ary bursts. \newline
    Neutrini: particelle che interagiscono per interazione debole (anche il sole li emette). Esplosione supernova è una sorgente di neutrini; l'unica prova diretta di un'esplosione di una supernova è SN1987A
\end{itemize}

\section*{Onde gravitazionali}
Sono state captate da interferometri a terra nel 2015, ma predette già dalla relatività generale. Le distorsioni dello spazio-tempo si propagano come un'onda. \newline
Le sorgenti più luminose sono le binare, 2 oggetti compatti che ruotano.

\section*{Studio spettro elettromagnetico}
Oltre all'intensità della stella possiamo anche vedere lo spettro $\Rightarrow$ spettro quasi perfetto di corpo nero
\begin{equation*}
    U_\nu = \frac{8\pi \nu^3 }{c^3} \frac{h\nu}{e^{\frac{h\nu}{K_B t}}-1} \quad \text{densità di energia a una certa frequenza}
\end{equation*}
Densità di energia
\begin{equation*}
    u = \int _0^\infty d\nu \, U_\nu = aT^4
\end{equation*}
a: costante di radiazione e vale $7.6 \times 10^{-15} erg \, cm^{-3}K^{-4}$ \newline
Flusso per unità di frequenza misurato in $erg \, s^{-1} cm^{-2}{Hz}^{-1} \rightarrow$ $f_\nu$\newline
Densità di flusso (vettore di Pointing)
\begin{equation*}
    f = \int_0^\infty d\nu \, f_\nu = \sigma T^4
\end{equation*}
$\sigma$: costante di Stefan-Boltzman che vale $5.7 \times 10^{-5} erg \, s^{-1} {cm}^{-2}K^{-4}$ \newline
Potenza totale emessa da stella ad una certa frequenza assumendo stella come sfera perfetta misurata in $erg \, s^{-1}{Hz}^{-1}$
\begin{equation*}
    L_\nu = f_\nu (r_*) 4\pi r_*^2
\end{equation*}
Luminosità misurata in $erg \, s^{-1}$
\begin{equation*}
    L = f(r_*)4\pi r_*^2
\end{equation*}
Le onde elettromagnetiche di tipo sferico vanno come $\sim \frac{1}{{distanza}^2}$ 
\begin{equation*}
    f(d) = f(r_*){\left(\frac{r}{d}\right)}^2 = f(r_*)\frac{4\pi {r_*}^2}{4\pi d^2} = \frac{L}{4\pi d^2}
\end{equation*}
\begin{equation}
    L = 4\pi d^2 f(d)
\end{equation}
se vedo oggetto ho sempre il flusso; conoscendo anche distanza da oggetto posso anche stimare la sua luminosità

\subsection*{Metodo parallasse}
Metodo per stimare la distanza di un oggetto sfruttando il moto apparente di questo oggetto nel cielo; necessita la conoscenza della distanza tra noi e il sole come punto di riferimento \newline

\noindent In astrofisica come unità di misura per le distanze si usa il parsec: distanza tale per cui l'angolo $\alpha$ del moto apparente è di un arcosecondo
\begin{equation*}
    1 pc \simeq 3.1 \times 10^{18} cm \simeq 3.3 ly
\end{equation*}
La stella più vicina a noi è Proxima Centauri che dista $d \simeq 1.3 pc \rightarrow \alpha \sim 0.768 arcsec$ \newline
A causa del potere risolutivo con metodo parallasse posso stimare distanze oggetti fino ad una distanza di $\sim 100 \, pc$   

\begin{equation*}
    L =4\pi {r_*}^2 f(r_*) \underbrace{=}_{\text{per corpo nero}} 4\pi {r_*}^2 \sigma T^4
\end{equation*}
La temperatura la ottengo facendo un fit dei dati con la curva del corpo nero; dato che conosco la luminosità posso calcolare il raggio della stella. La temperatura che posso stimare è la temperatura superficiale della stella, perchè i fotoni che ci arrivano sono quelli della superficie

\subsection*{Magnitudine}
Unità di misura che è stata introdotta dagli antichi greci
\begin{itemize}
    \item 0: stella più luminosa che vedo a occhio nudo
    \item 5: stella più debole visibile a occhio nudo
\end{itemize}
Si può legare magnitudine a logaritmo del flusso

\subsubsection*{Magnitudine apparente m}
Da un punto di vista fisico è uguale al flusso
\begin{equation*}
    m = cost - 2.5 \log_{10}f
\end{equation*}
devo scegliere come definire la costante \newline

\noindent $\begin{cases}
    \text{Scelgo costante in modo che m di Vega sia 0} \rightarrow m = -2.5 \log_{10}\left(\frac{f}{f_{Vega}}\right) \\
    \text{Sistema AB} \rightarrow m = -2.5 \log_{10}\left(\frac{f}{f_0}\right) \text{ con $f_0$ costante per ogni frequenza}
\end{cases}$

\subsubsection*{Magnitudine assoluta M}
Da un punto di vista fisico è analogo luminosità. \'E la distanza che dovrebbe avere un oggetto se si trovasse ad una distanza di 10 parsec
\begin{equation*}
    M = m - 5\log_{10} \left(\frac{d}{10 \, pc}\right)
\end{equation*}
\begin{equation*}
    M = M_{sole} - 2.5 \log_{10}\left(\frac{L}{L_{sole}}\right) \qquad M_{sole} \simeq 4.75 \; L_{sole} \simeq 3.8 \times 10^{33} erg \, s^{-1} \simeq 3.8 \times 10^{26} W
\end{equation*}
Le bande di frequenza vengono segnate usando come pedice una lettera

\subsubsection*{Colore}
Calcolo della differenza tra magnitudine apparente di una frequenza e un'altra; fisicamente mi dice se oggetto picca più su una frequenza o su un'altra. \'E legata alla temperatura

\section*{Calcolo della massa}
Per poterla stimare con questo metodo ho bisogno di un sistema binario (2 stelle che orbitano una intorno all'altra). Il piano dell'orbita lo poniamo per semplicità ortogonale a piano vista $\rightarrow$ percorrono circonferenza nel cielo. \newline
Per definizione centro di massa
\begin{equation*}
    r_1M_1 = r_2M_2
\end{equation*}
Chiamo la separazione tra le 2 stelle $a = r_1+r_2$ \newline
Applico legge di Keplero
\begin{equation*}
    w^2 = \frac{G(M_1+M_2)}{a^3}
\end{equation*}
Tramite separazione angolare $\theta$ che posso calcolare perchè vedo le 2 stelle 
$\begin{cases*}
    r_1 = \theta_1 d \\
    r_2 = \theta_2 d
\end{cases*} 
\rightarrow \frac{r_1}{r_2} = \frac{\theta_1}{\theta2}= \frac{M_1}{M_2}$ \newline
Se riesco a misurarela distanza tra me e sistema con leggi Keplero posso ottenere massa totale del sistema $\Rightarrow$ ho un sistema con 2 equazioni e 2 incognite

\subsection*{Spectroscopy bynarie}
Deduco presenze di binarie da studio dello spettro perchè risoluzione non mi permette di vedere le due stelle separate (esempio da effetto Doppler dello spettro di assorbimento)

\chapter{Le stelle}
Stelle con masse inferiori hanno luminosità inferiore
\begin{equation*}
    L = 4 \pi R^2 \sigma T^4 
\end{equation*}
Esempio: $L_{sole} \simeq 4 \times 10^33 \, erg \cdot s{-1}$

\section{Diagramma di Hertzsprung-Russell}

Anche detto diagramma magnitudine-colore. Si usa per stelle di cui conosco la distanza. \newline
Asse x: colore B-V; più mi trovo verso sinistra, più picco sul blu. Temperatura cresce verso sinistra \newline
Asse y: magnitudine assoluta nel V; valori più bassi hanno luminosità più alta \newline
La maggior parte delle stelle si trova sulla diagonale principale $\rightarrow$ SEQUENZA PRINCIPALE (racchiude le stelle che bruciano H in He). Qui si trova anche il sole, più o meno al centro. La maggior parte della loro vita le stelle la passano sulla sequenza principale (sole $\sim 10$ miliardi di anni) \newline
Diagonale minore: RAMO DELLE GIGANTI. Si trovano stelle che hanno raggi decisamente più grandi di quelle che si trovano sulla sequenza prinicipale. Vediamo meno stelle sul ramo delle giganti, perchè le stelle passano molto meno tempo in questa fase, e quindi è statisticamente più difficile osservarle \newline
In basso a sinistra si trovano le NANE BIANCHE: sono l'unico oggetto compatto visibile su diagramma H-R; hanno raggi decisamente più piccoli di quelle sulla sequenza principale \newline
La massa (indicata con unità di massa solare) andando verso l'alto a sinistra. Sulla sequenza principale massa più grande indica raggi più grandi. La vita lungo sequenza prinicipale diminuisce andando verso alto a sinistra: questo perchè la luminosità è proporzionale a $M^\alpha$ con $\alpha>1$, quindi maggiore è la massa, maggiore è la lumiosità, quindi muore prima

\section{Classificazione delle stelle}
La classificazione delle stelle avviene in base a spettro di assorbimento: si associano lettere a spettro di assorbimento (OBAFGKM). La "O" è associata ha temperature alte $\sim 40000 K$, mentre la "M" è associata a tempertaure minori $\sim 2400 K$. Ad esempio il sole è di tipo "G". Esiste sottoclassificazione per le temperature: si indica con un numero. Per indicare la posizione della stella nel diagramma si usa un numero romano (V indica che la stella si trova nella sequenza principale)

\section{Descrizione struttura stellare}

Sulla sequenza principale le stelle fanno fusione nucleare per convertire H in He. Per il sole l'efficienza di questo processo è di circa lo $0.7\%$ 

\noindent Composizione massa del sole $\begin{cases}
    71 \% H \\
    27 \% He \\
    2 \% \text{ C,N,O,Fe: metalli (così indico tutti gli elementi più pesanti dell'elio)}
\end{cases}$ 
Assunzioni:
\begin{enumerate}
    \item Le stelle hanno simmetria sferica: grandezze dipendono solo da r, non da $\theta,\phi,t$
    \item Le stelle non ruotano
    \item No campo magnetico
\end{enumerate}

\subsection{Equazione di equilibrio idrostatico}

Equazione per descrivere la pressione in funzione del raggio \newline
Forza gravitazionale è uguale alla forza dovuta al gradiente di pressione che bilancia la forza gravitazionale
\begin{equation*}
    -\frac{GM(r)dm}{r^2} = dP A
\end{equation*}
Considerando che $dm = \rho(r)A dr$ dove $\rho$ è la densità si ottiene
\begin{equation}
    -\frac{GM(r)\rho(r)A dr}{r^2} = dP A
\end{equation}
che ci porta a scrivere la prima equazione di evoluzione stellare, che però non può essere risolta da sola, perchè non ho espressione di $M(r), \rho(r)$
\begin{equation*}
   \frac{dP}{dr} -\frac{GM(r)\rho(r)}{r^2}
\end{equation*}

\subsection{Equazione di conservazione della massa}

Equazione per descrivere la massa in funzione del raggio \newline
Considerando che 
\begin{equation*}
    dM = \rho(r) 4 \pi r^2 dr
\end{equation*}
posso scrivere la seconda equazione di evoluzione stellare
\begin{equation}
    \frac{dM}{dr} = \rho(r) 4 \pi r^2
\end{equation}

\subsection{Teorema del viriale}
\begin{equation*}
    \int_0^{r_*} dr \, 4 \pi r^3 \frac{dP}{dr} = -\int_0^{r_*} dr \, 4 \pi r^3 \frac{GM(r)\rho(r)}{r^2}
\end{equation*}
\begin{equation*}
    \int_0^{r_*} dr \, 4 \pi r^3 \frac{dP}{dr} = - 3 \int_0^{r_*} dr \, 4 \pi r^2 P(r)
\end{equation*}
\begin{equation*}
    -\int_0^{r_*} dr \, 4 \pi r^3 \frac{GM(r)\rho(r)}{r^2} = -\int_0^{r_*} dr \, \frac{GM(r)dM}{rdr} = -\int_0^{M_*} \frac{GM(r)dM}{r} = E_{\mbox{gravitazionale}} 
\end{equation*}
Allora 
\begin{equation*}
    - 3 \int_0^{r_*} dr \, 4 \pi r^2 P(r) = E_{\mbox{gr}}
\end{equation*}
assumendo che la stella sia fatta di gas perfetto e che $\Gamma$ sia costante, considerando la densità di energia interna (e)
\begin{equation*}
    P = (\Gamma - 1)e
\end{equation*}
si ottiene 
\begin{equation*}
    -3 (\Gamma - 1) E_{\mbox{termica}} = E_{gr}
\end{equation*}
\begin{itemize}
    \item gas monoatomico classico, senza effetti quantistici
    \begin{equation*}
        P = n K_B T \Rightarrow e = \frac{3}{2}n K_B T = \frac{P}{(\Gamma - 1)}
    \end{equation*}
    si ottiene
    \begin{equation*}
        \Gamma - 1 = \frac{2}{3} \Rightarrow \Gamma = \frac{5}{3}
    \end{equation*}
    sostituendo nel teorema del viriale
    \begin{equation*}
        -2 E_{TH} = E_{GR}
    \end{equation*}
    L'energia totale è data dell'energia gravitazionale + energia termica
    \begin{equation*}
        E_{TOT} = E_{TH} + E_{GR} = -E_{TH} < 0
    \end{equation*}
    allora stella è sistema legato: se viene perturbata la stella oscilla attorno al suo punto di equilibrio. Conseguenza: le reazioni nucleari sono stabili $\rightarrow$ se energia termica aumenta per perturbazione allora la pressione fa espandere la stella; allora diminuisce l'energia gravitazionale che fa diminuire energia termica, quindi si torna ad equilibrio $\Rightarrow$ equilibrio stabile (torna a situazione iniziale)
    \item gas stella di tipo relativistico $\rightarrow$ pressione stella è dovuta principalmente a radiazione (fotoni)
    \begin{equation*}
        p = \frac{1}{3} e \Rightarrow \Gamma = \frac{4}{3}
    \end{equation*}
    si ottiene quindi 
    \begin{equation*}
        -E_{TH} = E_{GR}
    \end{equation*}
    di conseguenza l'energia totale è nulla: ciò significa che sistema è marginalmente legato $\rightarrow$ la minima perturbazione fa esplodere la stella
\end{itemize}

\chapter{Evoluzione stellare}

$M = 15 M_{sole}$: 
\begin{itemize}
    \item brucia H in He;  $10^7 yr$ sulla sequenza principale
    \item $He \rightarrow C,O,Ne \Rightarrow$ passa $10^6 yr$ a bruciare He: passa nel ramo delle giganti
    \item $C \rightarrow Ne,Mg, Na \Rightarrow 10^3 \mbox{yr}$ 
    \item $Ne \rightarrow O,Mg \Rightarrow 0.7 \mbox{yr}$
    \item $O \rightarrow Si,Ar,Ca \Rightarrow 2.6 \mbox{yr}$
    \item $Si\rightarrow Fe, Ni \Rightarrow 18 \mbox{days}$
\end{itemize}
\begin{enumerate}
    \item FOTODISINTEGRAZIONE 
    \begin{equation*}
        \gamma + ^{56}Fe \rightarrow 13 \,  ^4He + 4n 
    \end{equation*}
    \begin{equation*}
        \gamma + ^4He \rightarrow 2p + 2n 
    \end{equation*}
    \item NEUTRONIZZAZIONE (inverse $\beta$ decay)
    \begin{equation*}
        e^- + p \rightarrow n + \nu_e
    \end{equation*}
    $\rho \sim 10^{11} gcm^{-3} \Rightarrow n_e=n_p = \frac{1}{10}n_n$ 
\end{enumerate}
\begin{itemize}
    \item $8 < M < 20M_{sole} \Rightarrow$ stella di neutroni
    \item $M > 20 M_{sole} \Rightarrow$ buco nero
\end{itemize}
In entrambi i casi si ha prima un'esplosione di supernova; se la massa della stella però troppo massiva si può avere direttamente un buco nero senza avere l'esplosione di supernova. Durante queste esplosioni avviene rilascio degli stati più esterni, esplosione è quindi fondamentale anche per la nostra vita.

\end{document}