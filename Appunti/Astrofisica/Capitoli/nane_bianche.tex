\chapter{Nane bianche}

\begin{equation*}
    \frac{dN}{d^3pdV} = \frac{2}{h^3} \frac{1}{e^{\frac{E-\mu(t)}{K_BT}}+1}
\end{equation*}
Elettrone che va sbattere sulla superficie della nana bianca
\begin{gather*}
    dF_x = \frac{2p_x}{dt} \\
    \frac{dF_x}{dA} = \frac{2p_x}{dAdt}\frac{dx}{dx} = \frac{2p_x}{dAdx}\frac{dx}{dt} = \frac{2p_x}{dV}v_x
\end{gather*}
\begin{equation*}
    P = \int \frac{2p_xv_x}{dV} \frac{dN}{d^3p}d^3p \underbrace{\frac{1}{2}}_{\mbox{fattore correttivo}} \tag{a}
\end{equation*}
caso non relativistico
\begin{gather*}
    v^2 = v_x^2 + v^2_y v^2_z = 3v^2_x  \\
    p_xv_x = mv^2_x = \frac{1}{3}mv^2 = \frac{1}{3}pv   
\end{gather*}
allora la (a) diventa
\begin{equation*}
    P = \int \frac{1}{3}pv \frac{dN}{dVd^3p}d^3p
\end{equation*}
Calcolo la soluzione in 2 casi limiti
\begin{enumerate}
    \item \(v<<c\)
    \begin{equation*}
        P = \frac{1}{3}\int \frac{2}{h^3}pv d^3p = \frac{1}{3}\int_0^{p_f}\frac{2}{h^3}pv 4\pi p^2dp = \frac{1}{3}\int_0^{p_f} \frac{8\pi}{h^3}p\frac{p}{m_e}p^2dp = \frac{8\pi}{3h^3m_e}\int_0^{p_f}p^4dp = \frac{8\pi}{3h^3m_e}\frac{p^5_f}{5} \tag{b}
    \end{equation*}
    sapendo che 
    \begin{equation*}
        n = \frac{8\pi}{3h^3}p^3_f 
    \end{equation*}
    (b) diventa
    \begin{equation*}
        P = {\left(\frac{3}{8\pi}\right)}^{\frac{2}{3}}\frac{h^2}{5m_e}n^{\frac{5}{3}} \tag{c}
    \end{equation*}
    usando 
    \begin{equation*}
        n= \frac{Z}{A}\frac{\rho}{m_p}
    \end{equation*}
    (c) diventa
    \begin{equation*}
        P = {\left(\frac{3}{\pi}\right)}^{\frac{2}{3}} \frac{h^2}{20m_e{m_p}^{\frac{5}{3}}}{\left(\frac{Z}{A}\right)}^{\frac{5}{3}}\rho^{\frac{5}{3}}
    \end{equation*}
    se \(\rho \sim 10^6 g \,  cm^{-3} \Rightarrow P \simeq 10^{16}atm\) \newline
    Per i nuclei l'unica pressione è quella della temperatura, il gas è di tipo classico
    \begin{equation*}
        P_{TH} = n_N K_B T = \frac{\rho}{Am_p}K_BT
    \end{equation*}
    per \(^{12}C\) a \(T = 10^7K\) si ottiene
    \begin{equation*}
        P \simeq 10^{14} atm 
    \end{equation*}
    pressione termica per le nane bianche ha veramente un ruolo secondario, praticamente insignificante \newline
    \textit{Gas elettroni degenere non relativistico}
    \begin{equation*}
        P = K\rho^{\frac{5}{3}}
    \end{equation*}
    \item caso fortemente relativistico: tutti gli elettroni si stanno muovendo alla velocità della luce \([v = c]\); limite per densità che va all'infinito \((\rho \to \infty)\)
    \item \begin{equation*}
        P = \int_0^{p_f} \frac{2}{h^3}pc4\pi p^2 dp = \frac{8\pi c}{3h^3}\frac{{p_f}^4}{4}
    \end{equation*}
    facendo le stesse sostituzioni precedenti si ottiene l'espressione finale
    \begin{equation*}
        {\left(\frac{3}{8\pi}\right)}^{\frac{1}{3}}\frac{hc}{4{m_p}^{\frac{4}{3}}} {\left(\frac{Z}{A}\right)}^{\frac{4}{3}} \rho^{\frac{4}{3}}
    \end{equation*}
    quindi
    \begin{equation*}
        P = K' \rho^{\frac{4}{3}}
    \end{equation*}
    Equazione di stato politropica
    \begin{equation*}
        P = K \rho^\gamma
    \end{equation*}
\end{enumerate}
Caso non relativistico
\begin{equation*}
    \rho << 10^6 g \,  cm^{-3} \qquad \gamma = \frac{5}{3}
\end{equation*}
Caso fortemente relativistico
\begin{equation*}
    \rho \to \infty \qquad \gamma = \frac{4}{3}
\end{equation*}
Anche i neutroni sono fermioni, quindi avrei potuto fare lo stesso conto, sostituendo solo la massa elettroni con la massa dei neutroni \newline
Equazioni per la nana bianca 
\begin{gather*}
     \frac{dM}{dr} = 4\pi r^2 \rho \\
     \frac{dP}{dr} = - \frac{GM\rho}{r^2} \\
     P = Kp^\gamma
\end{gather*}
\begin{equation*}
    \frac{r^2}{\rho}\frac{dP}{dr}= -GM
\end{equation*}
derivo entrambi i membri per r
\begin{gather*}
    \frac{d}{dr}\left(\frac{r^2}{\rho}\frac{dP}{dr}\right) = -G \frac{dM}{dr} = -G 4\pi r^2 \rho  \\
    \frac{1}{r^2}\frac{d}{dr} \left(\frac{r^2}{\rho} \frac{dP}{dr}\right) = -4\pi G \rho
\end{gather*}
posso usare
\begin{equation*}
    P = K \rho^\gamma 
\end{equation*}
definisco
\begin{gather*}
    \gamma = 1 + \frac{1}{n}  \qquad \mbox{n: indice politropico}\\
    \theta^n = \frac{\rho}{\rho_c} \qquad \rho_c = \rho(r=0) \\
    \xi = \frac{r}{a} \qquad a = {\left[\frac{(n+1)K\rho_c^{\left(\frac{1}{n}-1\right)}}{4\pi G}\right]}^{\frac{1}{2}}
\end{gather*}
allora ottengo equazione di Lane-Emden
\begin{equation*}
    \frac{1}{\xi^2}\frac{d}{d\xi}\left(\xi^2 \frac{d\theta}{d\xi} = \theta^n\right)
\end{equation*}
anche questa equazione deve essere risolta numericamente. Condizioni iniziali
\begin{gather*}
    \theta (\xi=0) = 1 \\
    \theta'(\xi=0) = 0 
\end{gather*}
Posso fare ragionamento sulla massa
\begin{equation*}
    M = \int_0^R 4\pi \rho^2dr = 4\pi a^3\rho_c \int_0^{\xi_1} \xi^2\theta^n d\xi = -4\pi a^3 p_c \int_0^{\xi_1} \frac{d}{d\xi} \left(\xi^2 \frac{d\theta}{d\xi}\right) d\xi = 4\pi a^3 \rho_c \xi^2_1 |\theta' (\xi_1)| 
\end{equation*}
combinando con 
\begin{equation*}
    R = a \xi_1
\end{equation*}
si ottiene 
\begin{equation*}
    R \div M^{\frac{1-n}{3-n}}
\end{equation*}
caso non relativistico
\begin{equation*}
    \begin{cases}
        \gamma = \frac{5}{3} \\
        n = \frac{3}{2}
    \end{cases}
    \Rightarrow R \div M^{-\frac{1}{3}}
\end{equation*}
all'aumentare della massa diminuisce il raggio; allora ho un limite massimo della massa 
\begin{equation*}
    M = 4\pi {\left[\frac{(n+1)K}{4\pi G }\right]}^\frac{3}{2} \rho_c^{\frac{3-n}{2n}}\xi^2_1 |\theta'(\xi_1)|
\end{equation*}
Caso fortemente relativistico si ottiene il valore limite della massa: \textit{massa di Chandrasekhar}
\begin{equation*}
    \begin{cases}
        \gamma = \frac{4}{3} \\
        n = 3
    \end{cases}
    \Rightarrow M = 1.457 {\left(\frac{A}{2Z}\right)}^{-2}M_{\mbox{sole}}
\end{equation*}

