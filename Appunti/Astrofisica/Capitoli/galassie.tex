\chapter{Le galassie}

Le galassie sono formate da \(10^7-10^{11}\) stelle. 
Inizialmente sono state classificate da Hubble in base alla forma
\begin{enumerate}
    \item Ellittiche
    \item Spirali
\end{enumerate}
L'idea di Hubble era che le galassie nascono come ellittiche (EARLY TYPE GALAXY) e poi si evolvono come spirali (LATE TYPE GALAXY). 
In realtò sono più giovani le galassie spirali rispetto a quelle ellittiche. 
Per stabilire l'età di una galassia si guarda le stelle chela compongono: 
nell'ellittiche si vedono molte stelle di tipo rosso, mentre non vediamo quelle di tipo blu; dato che quando si forma una galassia si formano tutti i tipi di stelle, ciò significa che le stelle più massive sono già morte e non c'è idrogeno disponibile per formarne di nuove. \newline
Esistono anche  un altro tipo di galassie: le galassie peculiari/irregolari;
si crede che siano il risultato dell'interazione o della collisione di due galassie

\subsection*{Struttura di una galassia}

\begin{itemize}
    \item Bulge/sferoide: parte centrale più luminosa che tipicamente ospita stelle più vecchie (picca molto verso il rosso); dato che è così luminosa contiene la maggior parte della materia visibile
    \item Nucleo galattico: contiene un buco nero supermassivo; la nostra galassia ne ospita uno di circa \(10^6\) masse solari
    \item Disco: dove si trovano anche le spirali
    \item Alone stellare: formato da alcune stelle che si trovano fuori dal piano della galassia
    \item Globular cluster: ammassi di stelle fuori dal piano della galassia unite assieme dalla forza gravitazionale
    \item Raggi cosmici: elettroni, protoni e nuclei ad energie \(E \sim 10^8-10^{15} eV\)
    \item Materia oscura: alone oscuro che non emette radiazione elettromagnetica
\end{itemize}

\section{Parametri nostra galassia}

La Via Lattea ha una massa di circa \(10^{11}\) masse solari.
Il sole si trova ad una distanza di circa \(8.6 kpc\) dal centro, ed orbita ad una velocità di circa \(240 \frac{km}{s}\), con un periodo orbita, quindi, di circa \(10^8\) anni; 
si trova ad un'altezza rispetto al piano della galassia di circa \(25 pc\) \newline
\'E possibile calcolare la densità di stelle in una galassia 
\begin{equation*}
    \rho(r,z) = \rho_0 e^{-\frac{r}{r_d}}e^{-\frac{|z|}{h_d}}
\end{equation*}
si ottiene una densità media di 1 stella ogni parsec cubo

\section{Materia oscura}

\'E stata scoperta guardando le curve di rotazione delle galassie a spirale: per studiare la velocità di rotazione attorno al centro studio il blue-shift e il red-shift.
Nelle regioni più esterne le curve di rotazione sono tendenzialmente piatte: molto strano.
Posso pensare di studiare la velocità delle stelle del disco ipotizzando che tutta la massa sia nel centro, dato che vedo il centro molto più luminoso. 
Con le leggi di Keplero ottengo che 
\begin{equation*}
    v_{rot} \sim r^{-\frac{1}{2}}
\end{equation*}
Per avere quindi una velocità di rotazione costante devo quindi avere della massa extra che noi non vediamo: alone di materia oscura che genera forza gravitazione che tiene legate le stelle; 
ne percepiamo quindi gli effetti gravitazionali, ma non la vediamo