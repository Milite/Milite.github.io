\documentclass[a4paper,11pt]{report}

\usepackage[T1]{fontenc}
\usepackage[utf8]{inputenc}
\usepackage[italian]{babel}

\usepackage{mathdots}
\usepackage{mathtools}
\usepackage{graphicx}
\usepackage{amsfonts}
\usepackage{amsthm}
\usepackage{amsmath}
\usepackage{amssymb}
\usepackage{fancyhdr}
\usepackage{float}
\usepackage{geometry}
\geometry{a4paper, top=2.5cm, bottom=2cm, left=2cm, right=2cm}
\usepackage{hyperref}
\hypersetup{
	colorlinks=true,
	linkcolor=black,
	filecolor=blue,
	citecolor = black,      
	urlcolor=cyan,
}

\swapnumbers
\theoremstyle{remark}
\newtheorem*{oss}{Oss}
\newtheorem*{nb}{N.B}
\newtheorem*{coro}{Corollario}
\theoremstyle{definition}
\newtheorem*{teo}{Teorema}
\newtheorem*{Def}{Def}

\newcommand{\C}{\mathbb{C}}
\newcommand{\R}{\mathbb{R}}
\newcommand{\rot}{\mbox{rot}}
\newcommand{\grad}{\mbox{grad}}
\newcommand{\Div}{\mbox{div}}

\DeclarePairedDelimiter{\abs}{\lvert}{\rvert}

\begin{document}
	\date{}
	\author{Marco Militello}
	\title{Relatività}
	\maketitle
	\tableofcontents
	\newpage

\chapter{Richiami meccanica classica ed elettromagnetismo}

La meccanica di Newton si basa su 3 principi:
\begin{enumerate}
    \item In assenza di moto $\Rightarrow$ quiete o moto rettilineo uniforme
    \item $\frac{d}{dt}\vec{p} = F $ con $ \vec{p} = m\vec{v}$
    \item Principio di azione e reazione
\end{enumerate}
Se in un sistema S ho un moto rettilineo uniforme descritto da $\vec{x}(t) = \vec{x}_0 + \vec{u}t$ ed applico una trasformazione del tipo
\begin{equation*}
    \vec{x}' = \vec{x} + \vec{w}t^2
\end{equation*}
allora nel sistema S' avrò un moto accelerato descritto da $\vec{x}' = \vec{x}_0 + \vec{u}t + \vec{w}t^2$ \newline
SISTEMI DI RIFERMENTO INERZIALI (SDRI): sistemi in cui una particella di "test" (particella con massa e dimensioni trascurabili rispetto a quello a che sta intorno; non c'è perturbazione della misura) non soggetta a forza permane in stato di quiete o moto rettilineo uniforme \newline
Dato S che è SDRI e S' tale che 
\begin{equation*}
    \vec{x}' = \vec{x} - \vec{u}t
\end{equation*} 
dove $\vec{v}$ è la velocità relativa tra S e S', allora anche S' è SDRI \newline
Se faccio rotazione (che non dipenda dal tempo) allora permane il moto rettilineo \newline

\noindent PRINICIPIO DI RELATIVIT\'A: le leggi fisiche devono avere la stessa forma (es. F=ma deve diventare F'=ma') in tutti i SDRI; questo principio si basa su osservazioni empiriche. Enunciato in questo modo vale sia in meccanica classica, sia in relatività $\rightarrow$ cambia solo la trasformazione che uso \newline
COVARIANZA LEGGI FISICHE: significa invarianza in forma \newline
Sistema di rifermento $\rightarrow$ terna di assi cartesiasi e orologio (in fisica classica sono tutti sincronizzati) \newline
SDRI $ \rightarrow$ empiricamente sarà sistema inerziale in una certa regione di spazio, in un certo intervallo di tempo ed entro accuratezza delle misure che faccio

\section{Trasformazioni galileiane}

Costruite a partire da principio relatività con ipotesi del tempo unitario (t=t') \newline
Voglio trovare trasformazioni per passare da SDRI a SDRI del tipo
\begin{equation*}
    \begin{cases}
        t' = t'(t,x,y,z) \\
        x' = x'(t,x,y,z) \\
        y' = y'(t,x,y,z) \\
        z' = z'(t,x,y,z)
    \end{cases}
\end{equation*}
Nel sistema S descrivo con $\vec{x}_p(t) = \vec{x}_0 + \vec{u}t$: nello spazio è una retta $\Rightarrow$ in S' deve rimanere una retta, quindi deve essere una trasformazione lineare
\begin{equation*}
    \begin{cases}
        x' = a_{11}x +  a_{12}y +  a_{13}z +  a_{14}t \\
        x' = a_{21}x +  a_{22}y +  a_{23}z +  a_{24}t \\
        x' = a_{31}x +  a_{32}y +  a_{33}z +  a_{34}t \\
        x' = a_{41}x +  a_{42}y +  a_{43}z +  a_{44}t 
    \end{cases}
\end{equation*}
$a_{ij}(\vec{v})$ dipende da $\vec{v}$, ma non può dipendere da x,y,z,t altrimenti non sabbero trasformazioni lineari
\begin{itemize}
    \item asse $\hat{x}$ coincide con $\hat{x}'$ $\rightarrow$ y=z=0 $\Rightarrow$ y'=z'=0. Quindi: $a_{21}=a_{24}=a_{31}=a_{34}=0$
    \item piano xy deve coincidere con piano x'y' $\rightarrow z=0 \Rightarrow z'=0 $ Quindi: $a_{32}=0$
    \item piano xz deve coincidere con piano x'z' $\rightarrow y=0 \Rightarrow y'=0 $ Quindi: $a_{23}=0$
    \item Se ruoto asse x di 180° $\Rightarrow$ y va in -y e z in -z. Allora
    \begin{equation*}
        x' = a_{11}x +  a_{12}(-y)+  a_{13}(-z) +  a_{14}t
    \end{equation*}
    ma coordinata su x non deve cambiare su x'. Quindi $a_{12}=a_{13}=0$
    \item per simmetria cilindrica niente di particolare lungo asse y e z. Quindi $a_{22}=a_{33}$; di conseguenza anche $a_{43}=a_{42}$
    \item t non può dipendere da y e z perchè niente di speciale lungo y e z. Quindi $a_{42}=0$
\end{itemize}
Ottengo:
\begin{equation*}
    \begin{cases}
        x' = a_{11}x + a_{14}t \\
        y' = a_{22}y \\
        z' = a_{22}z \\
        t' = a_{41}x + a_{44}t
    \end{cases}
\end{equation*}
Trasformazioni devono dipendere al massimo da direzione moto $\rightarrow$ ISOTROPIA DELLO SPAZIO \newline
Riscrivendo:
\begin{equation*}
    \begin{cases}
        x' = Ax + Bt \\
        y' = Cy \\
        z' = Cz \\
        t' = \dots 
    \end{cases}
\end{equation*}
Moto in O' $\rightarrow$ se 
$\begin{cases}
    x'=0 \\
    y'=0 \\
    z'=0
\end{cases} 
\Rightarrow 
\begin{cases}
    0 = Ax + Bt  \Rightarrow x = - \frac{B}{A}t\\
    x = vt 
\end{cases}$ 
\newline Allora $x' = Ax +Bt$ diventa $x'=A(x-vt)$ [perchè $Bt = -Avt$] \newline
Usando ipotesi aggiuntiva del tempo assoluto ottengo:
\begin{equation*}
    \begin{cases}
        x' = A(v)(x -vt) \\
        y' = C(v)y \\
        z' = C(v)z \\
        t' = t 
    \end{cases}
\end{equation*}
\begin{itemize}
    \item se v=0 $\Rightarrow$ $A(v=0)=C(v=0)=1$
    \item per simmetria cilindrica $\Rightarrow$ $C(v)=C(-v)$
    \item $S \rightarrow S$ deve coincidere a $S \rightarrow S'$ se mando v in -v: $S \rightarrow S'$ stessa forma trasformazione, ma con velocità -v \newline
    \begin{equation*}\begin{cases}
        x = \frac{1}{A(v)}(x'+vt) \\
        y = \frac{1}{C(v)}y' \\
        z = \frac{1}{C(v)}z' \\
        t = t'
    \end{cases} \qquad 
    \begin{cases}
        x = A(-v)(x' +vt) \\
        y = C(-v)y' \\
        z = C(-v)z' \\
        t = t'
    \end{cases}
\end{equation*}
Allora ottengo:
\begin{equation*}
    \frac{1}{A(v)}= A(-v) \qquad \frac{1}{C(v)}= C(-v) \qquad v = A(-v)v
\end{equation*}
Quindi: $A(v)=1 $ e $C(v)=1$
\end{itemize}
Per un moto lungo asse x le trasformazioni di Galilei sono:
\begin{equation*}
    \begin{cases}
        x' = x -vt \\
        y' = y \\
        z' = z \\
        t' = t
    \end{cases}
\end{equation*}
In generale le trasformazioni di Galilei sono:
\begin{equation*}
    \begin{cases}
        \vec{x}' = \vec{x} - \vec{v}t \\
        t'=t
    \end{cases}
\end{equation*}
GRUPPO DI GALILEI: insieme trasformazioni di cui trasformazioni di Galilei fanno parte
\begin{itemize}
    \item traslazione rigida $\rightarrow \vec{x}' = \vec{x} + \vec{x}_0 \; t'=t + t_0$
    \item traslazione asse (no nel tempo) $\rightarrow \vec{x}' = R\vec{x}$ con R matrice di rotazione tale che $RR^T = R^TR = \mathbb{1}$
\end{itemize}

\chapter{Notazioni e formalismo (indici-vettori-operatori differenziali)}

\begin{enumerate}
    \item Convenzione di Einstein $\rightarrow$ indici ripetuti = indici sommati
    \begin{equation*}
        \vec{v} = \sum_{i=1}^3 v_i\vec{e}_i = v_i\vec{e}_i
    \end{equation*}
    \item Per vettori 4-dimensionali $\rightarrow (ct,x,y,z) = x^\mu$ con $\mu =0,1,2,3$
    \item $\delta_{ij}=
    \begin{cases}
        0 \quad i \ne j \\
        1 \quad i = j 
    \end{cases}$
    \item Simbolo di levi-civita (in 3D)
    \begin{equation*}
        \epsilon_{ijk} = \begin{cases}
            \epsilon_{123} = 1 \quad \text{per ogni permutazione pari di 1,2,3} \\
            \epsilon_{213} = -1 \quad \text{per ogni permutazione dispari di 1,2,3} \\
            \epsilon_{ii2} = 0 \quad \text{completamente asimmetrico}
        \end{cases}
    \end{equation*}
    $\epsilon_{ijk} = - \epsilon_{jik}$
\end{enumerate}

\section*{Matrici}
A,B matrici
\begin{equation*}
    {(AB)}_{ij} = A_{1k}B_{kj}
\end{equation*}
Per una matrice A $3 \times 3$ 
\begin{equation*}
    \det A = \epsilon_{ijk} a_{1i}a_{2j}a_{3k}
\end{equation*}

\begin{itemize}
    \item $\epsilon_{ijk}\epsilon_{ilm} = \delta_{il}\delta_{km} - \delta_{jm}\delta_{kl}$
    \item $\vec{a}\cdot \vec{b} = a_ib_i = \delta_{ij}a_ib_j$
    \item $\vec{a}\times \vec{b} = \epsilon_{ijk}a_jb_k\vec{e}_i$ \newline
    ${(\vec{a}\times \vec{b})}_i = \epsilon_{ijk}a_jb_k$
\end{itemize}

\section*{Contrazione di indici simmetrici/antisimmetrici}
$\begin{cases}
    \text{A: antisimmetrico su i e j} \rightarrow A_{ij}=-A_{ji} \\
    \text{S: simmetrico su i e j} \rightarrow S_{ij} = S_{ji}
\end{cases}$ \newline

\noindent Contrarre gli indici i e j significa sommare su i e j
\begin{equation*}
    \sum_{i,j=1}^3 A_{ij}S_{ij}\underbrace{=}_{\text{rename}}\sum_{i,j=1}^3 A_{ji}S_{ij} = \sum_{i,j=1}^3 -A_{ij}S_{ij} =0
\end{equation*}

\section{Operatori differenziali}

\begin{itemize}
    \item Gradiente
    \begin{equation*}
        \vec{\nabla}\phi = (\delta_i\phi)\vec{e}_i \qquad \delta_i\phi= \frac{\delta \phi}{\delta x_i}
    \end{equation*}
    \item Divergenza
    \begin{equation*}
        \vec{\nabla}\cdot\vec{V} = \delta_i V_i
    \end{equation*}
    \item Rotore 
    \begin{equation*}
        {(\vec{\nabla}\times \vec{V})} = \epsilon_{ijk}\delta_j V_k
    \end{equation*}
    \item Laplaciano
    \begin{equation*}
        \vec{\nabla}\cdot (\vec{\nabla}\phi) = \delta^2_x \phi + \delta^2_y \phi + \delta^2_z \phi 
    \end{equation*}
    \item D'Alambertiano
    \begin{equation*}
        \boxdot\phi = \frac{1}{c^2}\frac{\delta^2}{\delta t^2}\phi -\nabla^2\phi 
    \end{equation*}
\end{itemize}
Proprietà:
\begin{itemize}
    \item $\vec{a} \times \vec{b} = -\vec{b} \times \vec{a}$
    \item $\vec{a} \times (\vec{b} \times \vec{c}) = (\vec{a} \cdot \vec{c})\cdot \vec{b} - (\vec{a}\cdot \vec{b})\cdot \vec{c}$
    \item $\Div(\rot(\vec{V})) =0$
    \item $\rot(\grad(\phi)) = 0$
    \item $\rot(\rot(\vec{V})) = \grad(\Div(\vec{V}))- \nabla^2\vec{V}$
    \begin{equation*}
        {[\vec{\nabla} \times (\vec{\nabla} \times \vec{\nabla})]}_i = \delta_i(\vec{\nabla}\vec{V})-\nabla^2V_i
    \end{equation*}
\end{itemize}

\section*{Equazioni di Maxwell}

\begin{equation*}
    \begin{cases}
        \vec{\nabla} \cdot \vec{E} = 4\pi\rho \\
        \vec{B} \cdot \vec{B} = 0 \\
        \vec{\nabla} \times \vec{E} + \delta_t \vec{B} = 0 \\
        \vec{\nabla} \times \vec{B} - \delta_t \vec{E} = 4 \pi \vec{J}
    \end{cases} \quad \rightarrow \quad 
    \begin{cases}
        \delta_iE_i = 4\pi \rho  \qquad \qquad \text{M1: 1 equazione}\\
        \delta_iB_i = 0  \qquad \qquad \text{M2: 1 equazione}\\
        \epsilon_{ijk} \delta_jE_k + \delta_t B_i = 0 \quad \text{M3: 3 equazioni}\\
        \epsilon_{ijk} \delta_jB_k - \delta_t E_i = 4 \pi J_i \quad \text{M4: 3 equazioni}
    \end{cases}
\end{equation*}
Da M4 $\rightarrow$ applico divergenza
\begin{align*}
    \delta_i (\epsilon_{ijk} \delta_jB_k - \delta_t E_i) &= \delta_i 4 \pi J_i \\
    0 - \delta_t\delta_iE_i &= 4\pi \delta_iJ_i \\
    - 4\pi \delta_t\rho &= 4\pi (\vec{\nabla}\cdot \vec{J}) 
\end{align*}
Ottengo EQUAZIONE CONTINUIT\'A 
\begin{equation*}
    \frac{\delta}{\delta t}\rho + \vec{\nabla} \cdot \vec{J} = 0
\end{equation*}
Da M2 $\rightarrow$ poichè $\Div(\rot(\vec{A})) = 0$ posso ridefinire $\vec{B}$ come
\begin{equation*}
    \vec{B} = \vec{\nabla} \times \vec{A} \Rightarrow B_i = \epsilon_{ijk} \delta_j A_k
\end{equation*}
Da M3 $\rightarrow$ poichè
\begin{eqnarray*}
    \rot(\vec{E} + \delta_t \vec{A}) = 0 \\
    \rot(\grad(\phi)) = 0
\end{eqnarray*}
posso ridefinire $\vec{E}$ come
\begin{equation*}
    \vec{E} + \delta_t \vec{A} = -\vec{\nabla}\phi \Rightarrow \vec{E} = - \vec{\nabla} \phi - \frac{\delta}{\delta t} \vec{A}
\end{equation*}
Allora le equazioni di Maxwell

$\begin{cases}
    \delta_iE_i = 4\pi \rho  \\
    \epsilon_{ijk} \delta_jB_k - \delta_t E_i = 4 \pi J_i 
\end{cases}$ \newline

\noindent diventano
\begin{equation*}
    \begin{cases}
        4\pi \rho = \delta_iE_i = - \nabla^2 \phi - \delta_t (\delta_iA_i) \\
        4 \pi J_i = \epsilon_{ijk} \delta_j (\epsilon_{ijk} \delta_j A_k) - \delta_t (-\delta_i \phi - \delta_t A_i) = \delta_i\delta_j A_j - \nabla^2 A_i + \delta^2_t A_i + \delta_i\delta_t\phi
    \end{cases}
\end{equation*}
sommando $+ \delta_t^2 \phi - \delta_t^2 \phi$ alla prima equazione si ottiene:
\begin{equation*}
    \begin{cases}
        4 \pi \rho = - \nabla^2 \phi + \delta_t^2 \phi - \delta_t (\delta_t \phi + \vec{\nabla} \cdot \vec{A}) \\
        4 \pi J_i = \nabla^2 A_i + \delta_t^2 A_i +\delta_i (\delta_t \phi + \vec{\nabla} \cdot \vec{A})
    \end{cases}
\end{equation*}
Utilizzando operatore d'almenrtino 
\begin{equation*}
    \boxdot = \frac{\delta^2}{\delta t^2 } - \nabla^2
\end{equation*}
posso riscrivere le equazioni di Maxwell come
\begin{equation*}
    \begin{cases}
        \boxdot \phi - \delta_t (\delta_t \phi + \vec{\nabla} \cdot \vec{A}) = 4 \pi \rho \\
        \boxdot A_i - \delta_i (\delta_t \phi + \vec{\nabla} \cdot \vec{A}) = 4 \pi J_i
    \end{cases}
\end{equation*}

\section*{Trasformazioni di Gauge}

Considero $\psi(t, \vec{x})$ arbitraria
\begin{equation*}
    \begin{cases}
        \vec{A} \mapsto \vec{A}' = \vec{A} + \vec{\nabla}\psi \\
        \phi \mapsto \phi' = \phi - \delta_t \psi 
    \end{cases}
\end{equation*}
Usando queste trasformazioni i campi $\vec{E}, \vec{B}$ sono invarianti $\Rightarrow$ invarianza di Gauge \newline
Posso scegliere $\psi$ per semplificare le scelte iniziali di $\phi,\vec{A} \rightarrow$ scelte di Gauge
\begin{itemize}
    \item Lorenz: $\delta_t \phi + \vec{\nabla} \cdot \vec{A} = 0$
    \item Coulomb: $\vec{\nabla} \cdot \vec{A} = 0$
    \item Temporale: $\phi = 0$
    \item Radiazione: $\vec{\nabla} \cdot \vec{A} = 0 \; \phi = 0$ (in assenza di sorgenti)
\end{itemize}
Applicando scelta di Lorenz ottengo
\begin{equation*}
    \begin{cases}
        \boxdot \phi = 4 \pi \rho \\
        \boxdot \vec{A} = 4 \pi \vec{J}
    \end{cases}
\end{equation*}

\section{Spazio euclideo 3D}

\subsection*{Distanza infinitesima}
\begin{equation*}
    d\vec{x} = (dx,dy,dx) = (dx_1,dx_2,dx_3)
\end{equation*}
Distanza euclidea
\begin{equation*}
    \underbrace{{|d\vec{x}|}^2}_{\text{metrica}} = d\vec{x} \cdot d\vec{x} = dx^2 + dy^2 + dz^2 = \delta_{ij}x_ix_j \rightarrow \text{tensore metrico}
\end{equation*}

\subsection*{Rotazioni}
Trasformazioni lineai che lasciano invariante ${|d\vec{x}|}^2$
\begin{equation*}
    \begin{cases}
        x \mapsto x' \quad x'_i = R_{ij} x_j \rightarrow \text{ R non dipende da $\vec{x}$} \\
        d\vec{x} \mapsto d\vec{x}_i = R_{ij}dx_j
    \end{cases}
\end{equation*}
Impongo che $d\vec{x}'^2 = d\vec{x}^2$
\begin{equation*}
    {|d\vec{x}'|}^2 = d\vec{x}' \cdot d\vec{x}' = R_{ij} dx_j R_{ik} x_k = R_{ij}R_{ik}dx_jdx_k = {(R^T)}_{ij}R_{ik}dx_jdx_k = {(R^TR)}_{jk}dx_jdx_k = \delta_{jk}dx_jdx_k
\end{equation*}
Allora 
\begin{equation*}
    R^TR = \mathbb{I}_{3\times3} \qquad R \in O(3)
\end{equation*}
Quindi se $R\in O(3)$ allora ${|d\vec{x}'|}^2$ è invariante \newline
ROTAZIONI PROPRIE $\rightarrow R \in SO(3)$
\begin{equation*}
    R^TR = RR^T = \mathbb{I} \qquad \det{R} = +1
\end{equation*}
\begin{itemize}
    \item Passivo
    \begin{equation*}
        \left(\begin{array}[]{c}
            x' \\
            y' \\
            z'
        \end{array}\right) =
        \left(\begin{array}[]{ccc}
            \cos\theta & \sin\theta & 0 \\
            -\sin\theta & \cos\theta & 0 \\
            0 & 0 & 1
        \end{array}\right)
        \left(\begin{array}[]{c}
            x \\
            y \\
            z 
        \end{array}\right)
    \end{equation*}
    \item Attivo
    \begin{equation*}
        \left(\begin{array}[]{c}
            x' \\
            y' \\
            z'
        \end{array}\right) =
        \left(\begin{array}[]{ccc}
            \cos\theta & -\sin\theta & 0 \\
            \sin\theta & \cos\theta & 0 \\
            0 & 0 & 1
        \end{array}\right)
        \left(\begin{array}[]{c}
            x \\
            y \\
            z 
        \end{array}\right)
    \end{equation*}
\end{itemize}

\subsection*{Vettori}

$\vec{A}$ è un vettore $(A_x,A_y,A_z)$ se sotto trasformazioni trasforma come vettore
\begin{equation*}
    A_i \mapsto A'_i = R_{ij}A_j
\end{equation*}

\subsection*{Campo vettoriale}
$\vec{A}(x)$ è un campo vettoriale tale che 
\begin{equation*}
    A_i(x) \mapsto A'_i (\vec{x}') = R_{ij} A_j(\vec{x}) \qquad \vec{x}' = R \vec{x}
\end{equation*}

\subsection*{Tensore di rango m}

\begin{equation*}
    T_{i_1,\dots, i_n} \underbrace{\mapsto}_{\text{rotazione}} R_{i_1j_1}R_{i_2j_2}\cdots R_{i_nj_n}T_{j_1,\dots,j_n}
\end{equation*}
Un tensore di rango n contiene $3^n$ elementi. Un vettore è un caso particolare di tensore: un vettore è un tensore di rango 1

\subsection*{Campo tensoriale}
\begin{equation*}
    T_{i_1,\dots, i_n}(\vec{x})
\end{equation*}

\subsection*{Scalare}

Uno scalare è una quantità invariante sotto rotazione
\begin{itemize}
    \item $\vec{A} \cdot \vec{B} = A_iB_i$ è uno scalare $\rightarrow$ rappresenta l'angolo tra i due vettori
    \item $|\vec{A}| = \vec{A} \cdot \vec{A}$ è uno scalare
    \item Il prodotto tra due tensori con tutti gli indici contratti è uno scalare
\end{itemize}

\subsection*{Campo scalare}
\begin{equation*}
    \phi(\vec{x}) \qquad \phi'(\vec{x}) = \phi(\vec{x})
\end{equation*}
Un esempio è la temperatura

\section{Spazi di Riemann}
Uno spazio di Riemann è uno spazio N-dimensionale dotato di metrica che esprime la distanza infinitesima tra due punti nello spazio stesso
\begin{equation*}
    ds^2 = g_{ab}dx^adx^b \qquad a,b = 1,\dots, N
\end{equation*}
con $ds^2$ metrica e $g_{ab}$ tensore metrico, che i generale può dipendere dal punto x
\begin{itemize}
    \item g è invertibile $(\det{g}\neq 0)$
    \item g è simmetrico $(g_{ab}=g_{ba})$
    \item g non per forza diagonale
    \item $ds^2$ è invariante, non dipende dal sistema di coordinate
\end{itemize}
\begin{enumerate}
    \item Se $g_{ab}$ non dipende da x $\Rightarrow$ spazio piatto
    \item Se ho uno spazio piatto $\Rightarrow$ esiste sistema di coordinare in cui $g_{ab}$ è costante su tutto lo spazio
    \item Segnatura: denota il numero di autovalori positivi e negativi della matrice $g_{ab}$
    \item n = numero autovalori positivi e N = dimensione dello spazio
    \begin{equation*}
        \begin{cases}
            n = N \Rightarrow \text{Varietà riemanniana $\rightarrow$ metrica euclidea} \\
            \begin{cases}
                n = 1 \\
                n = N-1 
            \end{cases}
            \Rightarrow \text{Varietà pseudo riemanniana $\rightarrow$ metrica pseudo euclidea}
        \end{cases}
    \end{equation*}
    La metrica pseudo euclidea comprende anche lo spazio di Minkowski
\end{enumerate}

\section{Spazio di Minkowski}
\subsection*{Coordinate controvarianti}
\begin{equation*}
    x^\mu = (ct,x,y,z) = (x^0,x^1,x^2,x^3) = (x^0,x^i)
\end{equation*}
\begin{equation*}
    x = x^\mu \hat{e}_\mu
\end{equation*}

\subsection*{Struttura riemanniana}
\begin{equation*}
    M = \R^{1,3} \rightarrow \text{ Spazio di Riemann}
\end{equation*}
\begin{equation*}
    ds^2 = g_{\mu\nu} dx^\mu dx^\nu  \qquad g_{\mu\nu} = \mbox{diag}(+1,-1,-1,-1)
\end{equation*}
$ds^2$ non dipende da sistema di coordinate, mentre $g_{\mu\nu}$ (tensore metrico) non dipende da x. M è uno spazio piatto perchè il tensore metrico non dipende da x

\subsection*{Coordinate covarianti}

\end{document}