\chapter{Notazioni e formalismo (indici-vettori-operatori differenziali)}

\begin{enumerate}
    \item Convenzione di Einstein $\rightarrow$ indici ripetuti = indici sommati
    \begin{equation*}
        \vec{v} = \sum_{i=1}^3 v_i\vec{e}_i = v_i\vec{e}_i
    \end{equation*}
    \item Per vettori 4-dimensionali $\rightarrow (ct,x,y,z) = x^\mu$ con $\mu =0,1,2,3$
    \item $\partial_{ij}=
    \begin{cases}
        0 & i \ne j \\
        1 & i = j 
    \end{cases}$
    \item Simbolo di levi-civita (in 3D)
    \begin{equation*}
        \varepsilon_{ijk} = 
            \begin{cases}
                \varepsilon_{123} = 1 & \text{per ogni permutazione pari di 1,2,3} \\
                \varepsilon_{213} = -1 & \text{per ogni permutazione dispari di 1,2,3} \\
                \varepsilon_{ii2} = 0 & \text{completamente asimmetrico}
            \end{cases}
    \end{equation*}
    $\varepsilon_{ijk} = - \varepsilon_{jik}$
\end{enumerate}

\section*{Matrici}
A,B matrici
\begin{equation*}
    {(AB)}_{ij} = A_{1k}B_{kj}
\end{equation*}
Per una matrice A $3 \times 3$ 
\begin{equation*}
    \det A = \varepsilon_{ijk} a_{1i}a_{2j}a_{3k}
\end{equation*}

\begin{itemize}
    \item $\varepsilon_{ijk}\varepsilon_{ilm} = \partial_{il}\partial_{km} - \partial_{jm}\partial_{kl}$
    \item $\vec{a}\cdot \vec{b} = a_ib_i = \partial_{ij}a_ib_j$
    \item $\vec{a}\times \vec{b} = \varepsilon_{ijk}a_jb_k\vec{e}_i$ \newline
    ${(\vec{a}\times \vec{b})}_i = \varepsilon_{ijk}a_jb_k$
\end{itemize}

\section*{Contrazione di indici simmetrici/antisimmetrici}
$\begin{cases}
    \text{A: antisimmetrico su i e j} \rightarrow A_{ij}=-A_{ji} \\
    \text{S: simmetrico su i e j} \rightarrow S_{ij} = S_{ji}
\end{cases}$ \newline

\noindent Contrarre gli indici i e j significa sommare su i e j
\begin{equation*}
    \sum_{i,j=1}^3 A_{ij}S_{ij}\underbrace{=}_{\text{rename}}\sum_{i,j=1}^3 A_{ji}S_{ij} = \sum_{i,j=1}^3 -A_{ij}S_{ij} =0
\end{equation*}

\section{Operatori differenziali}

\begin{itemize}
    \item Gradiente
    \begin{equation*}
        \vec{\nabla}\phi = (\partial_i\phi)\vec{e}_i \qquad \partial_i\phi= \frac{\partial \phi}{\partial x_i}
    \end{equation*}
    \item Divergenza
    \begin{equation*}
        \vec{\nabla}\cdot\vec{V} = \partial_i V_i
    \end{equation*}
    \item Rotore 
    \begin{equation*}
        {(\vec{\nabla}\times \vec{V})} = \varepsilon_{ijk}\partial_j V_k
    \end{equation*}
    \item Laplaciano
    \begin{equation*}
        \vec{\nabla}\cdot (\vec{\nabla}\phi) = \partial^2_x \phi + \partial^2_y \phi + \partial^2_z \phi 
    \end{equation*}
    \item D'Alambertiano
    \begin{equation*}
        \Box\phi = \frac{1}{c^2}\frac{\partial^2}{\partial t^2}\phi -\nabla^2\phi 
    \end{equation*}
\end{itemize}
Proprietà:
\begin{itemize}
    \item $\vec{a} \times \vec{b} = -\vec{b} \times \vec{a}$
    \item $\vec{a} \times (\vec{b} \times \vec{c}) = (\vec{a} \cdot \vec{c})\cdot \vec{b} - (\vec{a}\cdot \vec{b})\cdot \vec{c}$
    \item $\Div(\rot(\vec{V})) =0$
    \item $\rot(\grad(\phi)) = 0$
    \item $\rot(\rot(\vec{V})) = \grad(\Div(\vec{V}))- \nabla^2\vec{V}$
    \begin{equation*}
        {[\vec{\nabla} \times (\vec{\nabla} \times \vec{\nabla})]}_i = \partial_i(\vec{\nabla}\vec{V})-\nabla^2V_i
    \end{equation*}
\end{itemize}

\section*{Equazioni di Maxwell}

\begin{equation*}
    \begin{cases}
        \vec{\nabla} \cdot \vec{E} = 4\pi\rho \\
        \vec{B} \cdot \vec{B} = 0 \\
        \vec{\nabla} \times \vec{E} + \partial_t \vec{B} = 0 \\
        \vec{\nabla} \times \vec{B} - \partial_t \vec{E} = 4 \pi \vec{J}
    \end{cases} \quad \rightarrow \quad 
    \begin{cases}
        \partial_i E_i = 4\pi \rho  & \text{M1: 1 equazione}\\
        \partial_i B_i = 0  & \text{M2: 1 equazione}\\
        \varepsilon_{ijk} \partial_j E_k + \partial_t B_i = 0 & \text{M3: 3 equazioni}\\
        \varepsilon_{ijk} \partial_j B_k - \partial_t E_i = 4 \pi J_i & \text{M4: 3 equazioni}
    \end{cases}
\end{equation*}
Da M4 $\rightarrow$ applico divergenza
\begin{gather*}
    \partial_i (\varepsilon_{ijk} \partial_j B_k - \partial_t E_i) = \partial_i 4 \pi J_i \\
    0 - \partial_t \partial_i E_i = 4\pi\partial_i J_i \\
    - 4\pi \partial_t \rho = 4\pi (\vec{\nabla}\cdot \vec{J}) \\
    \intertext{ottengo EQUAZIONE CONTINUIT\'A } 
    \frac{\partial}{\partial t}\rho + \vec{\nabla} \cdot \vec{J} = 0
\end{gather*}
\noindent Da M2 $\rightarrow$ poichè $\Div(\rot(\vec{A})) = 0$ posso ridefinire $\vec{B}$ come
\begin{equation*}
    \vec{B} = \vec{\nabla} \times \vec{A} \Rightarrow B_i = \varepsilon_{ijk} \partial_j A_k
\end{equation*}
Da M3 $\rightarrow$ poichè
\begin{gather*}
    \rot(\vec{E} + \partial_t \vec{A}) = 0 \\
    \rot(\grad(\phi)) = 0
\end{gather*}
posso ridefinire $\vec{E}$ come
\begin{equation*}
    \vec{E} + \partial_t \vec{A} = -\vec{\nabla}\phi \Rightarrow \vec{E} = - \vec{\nabla} \phi - \frac{\partial}{\partial t} \vec{A}
\end{equation*}
Allora le equazioni di Maxwell
\begin{align*}
    &\begin{cases}
        \partial_iE_i = 4\pi \rho  \\
        \varepsilon_{ijk} \partial_jB_k - \partial_t E_i = 4 \pi J_i 
    \end{cases} \\
    \intertext{diventano} 
    &\begin{cases}
        4\pi \rho = \partial_iE_i = - \nabla^2 \phi - \partial_t (\partial_iA_i) \\
        4 \pi J_i = \varepsilon_{ijk} \partial_j (\varepsilon_{ijk} \partial_j A_k) - \partial_t (-\partial_i \phi - \partial_t A_i) = \partial_i\partial_j A_j - \nabla^2 A_i + \partial^2_t A_i + \partial_i\partial_t\phi
    \end{cases}
\end{align*}
sommando $+ \partial_t^2 \phi - \partial_t^2 \phi$ alla prima equazione si ottiene:
\begin{equation*}
    \begin{cases}
        4 \pi \rho = - \nabla^2 \phi + \partial_t^2 \phi - \partial_t (\partial_t \phi + \vec{\nabla} \cdot \vec{A}) \\
        4 \pi J_i = \nabla^2 A_i + \partial_t^2 A_i +\partial_i (\partial_t \phi + \vec{\nabla} \cdot \vec{A})
    \end{cases}
\end{equation*}
Utilizzando operatore d'almenrtino 
\begin{equation*}
    \Box = \frac{\partial^2}{\partial t^2 } - \nabla^2
\end{equation*}
posso riscrivere le equazioni di Maxwell come
\begin{equation*}
    \boxed{
    \begin{cases}
        \Box \phi - \partial_t (\partial_t \phi + \vec{\nabla} \cdot \vec{A}) = 4 \pi \rho \\
        \Box A_i - \partial_i (\partial_t \phi + \vec{\nabla} \cdot \vec{A}) = 4 \pi J_i
    \end{cases}}
\end{equation*}

\section*{Trasformazioni di Gauge}

Considero $\psi(t, \vec{x})$ arbitraria
\begin{equation*}
    \begin{cases}
        \vec{A} \mapsto \vec{A}' = \vec{A} + \vec{\nabla}\psi \\
        \phi \mapsto \phi' = \phi - \partial_t \psi 
    \end{cases}
\end{equation*}
Usando queste trasformazioni i campi $\vec{E}, \vec{B}$ sono invarianti $\Rightarrow$ invarianza di Gauge \newline
Posso scegliere $\psi$ per semplificare le scelte iniziali di $\phi,\vec{A} \rightarrow$ scelte di Gauge
\begin{itemize}
    \item Lorenz: $\partial_t \phi + \vec{\nabla} \cdot \vec{A} = 0$
    \item Coulomb: $\vec{\nabla} \cdot \vec{A} = 0$
    \item Temporale: $\phi = 0$
    \item Radiazione: $\vec{\nabla} \cdot \vec{A} = 0 \; \phi = 0$ (in assenza di sorgenti)
\end{itemize}
Applicando scelta di Lorenz ottengo
\begin{equation*}
    \begin{cases}
        \Box \phi = 4 \pi \rho \\
        \Box \vec{A} = 4 \pi \vec{J}
    \end{cases}
\end{equation*}

\section{Spazio euclideo 3D}

\subsection*{Distanza infinitesima}
\begin{equation*}
    d\vec{x} = (dx,dy,dx) = (dx_1,dx_2,dx_3)
\end{equation*}
Distanza euclidea
\begin{equation*}
    \underbrace{{|d\vec{x}|}^2}_{\text{metrica}} = d\vec{x} \cdot d\vec{x} = dx^2 + dy^2 + dz^2 = \partial_{ij}x_ix_j \rightarrow \text{tensore metrico}
\end{equation*}

\subsection*{Rotazioni}
Trasformazioni lineai che lasciano invariante ${|d\vec{x}|}^2$
\begin{equation*}
    \begin{cases}
        x \mapsto x' \quad x'_i = R_{ij} x_j \rightarrow \text{ R non dipende da $\vec{x}$} \\
        d\vec{x} \mapsto d\vec{x}_i = R_{ij}dx_j
    \end{cases}
\end{equation*}
Impongo che $d\vec{x}'^2 = d\vec{x}^2$
\begin{equation*}
    {|d\vec{x}'|}^2 = d\vec{x}' \cdot d\vec{x}' = R_{ij} dx_j R_{ik} x_k = R_{ij}R_{ik}dx_jdx_k = {(R^T)}_{ij}R_{ik}dx_jdx_k = {(R^TR)}_{jk}dx_jdx_k = \partial_{jk}dx_jdx_k
\end{equation*}
Allora 
\begin{equation*}
    R^TR = \mathbb{I}_{3\times3} \qquad R \in O(3)
\end{equation*}
Quindi se $R\in O(3)$ allora ${|d\vec{x}'|}^2$ è invariante \newline
ROTAZIONI PROPRIE $\rightarrow R \in SO(3)$
\begin{equation*}
    R^TR = RR^T = \mathbb{I} \qquad \det{R} = +1
\end{equation*}
\begin{itemize}
    \item Passivo
    \begin{equation*}
        \begin{pmatrix}
                x' \\
                y' \\
                z'
        \end{pmatrix} =
        \begin{pmatrix}
            \cos\theta & \sin\theta & 0 \\
            -\sin\theta & \cos\theta & 0 \\
            0 & 0 & 1
        \end{pmatrix}
        \begin{pmatrix}
            x \\
            y \\
            z 
        \end{pmatrix}
    \end{equation*}
    \item Attivo
    \begin{equation*}
        \begin{pmatrix}
            x' \\
            y' \\
            z'
        \end{pmatrix} =
        \begin{pmatrix}
            \cos\theta & -\sin\theta & 0 \\
            \sin\theta & \cos\theta & 0 \\
            0 & 0 & 1
        \end{pmatrix}
        \begin{pmatrix}
            x \\
            y \\
            z 
        \end{pmatrix}
    \end{equation*}
\end{itemize}

\subsection*{Vettori}

$\vec{A}$ è un vettore $(A_x,A_y,A_z)$ se sotto trasformazioni trasforma come vettore
\begin{equation*}
    A_i \mapsto A'_i = R_{ij}A_j
\end{equation*}

\subsection*{Campo vettoriale}
$\vec{A}(x)$ è un campo vettoriale tale che 
\begin{equation*}
    A_i(x) \mapsto A'_i (\vec{x}') = R_{ij} A_j(\vec{x}) \qquad \vec{x}' = R \vec{x}
\end{equation*}

\subsection*{Tensore di rango m}

\begin{equation*}
    T_{i_1,\dots, i_n} \xrightarrow[rotazione]{} R_{i_1j_1}R_{i_2j_2}\dotsm R_{i_nj_n}T_{j_1,\dots,j_n}
\end{equation*}
Un tensore di rango n contiene $3^n$ elementi. Un vettore è un caso particolare di tensore: un vettore è un tensore di rango 1

\subsection*{Campo tensoriale}
\begin{equation*}
    T_{i_1,\dots, i_n}(\vec{x})
\end{equation*}

\subsection*{Scalare}

Uno scalare è una quantità invariante sotto rotazione
\begin{itemize}
    \item $\vec{A} \cdot \vec{B} = A_iB_i$ è uno scalare $\rightarrow$ rappresenta l'angolo tra i due vettori
    \item $|\vec{A}| = \vec{A} \cdot \vec{A}$ è uno scalare
    \item Il prodotto tra due tensori con tutti gli indici contratti è uno scalare
\end{itemize}

\subsection*{Campo scalare}
\begin{equation*}
    \phi(\vec{x}) \qquad \phi'(\vec{x}) = \phi(\vec{x})
\end{equation*}
Un esempio è la temperatura

\section{Spazi di Riemann}
Uno spazio di Riemann è uno spazio N-dimensionale dotato di metrica che esprime la distanza infinitesima tra due punti nello spazio stesso
\begin{equation*}
    ds^2 = g_{ab}dx^adx^b \qquad a,b = 1,\dots, N
\end{equation*}
con $ds^2$ metrica e $g_{ab}$ tensore metrico, che i generale può dipendere dal punto x
\begin{itemize}
    \item g è invertibile $(\det{g}\neq 0)$
    \item g è simmetrico $(g_{ab}=g_{ba})$
    \item g non per forza diagonale
    \item $ds^2$ è invariante, non dipende dal sistema di coordinate
\end{itemize}
\begin{enumerate}
    \item Se $g_{ab}$ non dipende da x $\Rightarrow$ spazio piatto
    \item Se ho uno spazio piatto $\Rightarrow$ esiste sistema di coordinare in cui $g_{ab}$ è costante su tutto lo spazio
    \item Segnatura: denota il numero di autovalori positivi e negativi della matrice $g_{ab}$
    \item n = numero autovalori positivi e N = dimensione dello spazio
    \begin{align*}
            &n = N && \text{Varietà riemanniana $\rightarrow$ metrica euclidea} \\
            &\left.
            \begin{aligned}
                &n = 1 \\
                &n = N-1 
            \end{aligned}
            \right\}
            && \text{Varietà pseudo riemanniana $\rightarrow$ metrica pseudo euclidea}
    \end{align*}
    La metrica pseudo euclidea comprende anche lo spazio di Minkowski
\end{enumerate}

\section{Spazio di Minkowski}
\subsection*{Coordinate controvarianti}
\begin{equation*}
    x^\mu = (ct,x,y,z) = (x^0,x^1,x^2,x^3) = (x^0,x^i)
\end{equation*}
\begin{equation*}
    x = x^\mu \hat{e}_\mu
\end{equation*}

\subsection*{Struttura riemanniana}
\begin{equation*}
    M = \R^{1,3} \rightarrow \text{ Spazio di Riemann}
\end{equation*}
\begin{equation*}
    ds^2 = g_{\mu\nu} dx^\mu dx^\nu  \qquad g_{\mu\nu} = \mbox{diag}(+1,-1,-1,-1)
\end{equation*}
$ds^2$ non dipende da sistema di coordinate, mentre $g_{\mu\nu}$ (tensore metrico) non dipende da x. M è uno spazio piatto perchè il tensore metrico non dipende da x

\subsection*{Coordinate covarianti}
