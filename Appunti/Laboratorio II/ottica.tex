\documentclass[a4paper,11pt]{report}

\usepackage[T1]{fontenc}
\usepackage[utf8]{inputenc}
\usepackage[italian]{babel}

\usepackage{mathdots}
\usepackage{mathtools}
\usepackage{graphicx}
\usepackage{amsfonts}
\usepackage{amsthm}
\usepackage{amsmath}
\usepackage{amssymb}
\usepackage{fancyhdr}
\usepackage{float}
\usepackage{geometry}
\geometry{a4paper, top=2.5cm, bottom=2cm, left=2cm, right=2cm}
\usepackage{hyperref}
\hypersetup{
	colorlinks=true,
	linkcolor=blue,
	filecolor=blue,
	citecolor = black,      
	urlcolor=cyan,
}

\begin{document}
	\date{}
	\author{Marco Militello}
	\title{Ottica geometrica}
	\maketitle
	\tableofcontents
	\newpage

\chapter{Ottica geometrica}

L'ottica geometrica è un modello che descrive i fenomeni legati alla propagazione della luce in termini di raggi.
Normalemente si dice che l'ottica geometrica è un approssimazione valida finchè le dimensioni in gioco sono maggiori della lunghezza d'onda

\section{Leggi dell'ottica geometrica}

\begin{enumerate}
    \item in un mezzo omogeneo la luce si propaga in linea retta
    \item legge della riflessione
    \item legge della rifrazione
\end{enumerate}
Quando un raggio luminoso incontra una disomogeneità, come è l'interfaccia tra un mezzo e un altro, si separa in due raggi: uno riflesso e uno trasmesso (rifratto)

\section{Riflessione e rifrazione}

Un raggio che incide su una superficie di separazione piana tra due mezzi differenti viene in parte trasmesso e in parte riflesso.
Raggio incidente, riflesso e rifratto giacciono tutti nello stesso piano, perpendicolare all'interfaccia.
La direzione dei raggi è identificata con gli angoli che questi fomano rispetto allla normale all'interfaccia stessa.

\subsection*{Legge della riflessione}

\begin{equation*}
    \theta_{incidente} = \theta_{riflesso}
\end{equation*}

\subsection*{Legge della riflessione}

Il rapporto tra i seni degli angoli di incidenza e di rifrazione dipende dai due mezzi materiali affacciati, questa caratteristica dei mezzi si chiama indice di rifrazione

\begin{equation*}
    n_1 \sin{\theta_{inc}} = n_2 \sin{\theta_{rifr}}
\end{equation*}

\noindent L'indice di rifrazione di un mezzo dipende dalla lunghezza d'onda della luce che lo attraversa, questo ha come conseguenza che l'angolo di rifrazione cambia a seconda del colore della luce incidente.
la dipendenza dell'indice di rifrazione dalla lunghezza d'onda è descritto dalla legge di Cauchy

\begin{equation*}
    n(\lambda) = A + \frac{B}{\lambda^2} + \frac{C}{\lambda^4} + \dots 
\end{equation*}

\noindent Con un prisma realizzato con un materiale sufficientemente dispersivo, un fascio di luce bianaca è scomposto nelle sue componeti cromatiche

\section{Riflessione totale}

Se $n_2 < n_1$ il raggio rifratto si allontana dalla normale. 
All'aumentare dell'angolo di incidenza l'angolo di rifrazione aumenta. 
Quando $\theta_{rifr} = 90$ il raggio rifratto scompare, questo succede se $\arcsin{\theta_{inc}} = \frac{n_2}{n_1}$

\section{Riflessione, trasmissione e assorbimento}

Attraverso un mezzo la luce è parzialmente assorbita e si definisce il coefficiente di trasmissione $k_\lambda$ come 

\begin{equation*}
    I_{trasmessa} = I_0\exp^{-k_\lambda d}
\end{equation*}

\noindent dove d è la distanza percorsa dal mezzo. \newline
Quando la luce è scomposta in un raggio riflesso e uno rifratto, si ha una suddivisione dell'intensità luminosa che è misurata dai coefficienti di riflessione e di trasmissione che misurano i rapporti tra le intensità delle tre onde.
Il valore di tali coefficienti dipende dall'indice di rifrazione e dallo stato di polarizzazione

\section{Principio di Fermat}

In generale la propagazione della luce attraverso mezzi differenti è ben descritta dal principio di Fermat: \newline
propagandosi da A a B la luce segue il percorso che richiede il minimo di tempo di percerrenza, ovvero il percorso che richiede il cammino ottico più breve.
Data una distanza d in un mezzo di indice di rifrazione n, il tempo di percorrenza è:

\begin{equation*}
    t = \frac{d}{w} = \frac{d \cdot n}{c} = \frac{\Delta}{c}
\end{equation*}

\noindent La velocità della luce differente nei due mezzi fa sì che il percorso più veloce sia quello dato dalla legge di rifrazione

\chapter{Ottica ondulatoria}

Una serie di fenomeni ottici non trova spiegazione nell'ottica geometrica, questi fenomeni (interferenza e diffrazione) sono dovuti al fatto che la luce sia un'onda

\section{Onde}

Le equazioni di Maxwell prevedono la possibilità che i campi elettromagnetici si propaghino nello spazio e nel tempo secondo le equazioni delle onde.
Un'onda è identificata da una direzione di propagazione e una legge che esprime l'evoluzione dei campi nel tempo e nello spazioe. 
La velocità di propagazioen è definita dalle leggi di Maxwell come:

\begin{equation*}
    v_{luce} = \frac{c}{n}    
\end{equation*}

\subsection*{Onda piana}

Il fronte d'onda è definito come l'insieme dei punti dello spazio in cui i campi, a t fissato, assumono lo stesso valore

\begin{itemize}
    \item $(x \pm v_{luce}t)$ = costante $\Rightarrow$ onda piana
    \item $(r \pm v_{luce}t)$ = costante $\Rightarrow$ onda sperica
\end{itemize}

\noindent I segni - o + indicano l'onda progressiva e regressiva \newline
Comunemente si considera il caso in cui la dipendenza dei campi dalle coordinate spazio-tempo sia di tipo sinusoidale

\begin{equation*}
    \vec{E}(x,t) = \vec{E}_0 \sin{(kx \pm \omega t)}
\end{equation*}

\noindent con $k = \frac{2\pi}{\lambda}$ numero d'onda, $\lambda$ lunghezza d'onda, $\omega = \frac{2\pi}{T}$ pulsazione, T periodo, $\nu = \frac{1}{T}$ frequenza e $v_{luce} = \nu \lambda$

\section{Polarizzazione}

Il vettore $\vec{E}$ è perpendicolare alla direzione di propagazione x e varia la sua ampiezza in modo sinusoidale sia nello spazio che nel tempo.
La direzione del vettore può cambiar anch'essa.
Parlo di onda polarizzata linearmente se la direzione di $\vec{E}$ si mantiene sempre fissa; parlo di onda polarizzata circolarmente se la direzione di $\vec{E}$ varia descrivendo uan circoferenza nel piano yz

\section{Intensità}

L'intensità I è la grandezza a cui l'occhio e la maggior parte degli strumenti ottici è sensibile.
Poichè le oscillazioni che caratterizzano l'onda hanno frequenze molto alte, la grandezza di riferimento è l'intensità media, che viene calcolata come media temporale

\section{Interferenza}

Considero due sorgenti puntiformi $S_A$ e $S_B$ che producono un'onda monocromatica con fronte d'onda sferico.
Per il principio di sovrapposizione il campo nel punto P è dovuto alla sovrapposizione dei campi dovuti alla due sorgenti; l'intesità dell'onda è proporzionale al quadrato del campo elettrico

\begin{equation*}
    I(P) = I_A + I_B + \underbrace{2\sqrt[]{I_AI_B} \cos{((k_AR_A - \omega_At)- (k_BR_B - \omega_Bt + \phi))}}_{termine di interferenza}
\end{equation*}

\noindent Il termine di inteferenza ha valore nullo tranne che in condizioni speciali:

\begin{itemize}
    \item la frequenza e la lunghezza d'onda delle due sorgenti è uguale
    \item lo sfasamento tra le due è costante nel tempo $\rightarrow \phi(t) = cost $ $\Rightarrow$ coerenza
\end{itemize}

\noindent Si ottiene quindi

\begin{equation*}
    I(P) = I_A + I_B + 2\sqrt[]{I_AI_B} \cos{(k(R_A - R_B) -\phi)}
\end{equation*}

\begin{itemize}
    \item massimi (onde in fase): $k(R_A - R_B) -\phi = 2n\pi$
    \item minimi (onde in opposizione di fase): $k(R_A - R_B) -\phi = (2n +1)\pi$
\end{itemize}

\noindent Il luogo dei massimi e dei minimi è un iperboloide di rotazione

\section{Monocromaticità e coerenza}

Il fenomeno di interferenza, spiegato solo ammettedo la natura ondulatoria della luce, è osservabile esclusivamente quando si usano sorgenti monocromatiche e coerenti tra di loro.
Una sorgente si dice monocromatica se emette radiazione di una sola lunghezza d'onda.
Questo avviene nel caso di processi di diseccitazione radiativa di un atomo o di una molecola eccitati o ionizzati: a un singolo processo corrisponde un'energia caratteristica $E_0$ e quindi una $\lambda_0$ (spesso chiamata riga).
La monocromaticità della radiazione emessa non è perfetta. 
Ogni riga di emissione ha una larghezza intrinseca per cui l'intensità della riga ha una distribuzione piccata su $\lambda_0$ e di laghezza $\Delta\lambda$.
Questa larghezza è determinata dal tempo caratteristico di dieccitazione, quindi è una proprietà della sorgente.
L'emissione è quindi più assimilabile a una sovrapposizione d onde sinusoidali, tutte in fase tra di loro, la più intensa di lunghezza d'onda $\lambda_0$. 
Questo insieme di sinusoidi, denominato pacchetto d'onda, non ha però una durata infinita nel tempo, la sua durata dipende dal tempo caratteristico, detto tempo di corenza.
il tempo di coerenza moltiplicato per la velocità della luce dà la lunghezza spaziale del pacchetto, detta lunghezza di coerenza..
In una lampada ad Hg come quella in laboratorio, la lunghezza di coerenza dovuta alla larghezza intrinseca della riga è di circa 50 cm (riga 541.6 nm)

\section{Laser}

Dispositivo in grado di emettere un fascio di luce coerente, generalemente monocromatica, e concentrata in un raggio rettilineo estremamente collimato.
La luminosità delle sorgenti laser è elevatissima a paragone di quella delle sorgenti luminose tradizionali

\section{Coerenza}

Due sorgenti distinte, pur emettendo sulla stessa lunghezza d'onda, non sono coerenti tra di loro, pertanto non possono essere utilizzate per osservare il fenomeno di interferenza.
Questo richiede infatti di:

\begin{itemize}
    \item sovrapporre raggi luminosi che provengano dalla stessa sorgente fisica
    \item far percorrere a questi raggi cammini ottici differenti, che non eccedano la lunghezza di coerenza della sorgente
\end{itemize}


\end{document}