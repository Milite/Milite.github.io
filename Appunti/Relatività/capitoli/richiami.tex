\chapter{Richiami meccanica classica ed elettromagnetismo}

La meccanica di Newton si basa su 3 principi:
\begin{enumerate}
    \item In assenza di moto $\Rightarrow$ quiete o moto rettilineo uniforme
    \item $\frac{d}{dt}\vec{p} = F $ con $ \vec{p} = m\vec{v}$
    \item Principio di azione e reazione
\end{enumerate}
Se in un sistema S ho un moto rettilineo uniforme descritto da $\vec{x}(t) = \vec{x}_0 + \vec{u}t$ ed applico una trasformazione del tipo
\begin{equation*}
    \vec{x}' = \vec{x} + \vec{w}t^2
\end{equation*}
allora nel sistema S' avrò un moto accelerato descritto da $\vec{x}' = \vec{x}_0 + \vec{u}t + \vec{w}t^2$ \newline
SISTEMI DI RIFERMENTO INERZIALI (SDRI): sistemi in cui una particella di "test" (particella con massa e dimensioni trascurabili rispetto a quello a che sta intorno; non c'è perturbazione della misura) non soggetta a forza permane in stato di quiete o moto rettilineo uniforme \newline
Dato S che è SDRI e S' tale che 
\begin{equation*}
    \vec{x}' = \vec{x} - \vec{u}t
\end{equation*} 
dove $\vec{v}$ è la velocità relativa tra S e S', allora anche S' è SDRI \newline
Se faccio rotazione (che non dipenda dal tempo) allora permane il moto rettilineo \newline

\noindent PRINICIPIO DI RELATIVIT\'A: le leggi fisiche devono avere la stessa forma (es. F=ma deve diventare F'=ma') in tutti i SDRI; questo principio si basa su osservazioni empiriche. Enunciato in questo modo vale sia in meccanica classica, sia in relatività $\rightarrow$ cambia solo la trasformazione che uso \newline
COVARIANZA LEGGI FISICHE: significa invarianza in forma \newline
Sistema di rifermento $\rightarrow$ terna di assi cartesiasi e orologio (in fisica classica sono tutti sincronizzati) \newline
SDRI $ \rightarrow$ empiricamente sarà sistema inerziale in una certa regione di spazio, in un certo intervallo di tempo ed entro accuratezza delle misure che faccio

\section{Trasformazioni galileiane}

Costruite a partire da principio relatività con ipotesi del tempo unitario (t=t') \newline
Voglio trovare trasformazioni per passare da SDRI a SDRI del tipo
\begin{equation*}
    \begin{cases}
        t' = t'(t,x,y,z) \\
        x' = x'(t,x,y,z) \\
        y' = y'(t,x,y,z) \\
        z' = z'(t,x,y,z)
    \end{cases}
\end{equation*}
Nel sistema S descrivo con $\vec{x}_p(t) = \vec{x}_0 + \vec{u}t$: nello spazio è una retta $\Rightarrow$ in S' deve rimanere una retta, quindi deve essere una trasformazione lineare
\begin{equation*}
    \begin{cases}
        x' = a_{11}x +  a_{12}y +  a_{13}z +  a_{14}t \\
        x' = a_{21}x +  a_{22}y +  a_{23}z +  a_{24}t \\
        x' = a_{31}x +  a_{32}y +  a_{33}z +  a_{34}t \\
        x' = a_{41}x +  a_{42}y +  a_{43}z +  a_{44}t 
    \end{cases}
\end{equation*}
$a_{ij}(\vec{v})$ dipende da $\vec{v}$, ma non può dipendere da x,y,z,t altrimenti non sabbero trasformazioni lineari
\begin{itemize}
    \item asse $\hat{x}$ coincide con $\hat{x}'$ $\rightarrow$ y=z=0 $\Rightarrow$ y'=z'=0. Quindi: $a_{21}=a_{24}=a_{31}=a_{34}=0$
    \item piano xy deve coincidere con piano x'y' $\rightarrow z=0 \Rightarrow z'=0 $ Quindi: $a_{32}=0$
    \item piano xz deve coincidere con piano x'z' $\rightarrow y=0 \Rightarrow y'=0 $ Quindi: $a_{23}=0$
    \item Se ruoto asse x di 180° $\Rightarrow$ y va in -y e z in -z. Allora
    \begin{equation*}
        x' = a_{11}x +  a_{12}(-y)+  a_{13}(-z) +  a_{14}t
    \end{equation*}
    ma coordinata su x non deve cambiare su x'. Quindi $a_{12}=a_{13}=0$
    \item per simmetria cilindrica niente di particolare lungo asse y e z. Quindi $a_{22}=a_{33}$; di conseguenza anche $a_{43}=a_{42}$
    \item t non può dipendere da y e z perchè niente di speciale lungo y e z. Quindi $a_{42}=0$
\end{itemize}
Ottengo:
\begin{equation*}
    \begin{cases}
        x' = a_{11}x + a_{14}t \\
        y' = a_{22}y \\
        z' = a_{22}z \\
        t' = a_{41}x + a_{44}t
    \end{cases}
\end{equation*}
Trasformazioni devono dipendere al massimo da direzione moto $\rightarrow$ ISOTROPIA DELLO SPAZIO \newline
Riscrivendo:
\begin{equation*}
    \begin{cases}
        x' = Ax + Bt \\
        y' = Cy \\
        z' = Cz \\
        t' = \dotsb 
    \end{cases}
\end{equation*}
Moto in O' $\rightarrow$ se 
$\begin{cases}
    x'=0 \\
    y'=0 \\
    z'=0
\end{cases} 
\Rightarrow 
\begin{cases}
    0 = Ax + Bt  \Rightarrow x = - \frac{B}{A}t\\
    x = vt 
\end{cases}$ 
\newline Allora $x' = Ax +Bt$ diventa $x'=A(x-vt)$ [perchè $Bt = -Avt$] \newline
Usando ipotesi aggiuntiva del tempo assoluto ottengo:
\begin{equation*}
    \begin{cases}
        x' = A(v)(x -vt) \\
        y' = C(v)y \\
        z' = C(v)z \\
        t' = t 
    \end{cases}
\end{equation*}
\begin{itemize}
    \item se v=0 $\Rightarrow$ $A(v=0)=C(v=0)=1$
    \item per simmetria cilindrica $\Rightarrow$ $C(v)=C(-v)$
    \item $S \rightarrow S$ deve coincidere a $S \rightarrow S'$ se mando v in -v: $S \rightarrow S'$ stessa forma trasformazione, ma con velocità -v \newline
    \begin{equation*}\begin{cases}
        x = \frac{1}{A(v)}(x'+vt) \\
        y = \frac{1}{C(v)}y' \\
        z = \frac{1}{C(v)}z' \\
        t = t'
    \end{cases} \qquad 
    \begin{cases}
        x = A(-v)(x' +vt) \\
        y = C(-v)y' \\
        z = C(-v)z' \\
        t = t'
    \end{cases}
\end{equation*}
Allora ottengo:
\begin{equation*}
    \frac{1}{A(v)}= A(-v) \qquad \frac{1}{C(v)}= C(-v) \qquad v = A(-v)v
\end{equation*}
Quindi: $A(v)=1 $ e $C(v)=1$
\end{itemize}
Per un moto lungo asse x le trasformazioni di Galilei sono:
\begin{equation*}
    \begin{cases}
        x' = x -vt \\
        y' = y \\
        z' = z \\
        t' = t
    \end{cases}
\end{equation*}
In generale le trasformazioni di Galilei sono:
\begin{equation*}
    \begin{cases}
        \vec{x}' = \vec{x} - \vec{v}t \\
        t'=t
    \end{cases}
\end{equation*}
GRUPPO DI GALILEI: insieme trasformazioni di cui trasformazioni di Galilei fanno parte
\begin{itemize}
    \item traslazione rigida $\rightarrow \vec{x}' = \vec{x} + \vec{x}_0 \; t'=t + t_0$
    \item traslazione asse (no nel tempo) $\rightarrow \vec{x}' = R\vec{x}$ con R matrice di rotazione tale che $RR^T = R^TR = \mathbb{1}$
\end{itemize}